\section{\Questions к главе \ref{chapter04}} %\label{chapter04-questions}

\todo ДОПОЛНИТЬ

\subsection*{Вариант 1}

\begin{questions}


\question[3] Какой вид программ обычно исполняется в непривилегированном режиме процессора?
\begin{solution}[1cm]
Пользовательские приложения.
\end{solution}

\question[3] Какие из нижеперечисленных сценариев подпадают под определение \emph{самомодифицирующийся код}:
\begin{choices}
    \choice программа читает один байт  секции кода,
    \choice программа изменяет один байт в секции данных,
    \choice программа читает один байт из секции данных,
    \correctchoice программа изменяет байт в секции кода?
\end{choices}

\question[3] Какой вид преобразования адресов специфичен только для систем виртуализации:
\begin{choices}
    \choice v2p,
    \correctchoice v2h,
    \choice p2h?
\end{choices}


\question[3] Перечислите отличия ДТ от компиляции с ЯВО, мешаюшие применению классических оптимизаций последнего.
\begin{solution}[4cm]
В входном языке ДТ отсутствуют имена переменных, границы блоков алгоритмов, нет разделения между кодом и данными.
\end{solution}

\question[3] Выберите правильные составляющие задачи <<code discovery>> (обнаружение кода) в ДТ:
\begin{choices}
    \choice поиск кода внутри исполняемого файла,
    \correctchoice поиск границ инструкций при работе двоичного транслятора,
    \choice     поиск границ инструкций при работе интерпретатора,
    \correctchoice различение гостевого кода от гостевых данных,
    \choice     декодирование гостевых инструкций,
    \choice поиск некорректных гостевых инструкций.
\end{choices}


\end{questions}

\subsection*{Вариант 2}

\begin{questions}


\question[3] Какой тип инструкций наиболее сложен с точки зрения симуляции в режиме прямого исполнения:
\begin{choices}
    \choice арифметические,
    \correctchoice привилегированные,
    \choice с плавающей запятой,
    \choice условные и безусловные переходы?
\end{choices}

\question[3] Какой вид программ обычно исполняется в привилегированном режиме процессора?
\begin{solution}[1cm]
Операционные системы, мониторы виртуальных машин первого типа.
\end{solution}

\question[3] Определение понятия \emph{капсула}, используемого в двоичной трансляции.
\begin{solution}[1cm]
Блок хозяйского машинного кода, моделирующий одну конкретную гостевую инструкцию.
\end{solution}

\question[3] Какие порядки размеров капсул в системе двоичной трансляции наиболее вероятны:
\begin{choices}
    \choice 1 инструкция,
    \correctchoice 10 инструкций,
    \correctchoice 100 инструкций,
    \choice 1000 инструкций,
    \choice 10000 инструкций?
\end{choices}

\question[3] Выберите все необходимые условия корректности применения гиперсимуляции процессора:
\begin{choices}
    \correctchoice нет обращений к внешней памяти,
    \choice  нет обращений к внешним устройствам,
    \choice только один процессор в системе,
    \correctchoice состояние внешних устройств не меняется,
    \choice состояние процессора не меняется.
\end{choices}

% \question[3] Чем различаются понятия <<двоичный транслятор>> (ДТ) и <<бинарный транслятор>> (БТ)?
% \begin{choices}
%     \choice ДТ генерирует динамический код, а БТ --- статический
%     \correctchoice ничем, это синонимы
%     \choice БТ генерирует динамический код, а ДТ --- статический    
%     \choice     ДТ , а БТ --- статический
% \end{choices}

\end{questions}


% К каждой лекции должно быть от 8 до 12 задач, у каждой задачи должно быть 3-5 вариантов формулировок примерно одинаковой сложности. Допускается объединение нескольких последовательных лекций в одну тему и подготовка тестов к темам.
% Задачи должны полностью соответствовать материалам лекций, то есть лекциях должно быть достаточно информации для ответа на все вопросы.
% Формулировка каждого варианта задачи должна содержать всю необходимую информацию и не должна ссылаться на тексты внутри лекции, картинки или другие задачи или варианты задачи.
% Правильные ответы выделяются знаком «+» перед их формулировкой. Правильных ответов может быть несколько. Для тестов с несколькими ответами как минимум один ответ должен быть правильным и как минимум один ответ должен быть неправильным. 
% 
% Структура теста к лекции
% 
% \subsection*{Задача 1}
% 
% \paragraph{Вариант 1} 
% 
%     Чему равно 2+2?
%         Ответ 1. 3
%         + Ответ 2. 4
%         …
%         Ответ N. 5
% \paragraph{Вариант 2}
%     Чему равно 2*2?
%         + Ответ 1. 4
%         + Ответ 2. 2+2
%         …
%         Ответ N. 5
% \paragraph{Вариант 3}
% 
%     Чему равно 2-2?
%         Ответ 1. 0
% 
% 
%         
% \section{Просто подборка вопросов}
% 


 
 