% Copyright (c) 2016 Grigory Rechistov <grigory.rechistov@phystech.edu>
% This work is licensed under the Creative Commons Attribution-NonCommercial-ShareAlike 4.0 Worldwide.
% To view a copy of this license, visit http://creativecommons.org/licenses/by-nc-sa/4.0/.

% This file allows to produce either a separate PDF/PNG image
% See standalone documentation to understand underlying magic

\documentclass[tikz,convert={density=150,size=600,outext=.png}]{standalone}
\usetikzlibrary{shapes, calc, arrows, fit, positioning, decorations, patterns, decorations.pathreplacing, chains, snakes}
% Copyright (c) 2016 Grigory Rechistov <grigory.rechistov@phystech.edu>
% This work is licensed under the Creative Commons Attribution-NonCommercial-ShareAlike 4.0 Worldwide.
% To view a copy of this license, visit http://creativecommons.org/licenses/by-nc-sa/4.0/.

\usepackage{fontspec}
\usepackage{xunicode} % some extra unicode support
\usepackage{xltxtra}

\usepackage{amsfonts}
\usepackage{amsmath}
\usepackage{longtable}
\usepackage{csquotes}

\usepackage{polyglossia}
\setdefaultlanguage[spelling=modern]{russian} % for polyglossia
\setotherlanguage{english} % for polyglossia

% Common settings for all fonts
% 1. Attempt to make fonts be of the same size
% 2. Support TeX ligatures like — = emdash, << >> = guillemets
\defaultfontfeatures{Scale=MatchLowercase, Mapping=tex-text}

% Use Computer Modern Unicode
\newfontfamily\russianfont{CMU Serif}
\setromanfont{CMU Serif}
\setsansfont{CMU Sans Serif}
\setmonofont{CMU Typewriter Text}

% Copyright (c) 2016 Grigory Rechistov <grigory.rechistov@phystech.edu>
% This work is licensed under the Creative Commons Attribution-NonCommercial-ShareAlike 4.0 Worldwide.
% To view a copy of this license, visit http://creativecommons.org/licenses/by-nc-sa/4.0/.

% Common packages, commands and their configuration

\newcommand{\abbr}{\textit{англ.}\ }
\newcommand{\todo}[1][]{\textcolor{red}{TODO #1}}

\usepackage{graphicx}
\graphicspath{{pictures/}} % path to pictures, trailing slash is mandatory.

\usepackage{hyperref}
\hypersetup{colorlinks=true, linkcolor=black, filecolor=black, citecolor=black, urlcolor=black , pdfauthor=Grigory Rechistov <grigory.rechistov@phystech.edu>, pdftitle=Программное моделирование вычислительных систем}

\usepackage{footnpag}
\usepackage{indentfirst}
\usepackage{underscore}
\usepackage{url}

\usepackage{listings}
\lstset{basicstyle=\footnotesize\ttfamily, breaklines=true, keepspaces=true }

\usepackage{tikz}
\usetikzlibrary{shapes, calc, arrows, fit, positioning, decorations, patterns, decorations.pathreplacing, chains, snakes}
\usepackage{bytefield}

\usepackage{pgfplots} % Draw plots inside TeX
\pgfplotsset{compat=1.10}

\usepackage{standalone} % Clever way to build TikZ pictures either to PDF or to PNG


\graphicspath{{../pictures/}} % path to pictures, trailing slash is mandatory.

% The actual drawing follows
\begin{document}
    \begin{tikzpicture}[>=latex, node distance = 1.5cm,
        guest-page/.style={draw, shape=signal, signal pointer angle=160, signal from=north, signal to=south, text width = 1.5cm, node distance = 0.5cm},
        host-page/.style={draw, shape=tape, text width = 1.5cm, node distance = 0.5cm},
        ]
    \node[diamond, aspect=2, inner sep=0pt, draw, text width = 2.5cm, text badly centered] (first-exec) {Первое исполнение?};
    \node[below of=first-exec] (dummy) {};
    \node[left of= dummy, anchor=east] (gc) {Гостевой код};
    \node[right of=dummy, anchor=west] (hc) {Трасса} ;
    
    \node[below of= gc, style=guest-page] (gp01) {};
    \node[below of= hc, style=host-page] (hp01) {};
    \foreach \x/\y/\s in {1/2/,2/3/,3/4/thick,4/5/,5/6/,6/7/} {
        \node[below of= gp0\x, style=guest-page, \s] (gp0\y) {};
    };
    \foreach \x/\y/\s in {1/2/,2/3/,3/4/thick} {
        \node[below of= hp0\x, style=host-page, \s ] (hp0\y) {};
    };
    \node[arrow box, text width = 2.0cm, draw, arrow box arrows={west:.4cm, east:0.35cm}, left of=hp04, node distance = 2.5cm] {\scriptsize Интерпретация и трансляция};

    \draw[->] (first-exec.west) -| (gc.north) node[near start, above] {Да};
    \draw[->] (first-exec.east) -| (hc.north) node[near start, above] {Нет};
    
    \end{tikzpicture}


\end{document}
