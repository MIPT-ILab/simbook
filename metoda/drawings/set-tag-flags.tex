% This file allows to produce either a separate PDF/PNG image
% See standalone documentation to understand underlying magic

\documentclass[tikz,convert={density=150,size=600,outext=.png}]{standalone}
\usetikzlibrary{shapes, calc, arrows, fit, positioning, decorations, patterns, decorations.pathreplacing, chains, snakes}
% Copyright (c) 2016 Grigory Rechistov <grigory.rechistov@phystech.edu>
% This work is licensed under the Creative Commons Attribution-NonCommercial-ShareAlike 4.0 Worldwide.
% To view a copy of this license, visit http://creativecommons.org/licenses/by-nc-sa/4.0/.

\usepackage{fontspec}
\usepackage{xunicode} % some extra unicode support
\usepackage{xltxtra}

\usepackage{amsfonts}
\usepackage{amsmath}
\usepackage{longtable}
\usepackage{csquotes}

\usepackage{polyglossia}
\setdefaultlanguage[spelling=modern]{russian} % for polyglossia
\setotherlanguage{english} % for polyglossia

% Common settings for all fonts
% 1. Attempt to make fonts be of the same size
% 2. Support TeX ligatures like — = emdash, << >> = guillemets
\defaultfontfeatures{Scale=MatchLowercase, Mapping=tex-text}

% Use Computer Modern Unicode
\newfontfamily\russianfont{CMU Serif}
\setromanfont{CMU Serif}
\setsansfont{CMU Sans Serif}
\setmonofont{CMU Typewriter Text}

% Copyright (c) 2016 Grigory Rechistov <grigory.rechistov@phystech.edu>
% This work is licensed under the Creative Commons Attribution-NonCommercial-ShareAlike 4.0 Worldwide.
% To view a copy of this license, visit http://creativecommons.org/licenses/by-nc-sa/4.0/.

% Common packages, commands and their configuration

\newcommand{\abbr}{\textit{англ.}\ }
\newcommand{\todo}[1][]{\textcolor{red}{TODO #1}}

\usepackage{graphicx}
\graphicspath{{pictures/}} % path to pictures, trailing slash is mandatory.

\usepackage{hyperref}
\hypersetup{colorlinks=true, linkcolor=black, filecolor=black, citecolor=black, urlcolor=black , pdfauthor=Grigory Rechistov <grigory.rechistov@phystech.edu>, pdftitle=Программное моделирование вычислительных систем}

\usepackage{footnpag}
\usepackage{indentfirst}
\usepackage{underscore}
\usepackage{url}

\usepackage{listings}
\lstset{basicstyle=\footnotesize\ttfamily, breaklines=true, keepspaces=true }

\usepackage{tikz}
\usetikzlibrary{shapes, calc, arrows, fit, positioning, decorations, patterns, decorations.pathreplacing, chains, snakes}
\usepackage{bytefield}

\usepackage{pgfplots} % Draw plots inside TeX
\pgfplotsset{compat=1.10}

\usepackage{standalone} % Clever way to build TikZ pictures either to PDF or to PNG


\graphicspath{{../pictures/}} % path to pictures, trailing slash is mandatory.

% The actual drawing follows
\begin{document}
\begin{tikzpicture}[>=latex]
    
\begin{scope} [start chain,every node/.style={, font=\huge\ttfamily, text height=0.9cm}, node distance=0.cm,, inner sep=0cm]
    \node [on chain] (0x) {0x};
    \node [on chain] (set) {a};
    \node [on chain] {b};
    \node [on chain] {c};
    \node [on chain] {d};
    \node [on chain] {e};
    \node [on chain] {f};
    \node [on chain] (tag) {1};
    \node [on chain] {2};
    \node [on chain] {3};
    \node [on chain] {4};
    \node [on chain] {5};
    \node [on chain] {6};
    \node [on chain] (offset) {7};
    \node [on chain] {8};
    \node [on chain] {9};
    \node [on chain] (lastbit) {0};
\end{scope}

\draw (set.north west) -- ([yshift=-2cm] set.north west) coordinate[pos=0.95] (set-left) coordinate[pos=0] (bit63);
\draw (tag.north west) -- ([yshift=-2cm] tag.north west) coordinate[pos=0.95] (tag-left) coordinate[pos=0] (bit40-39);
\draw (offset.north west) -- ([yshift=-2cm] offset.north west) coordinate[pos=0.95] (offset-left) coordinate[pos=0] (bit16-15);
\draw (lastbit.north east) -- ([yshift=-2cm] lastbit.north east) coordinate[pos=0.95] (offset-right) coordinate[pos=0] (bit0);
    
\draw[<->] (set-left) -- (tag-left) node[midway, above] {\small номер сета};
\draw[<->] (tag-left) -- (offset-left) node[midway, above] {\small номер тэга};
\draw[<->] (offset-left) -- (offset-right) node[midway, above] {\small смещение};
    
\node at ([xshift=0.25cm] bit63) {\small 63};    
\node at (bit40-39) {\small 40 39};    
\node at (bit16-15) {\small 16 15};    
\node at ([xshift=-0.25cm] bit0) {\small 0};    
    
\end{tikzpicture}


\end{document}
