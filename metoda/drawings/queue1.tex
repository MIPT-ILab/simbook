% This file allows to produce either a separate PDF/PNG image
% See standalone documentation to understand underlying magic

\documentclass[tikz,convert={density=150,size=600,outext=.png}]{standalone}
\usetikzlibrary{shapes, calc, arrows, fit, positioning, decorations, patterns, decorations.pathreplacing, chains, snakes}
% Copyright (c) 2016 Grigory Rechistov <grigory.rechistov@phystech.edu>
% This work is licensed under the Creative Commons Attribution-NonCommercial-ShareAlike 4.0 Worldwide.
% To view a copy of this license, visit http://creativecommons.org/licenses/by-nc-sa/4.0/.

\usepackage{fontspec}
\usepackage{xunicode} % some extra unicode support
\usepackage{xltxtra}

\usepackage{amsfonts}
\usepackage{amsmath}
\usepackage{longtable}
\usepackage{csquotes}

\usepackage{polyglossia}
\setdefaultlanguage[spelling=modern]{russian} % for polyglossia
\setotherlanguage{english} % for polyglossia

% Common settings for all fonts
% 1. Attempt to make fonts be of the same size
% 2. Support TeX ligatures like — = emdash, << >> = guillemets
\defaultfontfeatures{Scale=MatchLowercase, Mapping=tex-text}

% Use Computer Modern Unicode
\newfontfamily\russianfont{CMU Serif}
\setromanfont{CMU Serif}
\setsansfont{CMU Sans Serif}
\setmonofont{CMU Typewriter Text}

% Copyright (c) 2016 Grigory Rechistov <grigory.rechistov@phystech.edu>
% This work is licensed under the Creative Commons Attribution-NonCommercial-ShareAlike 4.0 Worldwide.
% To view a copy of this license, visit http://creativecommons.org/licenses/by-nc-sa/4.0/.

% Common packages, commands and their configuration

\newcommand{\abbr}{\textit{англ.}\ }
\newcommand{\todo}[1][]{\textcolor{red}{TODO #1}}

\usepackage{graphicx}
\graphicspath{{pictures/}} % path to pictures, trailing slash is mandatory.

\usepackage{hyperref}
\hypersetup{colorlinks=true, linkcolor=black, filecolor=black, citecolor=black, urlcolor=black , pdfauthor=Grigory Rechistov <grigory.rechistov@phystech.edu>, pdftitle=Программное моделирование вычислительных систем}

\usepackage{footnpag}
\usepackage{indentfirst}
\usepackage{underscore}
\usepackage{url}

\usepackage{listings}
\lstset{basicstyle=\footnotesize\ttfamily, breaklines=true, keepspaces=true }

\usepackage{tikz}
\usetikzlibrary{shapes, calc, arrows, fit, positioning, decorations, patterns, decorations.pathreplacing, chains, snakes}
\usepackage{bytefield}

\usepackage{pgfplots} % Draw plots inside TeX
\pgfplotsset{compat=1.10}

\usepackage{standalone} % Clever way to build TikZ pictures either to PDF or to PNG


\graphicspath{{../pictures/}} % path to pictures, trailing slash is mandatory.

% The actual drawing follows
\begin{document}
    \begin{tikzpicture}[>=latex, font=\scriptsize]
    \draw[->] (0,0) -- (10,0) node[pos=0.9, below] (sim-time) {Время};

    \foreach \x in { 1, 2, 3, 4, 5, 6, 7, 8, 9} { 
        \draw (\x,-0.15) -- (\x,0.15) node (tick\x) {};
    };
    \node[shape=dart, draw, shape border rotate=270 ] at (2, 0.5) (currevent) {};
    \node[draw, arrow box, arrow box arrows={north:.7cm}, text width=2.5cm, align=center, below = 0cm of currevent] (tsim) {Текущее симулируемое время};
    \node[align=center, above=0.2cm of currevent, text width=2cm]  {Обрабатываемое событие};
    
    \node[shape=dart, draw, shape border rotate=270 ] at (5, 0.5) (futureevent) {};
    \node[align=center, above=0.2cm of futureevent, text width=2cm] {Запланированное событие};
    
    \node[fill=black!10, shape=dart, draw, shape border rotate=270 ] at (7, 0.5) (newbornevent) {};
    \node[align=center, above=0.2cm of newbornevent, text width=1.5cm] {Новое событие};
    
    \node[shape=dart, draw, shape border rotate=270 ] at (9, 0.5) (futureevent2) {};
    \node[shape=dart, draw, shape border rotate=270, above=0cm of futureevent2 ] (futureevent3) {};
    
    \draw[dashed, ->] (currevent.south) .. controls +(1,-0.5) and +(-1,-0.5) .. (newbornevent.south); % node[below, pos=0.7, text width=3cm] {Обработка события порождает новое событие в будущем};
        
    \end{tikzpicture}


\end{document}
