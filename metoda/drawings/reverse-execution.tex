% This file allows to produce either a separate PDF/PNG image
% See standalone documentation to understand underlying magic

\documentclass[tikz,convert={density=150,size=600,outext=.png}]{standalone}
\usetikzlibrary{shapes, calc, arrows, fit, positioning, decorations, patterns, decorations.pathreplacing, chains, snakes}
% Copyright (c) 2016 Grigory Rechistov <grigory.rechistov@phystech.edu>
% This work is licensed under the Creative Commons Attribution-NonCommercial-ShareAlike 4.0 Worldwide.
% To view a copy of this license, visit http://creativecommons.org/licenses/by-nc-sa/4.0/.

\usepackage{fontspec}
\usepackage{xunicode} % some extra unicode support
\usepackage{xltxtra}

\usepackage{amsfonts}
\usepackage{amsmath}
\usepackage{longtable}
\usepackage{csquotes}

\usepackage{polyglossia}
\setdefaultlanguage[spelling=modern]{russian} % for polyglossia
\setotherlanguage{english} % for polyglossia

% Common settings for all fonts
% 1. Attempt to make fonts be of the same size
% 2. Support TeX ligatures like — = emdash, << >> = guillemets
\defaultfontfeatures{Scale=MatchLowercase, Mapping=tex-text}

% Use Computer Modern Unicode
\newfontfamily\russianfont{CMU Serif}
\setromanfont{CMU Serif}
\setsansfont{CMU Sans Serif}
\setmonofont{CMU Typewriter Text}

% Copyright (c) 2016 Grigory Rechistov <grigory.rechistov@phystech.edu>
% This work is licensed under the Creative Commons Attribution-NonCommercial-ShareAlike 4.0 Worldwide.
% To view a copy of this license, visit http://creativecommons.org/licenses/by-nc-sa/4.0/.

% Common packages, commands and their configuration

\newcommand{\abbr}{\textit{англ.}\ }
\newcommand{\todo}[1][]{\textcolor{red}{TODO #1}}

\usepackage{graphicx}
\graphicspath{{pictures/}} % path to pictures, trailing slash is mandatory.

\usepackage{hyperref}
\hypersetup{colorlinks=true, linkcolor=black, filecolor=black, citecolor=black, urlcolor=black , pdfauthor=Grigory Rechistov <grigory.rechistov@phystech.edu>, pdftitle=Программное моделирование вычислительных систем}

\usepackage{footnpag}
\usepackage{indentfirst}
\usepackage{underscore}
\usepackage{url}

\usepackage{listings}
\lstset{basicstyle=\footnotesize\ttfamily, breaklines=true, keepspaces=true }

\usepackage{tikz}
\usetikzlibrary{shapes, calc, arrows, fit, positioning, decorations, patterns, decorations.pathreplacing, chains, snakes}
\usepackage{bytefield}

\usepackage{pgfplots} % Draw plots inside TeX
\pgfplotsset{compat=1.10}

\usepackage{standalone} % Clever way to build TikZ pictures either to PDF or to PNG


\graphicspath{{../pictures/}} % path to pictures, trailing slash is mandatory.

% The actual drawing follows
\begin{document}
\begin{tikzpicture}[>=latex, font=\small]
    
    \draw[->] (0,0) -- (10,0) node[pos=0.9, below, font=\scriptsize, text width=1.5cm, align=left] {Симулируемое время};
    
    \draw (0,0.25) -- (0,-0.25) node[below] (tgamma) {$t^{\gamma}$};
    \draw (4,0.25) -- (4,-0.25) node[below] (tstar) {$t^{\star}$} coordinate[pos=0] (tstar-above);
    \draw[dashed] (8,0.25) -- (8,-0.25);
    
    \draw[<->] (0,0.2) -- (4,0.2) node[midway, above] (T) {$T$};
    
    \path (7,0) coordinate (currt-coord);
    \node[draw, single arrow, single arrow head extend = 0.1cm, inner xsep=2pt, inner ysep=0.25cm, shape border rotate=90, below=0cm of currt-coord, anchor=north] (currt) {};
    \node[below=0cmof currt] (t) {$t$};
    
    \path (6.,0) coordinate (newt-coord);
    \node[draw, single arrow, single arrow head extend = 0.1cm, inner xsep=2pt, inner ysep=0.25cm, shape border rotate=270, above=0cm of newt-coord, anchor=south] (newt) {};
    % \node[above=0cmof currt] (t) {$t$};
    
    \draw (currt-coord) -- ([yshift=1.5cm] currt-coord) coordinate[pos=0.9] (right-stick);
    \draw (newt-coord) --  ([yshift=1.5cm] newt-coord)  coordinate[pos=0.9] (left-stick);
    
    \draw[<->] (left-stick) -- (right-stick) node[midway, above] {$\Delta t$};
    
    \draw[->] (currt) .. controls +(-1,-1) and +(0,-1) .. (tstar) node[text width=3cm, align=flush left, midway, below, ] {1. Возвращение к ближайшей точке сохранения};
    
    \draw[->] (tstar-above) .. controls +(0.5,0.5) and +(-0.5,0.5) .. (newt) node[text width=2cm, align=flush left, midway, above, ] {2. Прямая симуляция};
    
    \node[below=of T, text width=3cm, align=flush center, inner sep=0pt] (comment) {Периодические точки сохранения};
    \draw[dotted] (tgamma) -- (comment);
    \draw[dotted] (tstar) -- (comment);
    
\end{tikzpicture}

\end{document}
