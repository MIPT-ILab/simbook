% This file allows to produce either a separate PDF/PNG image
% See standalone documentation to understand underlying magic

\documentclass[tikz,convert={density=150,size=600,outext=.png}]{standalone}
\usetikzlibrary{shapes, calc, arrows, fit, positioning, decorations, patterns, decorations.pathreplacing, chains, snakes}
% Copyright (c) 2016 Grigory Rechistov <grigory.rechistov@phystech.edu>
% This work is licensed under the Creative Commons Attribution-NonCommercial-ShareAlike 4.0 Worldwide.
% To view a copy of this license, visit http://creativecommons.org/licenses/by-nc-sa/4.0/.

\usepackage{fontspec}
\usepackage{xunicode} % some extra unicode support
\usepackage{xltxtra}

\usepackage{amsfonts}
\usepackage{amsmath}
\usepackage{longtable}
\usepackage{csquotes}

\usepackage{polyglossia}
\setdefaultlanguage[spelling=modern]{russian} % for polyglossia
\setotherlanguage{english} % for polyglossia

% Common settings for all fonts
% 1. Attempt to make fonts be of the same size
% 2. Support TeX ligatures like — = emdash, << >> = guillemets
\defaultfontfeatures{Scale=MatchLowercase, Mapping=tex-text}

% Use Computer Modern Unicode
\newfontfamily\russianfont{CMU Serif}
\setromanfont{CMU Serif}
\setsansfont{CMU Sans Serif}
\setmonofont{CMU Typewriter Text}

% Common packages, commands and their configuration

\newcommand{\abbr}{\textit{англ.}\ }
\newcommand{\todo}[1][]{\textcolor{red}{TODO #1}}


\usepackage{graphicx}
\graphicspath{{pictures/}} % path to pictures, trailing slash is mandatory.

\usepackage{hyperref}
\hypersetup{colorlinks=true, linkcolor=black, filecolor=black, citecolor=black, urlcolor=black , pdfauthor=Grigory Rechistov <grigory.rechistov@phystech.edu>, pdftitle=Программное моделирование вычислительных систем}

\usepackage{footnpag}
\usepackage{indentfirst}
\usepackage{underscore}
\usepackage{url}

\usepackage{listings}
\lstset{basicstyle=\footnotesize\ttfamily, breaklines=true, keepspaces=true }

\usepackage{listings}
\lstset{basicstyle=\footnotesize\ttfamily, breaklines=true, keepspaces=true}

\usepackage{tikz}
\usetikzlibrary{shapes, calc, arrows, fit, positioning, decorations, patterns, decorations.pathreplacing, chains, snakes}
\usepackage{bytefield}

\graphicspath{{../pictures/}} % path to pictures, trailing slash is mandatory.

% The actual drawing follows
\begin{document}
    \begin{tikzpicture}[font=\small, >=latex]
        \begin{scope}[start chain,node distance=5mm, every node/.style={on chain, draw},]
            \node []  (ref-start) {};
            \node []              {};
            \node []              {};
            \node []              {};
            \node []  (ref-true)  {};
            \node []              {};
            \node []              {};
            \node []  (ref-end)   {};
        \end{scope}
        
        \begin{scope}[start chain,node distance=5mm, every node/.style={on chain,join, draw}, every join/.style={->}]
            \node [below=of ref-start]  (test-start) {};
            \node []              {};
            \node []              {};
            \node []              {};
            \node [fill=black!10]  (test-end)  {};
        \end{scope}
        
        \coordinate  (pnt) at (barycentric cs:ref-start=0.5,test-start=0.5);              
        \node[thick, draw, left=0.5cm of pnt] (initial-state) {};    
        \draw[->] (initial-state) -- (ref-start);
        \draw[->] (initial-state) -- (test-start);
        
        \node[left=of ref-start] {Эталонная трасса};
        \node[left=of test-start] {Проверяемая смуляция};
        
        \node[align=center, below=1cm of initial-state, inner sep=1pt] (init-label) {Начальное состояние};
        \draw[dotted] (init-label) -- (initial-state);
        
        \node[draw, dashed, fit = (ref-true) (test-end)] (diffbox) {};
        \node[right = 0.5cm of init-label] (difflabel) {Расхождение состояний};
        \draw[dotted] (difflabel) -- (diffbox);
        
    \end{tikzpicture}


\end{document}
