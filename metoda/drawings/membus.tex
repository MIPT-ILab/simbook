% This file allows to produce either a separate PDF/PNG image
% See standalone documentation to understand underlying magic

\documentclass[tikz,convert={density=150,size=600,outext=.png}]{standalone}
\usetikzlibrary{shapes, calc, arrows, fit, positioning, decorations, patterns, decorations.pathreplacing, chains, snakes}
% Copyright (c) 2016 Grigory Rechistov <grigory.rechistov@phystech.edu>
% This work is licensed under the Creative Commons Attribution-NonCommercial-ShareAlike 4.0 Worldwide.
% To view a copy of this license, visit http://creativecommons.org/licenses/by-nc-sa/4.0/.

\usepackage{fontspec}
\usepackage{xunicode} % some extra unicode support
\usepackage{xltxtra}

\usepackage{amsfonts}
\usepackage{amsmath}
\usepackage{longtable}
\usepackage{csquotes}

\usepackage{polyglossia}
\setdefaultlanguage[spelling=modern]{russian} % for polyglossia
\setotherlanguage{english} % for polyglossia

% Common settings for all fonts
% 1. Attempt to make fonts be of the same size
% 2. Support TeX ligatures like — = emdash, << >> = guillemets
\defaultfontfeatures{Scale=MatchLowercase, Mapping=tex-text}

% Use Computer Modern Unicode
\newfontfamily\russianfont{CMU Serif}
\setromanfont{CMU Serif}
\setsansfont{CMU Sans Serif}
\setmonofont{CMU Typewriter Text}

% Copyright (c) 2016 Grigory Rechistov <grigory.rechistov@phystech.edu>
% This work is licensed under the Creative Commons Attribution-NonCommercial-ShareAlike 4.0 Worldwide.
% To view a copy of this license, visit http://creativecommons.org/licenses/by-nc-sa/4.0/.

% Common packages, commands and their configuration

\newcommand{\abbr}{\textit{англ.}\ }
\newcommand{\todo}[1][]{\textcolor{red}{TODO #1}}

\usepackage{graphicx}
\graphicspath{{pictures/}} % path to pictures, trailing slash is mandatory.

\usepackage{hyperref}
\hypersetup{colorlinks=true, linkcolor=black, filecolor=black, citecolor=black, urlcolor=black , pdfauthor=Grigory Rechistov <grigory.rechistov@phystech.edu>, pdftitle=Программное моделирование вычислительных систем}

\usepackage{footnpag}
\usepackage{indentfirst}
\usepackage{underscore}
\usepackage{url}

\usepackage{listings}
\lstset{basicstyle=\footnotesize\ttfamily, breaklines=true, keepspaces=true }

\usepackage{tikz}
\usetikzlibrary{shapes, calc, arrows, fit, positioning, decorations, patterns, decorations.pathreplacing, chains, snakes}
\usepackage{bytefield}

\usepackage{pgfplots} % Draw plots inside TeX
\pgfplotsset{compat=1.10}

\usepackage{standalone} % Clever way to build TikZ pictures either to PDF or to PNG


\graphicspath{{../pictures/}} % path to pictures, trailing slash is mandatory.

% The actual drawing follows
\begin{document}
\begin{tikzpicture}[inner xsep=1pt, inner ysep=0.2cm, node distance=0cm, font=\small,>=latex]
    \node[draw, text width = 0.8cm                       ] (data1) {};
    \node[draw, text width = 0.8cm, below =of data1.south] (data2) {};
    \node[draw, text width = 0.8cm, below =of data2.south] (data3) {};
    \node[draw, text width = 0.8cm, below =of data3.south] (data4) {};
    \node[draw, text width = 0.8cm, below =of data4.south] (data5) {};
    \node[very thick, draw, text width = 0.8cm, below =of data5.south] (data6) {};
    \node[draw, text width = 0.8cm, below =of data6.south] (data7) {};
    \node[draw, text width = 0.8cm, below =of data7.south] (data8) {};
    \node[draw, fit=(data1) (data8), inner sep=0.5cm] (bbox) {};
    
    \node[draw, text badly centered, text width = 2cm, node distance=4.5cm, left of=data4, ] (reader) {Читающее устройство};
    
    \draw[->] (data6.west) -- +(-2.,-0) node [pos=0.6, above] {32 бит} node[pos=0.6] {/} |- (reader.east)  node[pos=0.5, above] {Результат};
    \draw[->] (reader.north) |- (bbox.120) node [pos=0.6, above] {8 бит} node[pos=0.6] {/} node[pos=0.8, above] {Адрес};
    
\end{tikzpicture}

\end{document}
