% This file allows to produce either a separate PDF/PNG image
% See standalone documentation to understand underlying magic

\documentclass[tikz,convert={density=150,size=600,outext=.png}]{standalone}
\usetikzlibrary{shapes, calc, arrows, fit, positioning, decorations, patterns, decorations.pathreplacing, chains, snakes}
% Copyright (c) 2016 Grigory Rechistov <grigory.rechistov@phystech.edu>
% This work is licensed under the Creative Commons Attribution-NonCommercial-ShareAlike 4.0 Worldwide.
% To view a copy of this license, visit http://creativecommons.org/licenses/by-nc-sa/4.0/.

\usepackage{fontspec}
\usepackage{xunicode} % some extra unicode support
\usepackage{xltxtra}

\usepackage{amsfonts}
\usepackage{amsmath}
\usepackage{longtable}
\usepackage{csquotes}

\usepackage{polyglossia}
\setdefaultlanguage[spelling=modern]{russian} % for polyglossia
\setotherlanguage{english} % for polyglossia

% Common settings for all fonts
% 1. Attempt to make fonts be of the same size
% 2. Support TeX ligatures like — = emdash, << >> = guillemets
\defaultfontfeatures{Scale=MatchLowercase, Mapping=tex-text}

% Use Computer Modern Unicode
\newfontfamily\russianfont{CMU Serif}
\setromanfont{CMU Serif}
\setsansfont{CMU Sans Serif}
\setmonofont{CMU Typewriter Text}

% Copyright (c) 2016 Grigory Rechistov <grigory.rechistov@phystech.edu>
% This work is licensed under the Creative Commons Attribution-NonCommercial-ShareAlike 4.0 Worldwide.
% To view a copy of this license, visit http://creativecommons.org/licenses/by-nc-sa/4.0/.

% Common packages, commands and their configuration

\newcommand{\abbr}{\textit{англ.}\ }
\newcommand{\todo}[1][]{\textcolor{red}{TODO #1}}

\usepackage{graphicx}
\graphicspath{{pictures/}} % path to pictures, trailing slash is mandatory.

\usepackage{hyperref}
\hypersetup{colorlinks=true, linkcolor=black, filecolor=black, citecolor=black, urlcolor=black , pdfauthor=Grigory Rechistov <grigory.rechistov@phystech.edu>, pdftitle=Программное моделирование вычислительных систем}

\usepackage{footnpag}
\usepackage{indentfirst}
\usepackage{underscore}
\usepackage{url}

\usepackage{listings}
\lstset{basicstyle=\footnotesize\ttfamily, breaklines=true, keepspaces=true }

\usepackage{tikz}
\usetikzlibrary{shapes, calc, arrows, fit, positioning, decorations, patterns, decorations.pathreplacing, chains, snakes}
\usepackage{bytefield}

\usepackage{pgfplots} % Draw plots inside TeX
\pgfplotsset{compat=1.10}

\usepackage{standalone} % Clever way to build TikZ pictures either to PDF or to PNG


\graphicspath{{../pictures/}} % path to pictures, trailing slash is mandatory.

% The actual drawing follows
\begin{document}
\begin{tikzpicture}[>=latex, font=\scriptsize, scale=0.8]
    \foreach \x in {0,1} 
    {\foreach \y in {0,1,2,3,4,5,8,9,10,11,12,13} { % vertical offsets of two groups
            \node[draw, minimum width=1.2cm, minimum height=0.4cm, inner sep=0pt] at (\x*3.5cm, -\y * 0.6 cm) (vpage-\x-\y) {};
      };
    };
    
    \foreach \y in {0,1,2,3,4,5,6,7,8,9,10,11} {
            \node[draw, minimum width=1.2cm, minimum height=0.4cm, inner sep=0pt] at (7cm, -0.6cm -\y * 0.6 cm) (ppage-\y) {};
    };

    \node[draw, dashed, inner sep=0pt, fit=(vpage-0-0) (vpage-1-5) ]  (vm1-border) {} ;
    \node[draw, dashed,  inner sep=0pt, fit=(vpage-0-8) (vpage-1-13) ]  (vm2-border) {} ;
    \node[fill=white, inner sep=1pt] at (vm1-border.north) {Виртуальная машина 1};
    \node[fill=white, inner sep=1pt] at (vm2-border.north) {Виртуальная машина 2};
    
    \node[text width=3cm, below=0.5cm of vpage-0-13, align=flush left] {Виртуальные адреса приложений};
    \node[text width=3cm, below=0.5cm of vpage-1-13, align=flush left] {Физические адреса гостевых ОС};
    \node[text width=3cm, below=1.cm of ppage-11, align=flush left] {Настоящие физические адреса хозяина};
    
    \draw[->] (vpage-0-0.east) -- (vpage-1-3.west);
    \draw[->] (vpage-0-1.east) -- (vpage-1-2.west);
    \draw[->] (vpage-0-2.east) -- (vpage-1-1.west);
    \draw[->] (vpage-0-5.east) -- (vpage-1-0.west);
    \node[fill=white, rotate=90, text width=2.5cm, inner sep=0pt] at (1.75, -2*0.6) {Первый уровень преобразования адресов};
    
    \draw[->] (vpage-0-10.east) -- (vpage-1-8.west);
    \draw[->] (vpage-0-8.east) -- (vpage-1-10.west);
    \draw[->] (vpage-0-9.east) -- (vpage-1-10.west);
    \draw[->] (vpage-0-10.east) -- (vpage-1-12.west);
    \draw[->] (vpage-0-13.east) -- (vpage-1-13.west);
    \node[fill=white, rotate=90, text width=2.5cm, inner sep=0pt] at (1.75, -10*0.6) {Первый уровень преобразования адресов};
    
    \draw[->] (vpage-1-0.east) -- (ppage-9.west);
    \draw[->] (vpage-1-1.east) -- (ppage-5.west);
    \draw[->] (vpage-1-2.east) -- (ppage-3.west);
    \draw[->] (vpage-1-3.east) -- (ppage-1.west);
    
    \draw[->] (vpage-1-8.east) -- (ppage-0.west);
    \draw[->] (vpage-1-9.east) -- (ppage-10.west);
    \draw[->] (vpage-1-10.east) -- (ppage-11.west);
    \draw[->] (vpage-1-13.east) -- (ppage-11.west);
    \node[fill=white, rotate=90, inner sep=0pt] at (5.5, -7*0.6) {Второй уровень преобразования адресов};
\end{tikzpicture}



\end{document}
