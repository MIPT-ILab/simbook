% This file allows to produce either a separate PDF/PNG image
% See standalone documentation to understand underlying magic

\documentclass[tikz,convert={density=150,size=600,outext=.png}]{standalone}
\usetikzlibrary{shapes, calc, arrows, fit, positioning, decorations, patterns, decorations.pathreplacing, chains, snakes}
% Copyright (c) 2016 Grigory Rechistov <grigory.rechistov@phystech.edu>
% This work is licensed under the Creative Commons Attribution-NonCommercial-ShareAlike 4.0 Worldwide.
% To view a copy of this license, visit http://creativecommons.org/licenses/by-nc-sa/4.0/.

\usepackage{fontspec}
\usepackage{xunicode} % some extra unicode support
\usepackage{xltxtra}

\usepackage{amsfonts}
\usepackage{amsmath}
\usepackage{longtable}
\usepackage{csquotes}

\usepackage{polyglossia}
\setdefaultlanguage[spelling=modern]{russian} % for polyglossia
\setotherlanguage{english} % for polyglossia

% Common settings for all fonts
% 1. Attempt to make fonts be of the same size
% 2. Support TeX ligatures like — = emdash, << >> = guillemets
\defaultfontfeatures{Scale=MatchLowercase, Mapping=tex-text}

% Use Computer Modern Unicode
\newfontfamily\russianfont{CMU Serif}
\setromanfont{CMU Serif}
\setsansfont{CMU Sans Serif}
\setmonofont{CMU Typewriter Text}

% Copyright (c) 2016 Grigory Rechistov <grigory.rechistov@phystech.edu>
% This work is licensed under the Creative Commons Attribution-NonCommercial-ShareAlike 4.0 Worldwide.
% To view a copy of this license, visit http://creativecommons.org/licenses/by-nc-sa/4.0/.

% Common packages, commands and their configuration

\newcommand{\abbr}{\textit{англ.}\ }
\newcommand{\todo}[1][]{\textcolor{red}{TODO #1}}

\usepackage{graphicx}
\graphicspath{{pictures/}} % path to pictures, trailing slash is mandatory.

\usepackage{hyperref}
\hypersetup{colorlinks=true, linkcolor=black, filecolor=black, citecolor=black, urlcolor=black , pdfauthor=Grigory Rechistov <grigory.rechistov@phystech.edu>, pdftitle=Программное моделирование вычислительных систем}

\usepackage{footnpag}
\usepackage{indentfirst}
\usepackage{underscore}
\usepackage{url}

\usepackage{listings}
\lstset{basicstyle=\footnotesize\ttfamily, breaklines=true, keepspaces=true }

\usepackage{tikz}
\usetikzlibrary{shapes, calc, arrows, fit, positioning, decorations, patterns, decorations.pathreplacing, chains, snakes}
\usepackage{bytefield}

\usepackage{pgfplots} % Draw plots inside TeX
\pgfplotsset{compat=1.10}

\usepackage{standalone} % Clever way to build TikZ pictures either to PDF or to PNG


\graphicspath{{../pictures/}} % path to pictures, trailing slash is mandatory.

% The actual drawing follows
\begin{document}
    \begin{tikzpicture}[
        ctx/.style={draw, chamfered rectangle, chamfered rectangle corners={north west, south east}}
    ]
    \node[text centered] (mon-label) {}; % {Режим монитора};           
    \node[text centered, right of=mon-label, node distance = 5cm] (gst-label) {}; % {Режим гостя}; 
    % \node[dashed, draw, text width = 3cm] (root) {}; 
    % \node[dashed, draw, text width = 3cm] (non-root) {};
    
    \node[ctx, below of=mon-label, inner sep=1pt] (gst-ctx1) {\scriptsize Контекст гостя};
    \node[ctx, below of=gst-ctx1, xshift=-1cm, inner sep=1pt] (hst-ctx1) {\scriptsize Контекст хозяина};
    
    \node[ctx, below of=gst-label, inner sep=1pt] (hst-ctx2) {\scriptsize Контекст хозяина};
    \node[ctx, below of=hst-ctx2, xshift=-1cm, inner sep=1pt] (gst-ctx2) {\scriptsize Контекст гостя};

    \node[draw, text width = 2.5cm, below of=hst-ctx1, xshift=0cm, inner ysep=1cm] (mon-mode) {Режим монитора}; 
    \node[draw, double, text width = 3.5cm, below of=gst-ctx2, xshift=0cm, inner ysep=1cm] (gst-mode) {Режим виртуальной машины};
    
    
    \node[single arrow, draw, right of=mon-mode, yshift=1cm, xshift=1.3cm] (arr-to) {\scriptsize Вход};
    \node[single arrow, draw, left of=gst-mode, yshift=-1.5cm, xshift=-1.7cm, shape border rotate=180] (arr-to) {\scriptsize Выход};
    
    \node[starburst,fill=white,draw, inner sep=1pt, text width = 1.5cm, right of=arr-to, xshift=0.7cm] (bad-beh) {\scriptsize <<Плохое>> поведение};
    
    \end{tikzpicture}


\end{document}
