% This file allows to produce either a separate PDF/PNG image
% See standalone documentation to understand underlying magic

\documentclass[tikz,convert={density=150,size=600,outext=.png}]{standalone}
\usetikzlibrary{shapes, calc, arrows, fit, positioning, decorations, patterns, decorations.pathreplacing, chains, snakes}
% Copyright (c) 2016 Grigory Rechistov <grigory.rechistov@phystech.edu>
% This work is licensed under the Creative Commons Attribution-NonCommercial-ShareAlike 4.0 Worldwide.
% To view a copy of this license, visit http://creativecommons.org/licenses/by-nc-sa/4.0/.

\usepackage{fontspec}
\usepackage{xunicode} % some extra unicode support
\usepackage{xltxtra}

\usepackage{amsfonts}
\usepackage{amsmath}
\usepackage{longtable}
\usepackage{csquotes}

\usepackage{polyglossia}
\setdefaultlanguage[spelling=modern]{russian} % for polyglossia
\setotherlanguage{english} % for polyglossia

% Common settings for all fonts
% 1. Attempt to make fonts be of the same size
% 2. Support TeX ligatures like — = emdash, << >> = guillemets
\defaultfontfeatures{Scale=MatchLowercase, Mapping=tex-text}

% Use Computer Modern Unicode
\newfontfamily\russianfont{CMU Serif}
\setromanfont{CMU Serif}
\setsansfont{CMU Sans Serif}
\setmonofont{CMU Typewriter Text}

% Copyright (c) 2016 Grigory Rechistov <grigory.rechistov@phystech.edu>
% This work is licensed under the Creative Commons Attribution-NonCommercial-ShareAlike 4.0 Worldwide.
% To view a copy of this license, visit http://creativecommons.org/licenses/by-nc-sa/4.0/.

% Common packages, commands and their configuration

\newcommand{\abbr}{\textit{англ.}\ }
\newcommand{\todo}[1][]{\textcolor{red}{TODO #1}}

\usepackage{graphicx}
\graphicspath{{pictures/}} % path to pictures, trailing slash is mandatory.

\usepackage{hyperref}
\hypersetup{colorlinks=true, linkcolor=black, filecolor=black, citecolor=black, urlcolor=black , pdfauthor=Grigory Rechistov <grigory.rechistov@phystech.edu>, pdftitle=Программное моделирование вычислительных систем}

\usepackage{footnpag}
\usepackage{indentfirst}
\usepackage{underscore}
\usepackage{url}

\usepackage{listings}
\lstset{basicstyle=\footnotesize\ttfamily, breaklines=true, keepspaces=true }

\usepackage{tikz}
\usetikzlibrary{shapes, calc, arrows, fit, positioning, decorations, patterns, decorations.pathreplacing, chains, snakes}
\usepackage{bytefield}

\usepackage{pgfplots} % Draw plots inside TeX
\pgfplotsset{compat=1.10}

\usepackage{standalone} % Clever way to build TikZ pictures either to PDF or to PNG


\graphicspath{{../pictures/}} % path to pictures, trailing slash is mandatory.

% The actual drawing follows
\begin{document}
\begin{tikzpicture}
\path (-0.5,0) node[below left] (app-win) {Приложения Windows};

\path (-0.5,-1) node[below left] (win) {Windows};

\path (-0.5,-2) node[below left] (mon) {Монитор виртуальных машин};  \path (0.5,-2) node[below right] (app-lin) {Приложения Linux};
\path (0,-3) node[below] (lin) {Linux};
\path (-1,-4) node (hw) {Аппаратура IA-32};

\draw (0, -3) -|
	(node cs:name=mon, anchor=south east) -| (node cs:name=app-win,anchor=north east) -|
	(node cs:name=mon, anchor=south west);

\draw (node cs:name=win, anchor=south west) |- (node cs:name=app-win,anchor=south east);

\draw   (node cs:name=app-lin, anchor=north east) -- (node cs:name=app-lin, anchor=north west) |-
		(0, -3) ; 

\draw   (node cs:name=hw, anchor=south) -| (node cs:name=mon, anchor=north west);
\draw   (node cs:name=hw, anchor=south) -| (node cs:name=app-lin, anchor=north east);

\draw [snake=zigzag] (node cs:name=win, anchor=south east) --  
		(perpendicular cs:
		vertical line through  ={(mon.west)},
		horizontal line through={(win.south east)} 
		);
\draw [snake=snake] (node cs:name=mon, anchor=south west) --  
					(node cs:name=mon, anchor=south east) ;
\draw [snake=snake] (node cs:name=app-lin, anchor=south west) --  
					(node cs:name=app-lin, anchor=south east) ;

\draw[snake=saw]
(perpendicular cs:
		vertical line through  ={(app-lin.east)},
		horizontal line through={(hw.north)} 
		) --
(perpendicular cs:
		vertical line through  ={(mon.west)},
		horizontal line through={(hw.north)} 
		);
\end{tikzpicture}

\end{document}
