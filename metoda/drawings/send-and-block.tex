% This file allows to produce either a separate PDF/PNG image
% See standalone documentation to understand underlying magic

\documentclass[tikz,convert={density=150,size=600,outext=.png}]{standalone}
\usetikzlibrary{shapes, calc, arrows, fit, positioning, decorations, patterns, decorations.pathreplacing, chains, snakes}
% Copyright (c) 2016 Grigory Rechistov <grigory.rechistov@phystech.edu>
% This work is licensed under the Creative Commons Attribution-NonCommercial-ShareAlike 4.0 Worldwide.
% To view a copy of this license, visit http://creativecommons.org/licenses/by-nc-sa/4.0/.

\usepackage{fontspec}
\usepackage{xunicode} % some extra unicode support
\usepackage{xltxtra}

\usepackage{amsfonts}
\usepackage{amsmath}
\usepackage{longtable}
\usepackage{csquotes}

\usepackage{polyglossia}
\setdefaultlanguage[spelling=modern]{russian} % for polyglossia
\setotherlanguage{english} % for polyglossia

% Common settings for all fonts
% 1. Attempt to make fonts be of the same size
% 2. Support TeX ligatures like — = emdash, << >> = guillemets
\defaultfontfeatures{Scale=MatchLowercase, Mapping=tex-text}

% Use Computer Modern Unicode
\newfontfamily\russianfont{CMU Serif}
\setromanfont{CMU Serif}
\setsansfont{CMU Sans Serif}
\setmonofont{CMU Typewriter Text}

% Copyright (c) 2016 Grigory Rechistov <grigory.rechistov@phystech.edu>
% This work is licensed under the Creative Commons Attribution-NonCommercial-ShareAlike 4.0 Worldwide.
% To view a copy of this license, visit http://creativecommons.org/licenses/by-nc-sa/4.0/.

% Common packages, commands and their configuration

\newcommand{\abbr}{\textit{англ.}\ }
\newcommand{\todo}[1][]{\textcolor{red}{TODO #1}}

\usepackage{graphicx}
\graphicspath{{pictures/}} % path to pictures, trailing slash is mandatory.

\usepackage{hyperref}
\hypersetup{colorlinks=true, linkcolor=black, filecolor=black, citecolor=black, urlcolor=black , pdfauthor=Grigory Rechistov <grigory.rechistov@phystech.edu>, pdftitle=Программное моделирование вычислительных систем}

\usepackage{footnpag}
\usepackage{indentfirst}
\usepackage{underscore}
\usepackage{url}

\usepackage{listings}
\lstset{basicstyle=\footnotesize\ttfamily, breaklines=true, keepspaces=true }

\usepackage{tikz}
\usetikzlibrary{shapes, calc, arrows, fit, positioning, decorations, patterns, decorations.pathreplacing, chains, snakes}
\usepackage{bytefield}

\usepackage{pgfplots} % Draw plots inside TeX
\pgfplotsset{compat=1.10}

\usepackage{standalone} % Clever way to build TikZ pictures either to PDF or to PNG


\graphicspath{{../pictures/}} % path to pictures, trailing slash is mandatory.

% The actual drawing follows
\begin{document}
    \begin{tikzpicture}[>=latex, font=\scriptsize]
    
    \draw[->] (0,0) -- (10,0) node[pos=0.9, below] (sim-time1) {Время} node[pos=0., below] {Поток 1};
    \foreach \x in { 1, 2, 3, 4, 5, 6, 7, 8, 9} { 
        \draw (\x,-0.15) -- (\x,0.15) node (tick\x) {};
    };
    
    \node[draw, arrow box, arrow box arrows={north:.7cm}] at (2, -1) {$T_1$};
%     \node[draw, arrow box, arrow box arrows={south:.6cm}, text width = 0.6cm] at (2, 1) {};
    \node[shape=dart, draw, shape border rotate=270 ] at (2, 0.5)  {};
    
    \node[shape=dart, draw, shape border rotate=270 ] at (7, 0.5) (ev-unlock) {};
    \node[above=0cm of ev-unlock, text width=3.cm, inner sep=0.2cm] {
            \begin{enumerate*}
                \item Обработать событие 
                \item Разблокировать поток 2
            \end{enumerate*}
    };
    
    \draw[->] (0,-3) -- (10,-3) node[pos=0.9, below] (sim-time2) {Время} node[pos=0., below] {Поток 2};
    \foreach \x in { 1, 2, 3, 4, 5, 6, 7, 8, 9} { 
        \draw (\x,-3.15) -- (\x,-2.85) node (tick\x) {};
    };
    
    \node[draw, arrow box, arrow box arrows={north:.7cm}] at (4, -4) {$T_2$};
    
    \node[shape=dart, draw, shape border rotate=270 ] at (4, -2.5) (ev-lock) {};
    \node[text width=3.2cm, inner sep=0.2cm, above of=ev-lock] {
        \begin{enumerate*}
            \item Послать сообщение
            \item Заблокироваться
        \end{enumerate*}
    };
    
    \draw[->] (ev-lock.east) .. controls +(3,0) and +(-1,0) .. (ev-unlock.west);
    \end{tikzpicture}


\end{document}
