% This file allows to produce either a separate PDF/PNG image
% See standalone documentation to understand underlying magic

\documentclass[tikz,convert={density=150,size=600,outext=.png}]{standalone}
\usetikzlibrary{shapes, calc, arrows, fit, positioning, decorations, patterns, decorations.pathreplacing, chains, snakes}
% Copyright (c) 2016 Grigory Rechistov <grigory.rechistov@phystech.edu>
% This work is licensed under the Creative Commons Attribution-NonCommercial-ShareAlike 4.0 Worldwide.
% To view a copy of this license, visit http://creativecommons.org/licenses/by-nc-sa/4.0/.

\usepackage{fontspec}
\usepackage{xunicode} % some extra unicode support
\usepackage{xltxtra}

\usepackage{amsfonts}
\usepackage{amsmath}
\usepackage{longtable}
\usepackage{csquotes}

\usepackage{polyglossia}
\setdefaultlanguage[spelling=modern]{russian} % for polyglossia
\setotherlanguage{english} % for polyglossia

% Common settings for all fonts
% 1. Attempt to make fonts be of the same size
% 2. Support TeX ligatures like — = emdash, << >> = guillemets
\defaultfontfeatures{Scale=MatchLowercase, Mapping=tex-text}

% Use Computer Modern Unicode
\newfontfamily\russianfont{CMU Serif}
\setromanfont{CMU Serif}
\setsansfont{CMU Sans Serif}
\setmonofont{CMU Typewriter Text}

% Copyright (c) 2016 Grigory Rechistov <grigory.rechistov@phystech.edu>
% This work is licensed under the Creative Commons Attribution-NonCommercial-ShareAlike 4.0 Worldwide.
% To view a copy of this license, visit http://creativecommons.org/licenses/by-nc-sa/4.0/.

% Common packages, commands and their configuration

\newcommand{\abbr}{\textit{англ.}\ }
\newcommand{\todo}[1][]{\textcolor{red}{TODO #1}}

\usepackage{graphicx}
\graphicspath{{pictures/}} % path to pictures, trailing slash is mandatory.

\usepackage{hyperref}
\hypersetup{colorlinks=true, linkcolor=black, filecolor=black, citecolor=black, urlcolor=black , pdfauthor=Grigory Rechistov <grigory.rechistov@phystech.edu>, pdftitle=Программное моделирование вычислительных систем}

\usepackage{footnpag}
\usepackage{indentfirst}
\usepackage{underscore}
\usepackage{url}

\usepackage{listings}
\lstset{basicstyle=\footnotesize\ttfamily, breaklines=true, keepspaces=true }

\usepackage{tikz}
\usetikzlibrary{shapes, calc, arrows, fit, positioning, decorations, patterns, decorations.pathreplacing, chains, snakes}
\usepackage{bytefield}

\usepackage{pgfplots} % Draw plots inside TeX
\pgfplotsset{compat=1.10}

\usepackage{standalone} % Clever way to build TikZ pictures either to PDF or to PNG


\graphicspath{{../pictures/}} % path to pictures, trailing slash is mandatory.

% The actual drawing follows
\begin{document}
    \begin{tikzpicture}[>=latex, font=\scriptsize]
    
    \draw[->] (-0.5,0) -- (10.5,0); % node[pos=0.9, above] (sim-time) {Время};

    \begin{scope}
    \clip (0,-2) rectangle (10, 2.5);
    \foreach \x in { 1, 2, 3, 4, 5, 6, 7, 8, 9} { 
        \draw (\x,-0.15) -- (\x,0.15) node (tick\x) {};
    };
    
    \node[shape=dart, draw, shape border rotate=270 ] at (1, 0.5) (event1) {};
    \node[shape=dart, draw, shape border rotate=270 ] at (5, 0.5) (event2) {};
    \node[shape=dart, draw, shape border rotate=270 ] at (9, 0.5) (event3) {};
    
    \node[above of=event2] (desalabel) {Дискретные события};
    \draw[->] (desalabel) -- (event1);
    \draw[->] (desalabel) -- (event2);
    \draw[->] (desalabel) -- (event3);
    
    \draw (3,-0.5) ellipse[x radius = 2cm, y radius = 0.5cm] node {Исполнение процессора} ;
    \draw (7,-0.5) ellipse[x radius = 2cm, y radius = 0.5cm] node {Исполнение процессора} ;
    
    \draw (-1,-0.5) ellipse[x radius = 2cm, y radius = 0.5cm] node {} ;
    \draw (11,-0.5) ellipse[x radius = 2cm, y radius = 0.5cm] node {} ;
    \end{scope}
    \end{tikzpicture}


\end{document}
