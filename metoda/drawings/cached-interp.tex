% This file allows to produce either a separate PDF/PNG image
% See standalone documentation to understand underlying magic

\documentclass[tikz,convert={density=150,size=600,outext=.png}]{standalone}
\usetikzlibrary{shapes, calc, arrows, fit, positioning, decorations, patterns, decorations.pathreplacing, chains, snakes}
% Copyright (c) 2016 Grigory Rechistov <grigory.rechistov@phystech.edu>
% This work is licensed under the Creative Commons Attribution-NonCommercial-ShareAlike 4.0 Worldwide.
% To view a copy of this license, visit http://creativecommons.org/licenses/by-nc-sa/4.0/.

\usepackage{fontspec}
\usepackage{xunicode} % some extra unicode support
\usepackage{xltxtra}

\usepackage{amsfonts}
\usepackage{amsmath}
\usepackage{longtable}
\usepackage{csquotes}

\usepackage{polyglossia}
\setdefaultlanguage[spelling=modern]{russian} % for polyglossia
\setotherlanguage{english} % for polyglossia

% Common settings for all fonts
% 1. Attempt to make fonts be of the same size
% 2. Support TeX ligatures like — = emdash, << >> = guillemets
\defaultfontfeatures{Scale=MatchLowercase, Mapping=tex-text}

% Use Computer Modern Unicode
\newfontfamily\russianfont{CMU Serif}
\setromanfont{CMU Serif}
\setsansfont{CMU Sans Serif}
\setmonofont{CMU Typewriter Text}

% Common packages, commands and their configuration

\newcommand{\abbr}{\textit{англ.}\ }
\newcommand{\todo}[1][]{\textcolor{red}{TODO #1}}


\usepackage{graphicx}
\graphicspath{{pictures/}} % path to pictures, trailing slash is mandatory.

\usepackage{hyperref}
\hypersetup{colorlinks=true, linkcolor=black, filecolor=black, citecolor=black, urlcolor=black , pdfauthor=Grigory Rechistov <grigory.rechistov@phystech.edu>, pdftitle=Программное моделирование вычислительных систем}

\usepackage{footnpag}
\usepackage{indentfirst}
\usepackage{underscore}
\usepackage{url}

\usepackage{listings}
\lstset{basicstyle=\footnotesize\ttfamily, breaklines=true, keepspaces=true }

\usepackage{listings}
\lstset{basicstyle=\footnotesize\ttfamily, breaklines=true, keepspaces=true}

\usepackage{tikz}
\usetikzlibrary{shapes, calc, arrows, fit, positioning, decorations, patterns, decorations.pathreplacing, chains, snakes}
\usepackage{bytefield}

\graphicspath{{../pictures/}} % path to pictures, trailing slash is mandatory.

% The actual drawing follows
\begin{document}
    \begin{tikzpicture}[>=latex, font=\small,
    action/.style={ rectangle, draw, text badly centered, text width=2.cm},
    choice/.style={diamond, draw, aspect=2, text badly centered, text width=2.cm, inner sep=1pt},
    complexaction/.style={rectangle, draw, rounded corners, text badly centered},
    ]
    \node[choice]       (is-decoded)   {\scriptsize Инструкция декодирована?};
    \node[action, below=0.5cm of is-decoded] (fetch-instr)      {Извлечь инструкцию};
    \node[action, below=0.5cm  of fetch-instr] (decode)      {Декодировать};
    \node[complexaction, matrix, below=0.5cm  of decode, row sep=0.1cm, text width=4cm]   (cache-decode) {
        \node{Сохранить декодированное представление}; \\
        \node[complexaction] {Управление хранилищем кода}; \\
        };
    \node[action, below=0.5cm of cache-decode] (execute)   {Исполнить};
    
    \path[->, draw] (is-decoded.east) -| ([xshift=2cm] execute.east) node[near start, above] {Да} -- (execute.east);
    \path[->, draw] (is-decoded.south) -- (fetch-instr.north) node[midway, right] {Нет};
    \path[->, draw] (fetch-instr.south) -- (decode.north);
    \path[->, draw] (decode.south) -- (cache-decode.north);
    \path[->, draw] (cache-decode.south) -- (execute.north);
    \path[->, draw] (execute.south) |-  ([xshift=-2cm, yshift=-0.5cm] execute.west) |- ([xshift=-2.5cm,yshift=0.5cm] is-decoded.north) -| (is-decoded.north);
    
    \end{tikzpicture}


\end{document}
