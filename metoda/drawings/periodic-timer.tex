% Copyright (c) 2016 Grigory Rechistov <grigory.rechistov@phystech.edu>
% This work is licensed under the Creative Commons Attribution-NonCommercial-ShareAlike 4.0 Worldwide.
% To view a copy of this license, visit http://creativecommons.org/licenses/by-nc-sa/4.0/.

% This file allows to produce either a separate PDF/PNG image
% See standalone documentation to understand underlying magic

\documentclass[tikz,convert={density=150,size=600,outext=.png}]{standalone}
\usetikzlibrary{shapes, calc, arrows, fit, positioning, decorations, patterns, decorations.pathreplacing, chains, snakes}
% Copyright (c) 2016 Grigory Rechistov <grigory.rechistov@phystech.edu>
% This work is licensed under the Creative Commons Attribution-NonCommercial-ShareAlike 4.0 Worldwide.
% To view a copy of this license, visit http://creativecommons.org/licenses/by-nc-sa/4.0/.

\usepackage{fontspec}
\usepackage{xunicode} % some extra unicode support
\usepackage{xltxtra}

\usepackage{amsfonts}
\usepackage{amsmath}
\usepackage{longtable}
\usepackage{csquotes}

\usepackage{polyglossia}
\setdefaultlanguage[spelling=modern]{russian} % for polyglossia
\setotherlanguage{english} % for polyglossia

% Common settings for all fonts
% 1. Attempt to make fonts be of the same size
% 2. Support TeX ligatures like — = emdash, << >> = guillemets
\defaultfontfeatures{Scale=MatchLowercase, Mapping=tex-text}

% Use Computer Modern Unicode
\newfontfamily\russianfont{CMU Serif}
\setromanfont{CMU Serif}
\setsansfont{CMU Sans Serif}
\setmonofont{CMU Typewriter Text}

% Copyright (c) 2016 Grigory Rechistov <grigory.rechistov@phystech.edu>
% This work is licensed under the Creative Commons Attribution-NonCommercial-ShareAlike 4.0 Worldwide.
% To view a copy of this license, visit http://creativecommons.org/licenses/by-nc-sa/4.0/.

% Common packages, commands and their configuration

\newcommand{\abbr}{\textit{англ.}\ }
\newcommand{\todo}[1][]{\textcolor{red}{TODO #1}}

\usepackage{graphicx}
\graphicspath{{pictures/}} % path to pictures, trailing slash is mandatory.

\usepackage{hyperref}
\hypersetup{colorlinks=true, linkcolor=black, filecolor=black, citecolor=black, urlcolor=black , pdfauthor=Grigory Rechistov <grigory.rechistov@phystech.edu>, pdftitle=Программное моделирование вычислительных систем}

\usepackage{footnpag}
\usepackage{indentfirst}
\usepackage{underscore}
\usepackage{url}

\usepackage{listings}
\lstset{basicstyle=\footnotesize\ttfamily, breaklines=true, keepspaces=true }

\usepackage{tikz}
\usetikzlibrary{shapes, calc, arrows, fit, positioning, decorations, patterns, decorations.pathreplacing, chains, snakes}
\usepackage{bytefield}

\usepackage{pgfplots} % Draw plots inside TeX
\pgfplotsset{compat=1.10}

\usepackage{standalone} % Clever way to build TikZ pictures either to PDF or to PNG


\graphicspath{{../pictures/}} % path to pictures, trailing slash is mandatory.

% The actual drawing follows
\begin{document}

\begin{tikzpicture}[>=latex]
\coordinate (center) at (0,0);
\node[draw, text width = 2cm, above = 0.5 cmof center] (reference) {\texttt{reference}};
\node[draw, text width = 2cm, below = 0.5cm of center] (counter) {\texttt{counter}};
\node[draw, text width = 0.4cm, right = 2cm of center, shape = isosceles triangle, inner sep=1dd] (comparator) {=?};
\node[draw, left= 0.25cm of counter] (and) {\&};
\node[right = of comparator] (int) {\#INT};
\node[left = 2.5cm of reference] (ref-input) {REF};
\coordinate[below = 0.25cm of counter] (reset-point);
\coordinate[right=0.25cm of comparator.east] (comparator-exit);
\node[below =0.1cm of reset-point, inner sep=0pt] (reset) {\tiny{RESET}};

% draw a quartz
\coordinate (quartz) at ([xshift = -2cm]and.west);
\node[]  at (quartz) {\small{F}};
\draw (quartz) ++(-0.25,0.25) rectangle ++(0.5,-0.5);
\draw (quartz) ++(-0.25,0.35) -- ++ (0.5,0);
\draw (quartz) ++(-0.25,-0.35) -- ++ (0.5,0);
\draw (quartz) ++(0.25,0) |- (and.155) node[pos=0.75, above] (clk) {\small{CLK}};
\node[below = of clk] (enable) {\#ENABLE};
\draw (enable) |- (and.205);

% draw wires
\draw (and) -- (counter);
\draw (reference.east) -| ([xshift = -0.2cm]comparator.160) -- (comparator.160);
\draw (counter.east) -| ([xshift = -0.2cm]comparator.200) -- (comparator.200);
\draw (ref-input) -- (reference) node[pos=0.25] {\tiny{/}} node[pos=0.25, above] {16};

\draw[->] (comparator) -- (int);
\draw[->] (comparator-exit) |- (reset-point) -- (counter);

% draw a border
\node[draw, dashed, fit = (comparator-exit) (reset) (and) (reference) (counter) (comparator)] {};
\end{tikzpicture}



\end{document}
