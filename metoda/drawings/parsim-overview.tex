% This file allows to produce either a separate PDF/PNG image
% See standalone documentation to understand underlying magic

\documentclass[tikz,convert={density=150,size=600,outext=.png}]{standalone}
\usetikzlibrary{shapes, calc, arrows, fit, positioning, decorations, patterns, decorations.pathreplacing, chains, snakes}
% Copyright (c) 2016 Grigory Rechistov <grigory.rechistov@phystech.edu>
% This work is licensed under the Creative Commons Attribution-NonCommercial-ShareAlike 4.0 Worldwide.
% To view a copy of this license, visit http://creativecommons.org/licenses/by-nc-sa/4.0/.

\usepackage{fontspec}
\usepackage{xunicode} % some extra unicode support
\usepackage{xltxtra}

\usepackage{amsfonts}
\usepackage{amsmath}
\usepackage{longtable}
\usepackage{csquotes}

\usepackage{polyglossia}
\setdefaultlanguage[spelling=modern]{russian} % for polyglossia
\setotherlanguage{english} % for polyglossia

% Common settings for all fonts
% 1. Attempt to make fonts be of the same size
% 2. Support TeX ligatures like — = emdash, << >> = guillemets
\defaultfontfeatures{Scale=MatchLowercase, Mapping=tex-text}

% Use Computer Modern Unicode
\newfontfamily\russianfont{CMU Serif}
\setromanfont{CMU Serif}
\setsansfont{CMU Sans Serif}
\setmonofont{CMU Typewriter Text}

% Copyright (c) 2016 Grigory Rechistov <grigory.rechistov@phystech.edu>
% This work is licensed under the Creative Commons Attribution-NonCommercial-ShareAlike 4.0 Worldwide.
% To view a copy of this license, visit http://creativecommons.org/licenses/by-nc-sa/4.0/.

% Common packages, commands and their configuration

\newcommand{\abbr}{\textit{англ.}\ }
\newcommand{\todo}[1][]{\textcolor{red}{TODO #1}}

\usepackage{graphicx}
\graphicspath{{pictures/}} % path to pictures, trailing slash is mandatory.

\usepackage{hyperref}
\hypersetup{colorlinks=true, linkcolor=black, filecolor=black, citecolor=black, urlcolor=black , pdfauthor=Grigory Rechistov <grigory.rechistov@phystech.edu>, pdftitle=Программное моделирование вычислительных систем}

\usepackage{footnpag}
\usepackage{indentfirst}
\usepackage{underscore}
\usepackage{url}

\usepackage{listings}
\lstset{basicstyle=\footnotesize\ttfamily, breaklines=true, keepspaces=true }

\usepackage{tikz}
\usetikzlibrary{shapes, calc, arrows, fit, positioning, decorations, patterns, decorations.pathreplacing, chains, snakes}
\usepackage{bytefield}

\usepackage{pgfplots} % Draw plots inside TeX
\pgfplotsset{compat=1.10}

\usepackage{standalone} % Clever way to build TikZ pictures either to PDF or to PNG


\graphicspath{{../pictures/}} % path to pictures, trailing slash is mandatory.

% The actual drawing follows
\begin{document}
    \begin{tikzpicture}[>=latex, font=\scriptsize]
    \node[matrix, column sep={0.5cm}, row sep={0.1cm}, nodes={text width = 2.3cm, align=center, inner sep=3pt}] (separate)
    {
        \node (t1) {Хозяйский поток 1}; & \node (t2) {Хозяйский поток 2}; & \node (t3) {Хозяйский поток $N$}; \\[0.25cm]
        \node[draw] {Процессор 1};  & \node[draw] {Процессор 3}; & \node[draw] {Процессор $K-1$}; \\
        \node[draw] {Процессор 2};  & \node[draw] {Процессор 4}; & \node[draw] {Процессор $K$}; \\[0.25cm]
        \node[draw] {Устройство А};  & \node[draw] {Устройство В}; & \node[draw] {Устройство Ю}; \\
        \node[draw] {Устройство Б};  & \node[draw] {Устройство Г}; & \node[draw] {Устройство Я}; \\
        \node[draw] {Очередь 1};  & \node[draw] {Очередь 2}; & \node[draw] {Очередь $N$}; \\[0.25cm]
        \node[inner xsep=0pt] (z1) {};  & \node[inner xsep=0pt] (z2) {}; & \node[inner xsep=0pt] (z3) {}; \\
    };
    \node[draw, inner xsep=0pt, align=center, fit=(z1) (z2) (z3)] {Общая память};
    
    \draw[dashed] ([yshift=0.5cm] barycentric cs:t1=0.5,t2=0.5) -- ([yshift=-0.5cm] barycentric cs:z1=0.5,z2=0.5);
    \draw[dashed] ([yshift=0.5cm] barycentric cs:t2=0.5,t3=0.5) -- ([yshift=-0.5cm] barycentric cs:z2=0.5,z3=0.5);
    
    \end{tikzpicture}


\end{document}
