% This file allows to produce either a separate PDF/PNG image
% See standalone documentation to understand underlying magic

\documentclass[tikz,convert={density=150,size=600,outext=.png}]{standalone}
\usetikzlibrary{shapes, calc, arrows, fit, positioning, decorations, patterns, decorations.pathreplacing, chains, snakes}
% Copyright (c) 2016 Grigory Rechistov <grigory.rechistov@phystech.edu>
% This work is licensed under the Creative Commons Attribution-NonCommercial-ShareAlike 4.0 Worldwide.
% To view a copy of this license, visit http://creativecommons.org/licenses/by-nc-sa/4.0/.

\usepackage{fontspec}
\usepackage{xunicode} % some extra unicode support
\usepackage{xltxtra}

\usepackage{amsfonts}
\usepackage{amsmath}
\usepackage{longtable}
\usepackage{csquotes}

\usepackage{polyglossia}
\setdefaultlanguage[spelling=modern]{russian} % for polyglossia
\setotherlanguage{english} % for polyglossia

% Common settings for all fonts
% 1. Attempt to make fonts be of the same size
% 2. Support TeX ligatures like — = emdash, << >> = guillemets
\defaultfontfeatures{Scale=MatchLowercase, Mapping=tex-text}

% Use Computer Modern Unicode
\newfontfamily\russianfont{CMU Serif}
\setromanfont{CMU Serif}
\setsansfont{CMU Sans Serif}
\setmonofont{CMU Typewriter Text}

% Common packages, commands and their configuration

\newcommand{\abbr}{\textit{англ.}\ }
\newcommand{\todo}[1][]{\textcolor{red}{TODO #1}}


\usepackage{graphicx}
\graphicspath{{pictures/}} % path to pictures, trailing slash is mandatory.

\usepackage{hyperref}
\hypersetup{colorlinks=true, linkcolor=black, filecolor=black, citecolor=black, urlcolor=black , pdfauthor=Grigory Rechistov <grigory.rechistov@phystech.edu>, pdftitle=Программное моделирование вычислительных систем}

\usepackage{footnpag}
\usepackage{indentfirst}
\usepackage{underscore}
\usepackage{url}

\usepackage{listings}
\lstset{basicstyle=\footnotesize\ttfamily, breaklines=true, keepspaces=true }

\usepackage{listings}
\lstset{basicstyle=\footnotesize\ttfamily, breaklines=true, keepspaces=true}

\usepackage{tikz}
\usetikzlibrary{shapes, calc, arrows, fit, positioning, decorations, patterns, decorations.pathreplacing, chains, snakes}
\usepackage{bytefield}

\graphicspath{{../pictures/}} % path to pictures, trailing slash is mandatory.

% The actual drawing follows
\begin{document}
    \begin{tikzpicture}[font=\ttfamily\small, >=latex]
        \node[draw, align=left] (src) {
        mov \%r2, \%r1 \\
        \textit{mov \%cr0, \%r4} \\
        add \$0xaabb, \%r2 \\
        \textit{trap 0xbb}
        };
        
        \node[draw, align=left, right=0.5cm of src] (patched) {
        mov \%r2, \%r1 \\
        \textit{trap 0x001} \\
        add \$0xaabb, \%r2 \\
        \textit{trap 0x002}
        };
        
        \coordinate[right=2cm of patched] (mp);
        
        \node[draw, align=left, above=1.5cm of mp] (stub001) {
        handle_stub001: \\
        <обработка mov cr>
        };
        \node[draw, align=left, below=1.5cm of mp] (stub002) {
        handle_stub002: \\
        <обработка trap>
        };
        
        \node[above=0.5cm of src, font=\small] {Исходный код};
        \node[above=0.5cm of patched, font=\small, align=center ] {Код после \\ вставки заглушек};
        
        % And this is badly hard coded
        \coordinate[above=0.2cm of patched.east] (trap1);
        \coordinate[below=0.6cm of patched.east] (trap2);
        
        \draw[->] (src) -- (patched);
        \draw[->, dashed] ([xshift=-1.2cm] trap1) -| (stub001);
        \draw[->, dashed] ([xshift=-1.2cm] trap2) -| (stub002);
        
    \end{tikzpicture}


\end{document}
