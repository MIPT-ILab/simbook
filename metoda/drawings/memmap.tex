% This file allows to produce either a separate PDF/PNG image
% See standalone documentation to understand underlying magic

\documentclass[tikz,convert={density=150,size=600,outext=.png}]{standalone}
\usetikzlibrary{shapes, calc, arrows, fit, positioning, decorations, patterns, decorations.pathreplacing, chains, snakes}
% Copyright (c) 2016 Grigory Rechistov <grigory.rechistov@phystech.edu>
% This work is licensed under the Creative Commons Attribution-NonCommercial-ShareAlike 4.0 Worldwide.
% To view a copy of this license, visit http://creativecommons.org/licenses/by-nc-sa/4.0/.

\usepackage{fontspec}
\usepackage{xunicode} % some extra unicode support
\usepackage{xltxtra}

\usepackage{amsfonts}
\usepackage{amsmath}
\usepackage{longtable}
\usepackage{csquotes}

\usepackage{polyglossia}
\setdefaultlanguage[spelling=modern]{russian} % for polyglossia
\setotherlanguage{english} % for polyglossia

% Common settings for all fonts
% 1. Attempt to make fonts be of the same size
% 2. Support TeX ligatures like — = emdash, << >> = guillemets
\defaultfontfeatures{Scale=MatchLowercase, Mapping=tex-text}

% Use Computer Modern Unicode
\newfontfamily\russianfont{CMU Serif}
\setromanfont{CMU Serif}
\setsansfont{CMU Sans Serif}
\setmonofont{CMU Typewriter Text}

% Common packages, commands and their configuration

\newcommand{\abbr}{\textit{англ.}\ }
\newcommand{\todo}[1][]{\textcolor{red}{TODO #1}}


\usepackage{graphicx}
\graphicspath{{pictures/}} % path to pictures, trailing slash is mandatory.

\usepackage{hyperref}
\hypersetup{colorlinks=true, linkcolor=black, filecolor=black, citecolor=black, urlcolor=black , pdfauthor=Grigory Rechistov <grigory.rechistov@phystech.edu>, pdftitle=Программное моделирование вычислительных систем}

\usepackage{footnpag}
\usepackage{indentfirst}
\usepackage{underscore}
\usepackage{url}

\usepackage{listings}
\lstset{basicstyle=\footnotesize\ttfamily, breaklines=true, keepspaces=true }

\usepackage{listings}
\lstset{basicstyle=\footnotesize\ttfamily, breaklines=true, keepspaces=true}

\usepackage{tikz}
\usetikzlibrary{shapes, calc, arrows, fit, positioning, decorations, patterns, decorations.pathreplacing, chains, snakes}
\usepackage{bytefield}

\graphicspath{{../pictures/}} % path to pictures, trailing slash is mandatory.

% The actual drawing follows
\begin{document}
\begin{tikzpicture}[font=\small, >=latex]
    \path[font=\ttfamily, inner xsep=0pt] (0,0) -- (10,0) node[pos=0.05, above] (0x0) {0x0} node[pos=0.2, above] (0x30) {0x30} node[pos=0.35, above] {0x40} node[pos=0.5,above] (midpoint) {0xafff} node[pos=0.9, above] (0xffff) {0xffff};
    \node[draw, fit=(0x0) (0xffff)] (ruler) {};
    \node[draw, rounded corners, above=0.5cm of ruler] (cpu) {ЦПУ};
    
    \draw[<->] (cpu) -- (ruler);
    
    \draw (2,0.1) -- (2,-1.5) coordinate[very near end, inner sep=0pt] (left-c) {};
    \draw (3.5,0.1) -- (3.5,-1.5) coordinate[very near end, inner sep=0pt] (right-c) {};
    
    \draw (0.5,0.1) -- (0.5,-3) coordinate[very near end, inner sep=0pt] (left-a) {};
    \draw (5,0.1) -- (5,-3.) coordinate[very near end, inner sep=0pt] (right-a) {} node[very near end, inner sep=0pt] (left-b) {};
    \draw (9.5,0.1) -- (9.5,-3) coordinate[very near end, inner sep=0pt] (right-b) {};
    
    \draw[<->] (left-c) -- (right-c) node[midway, above] (dev-c) {Устр. В};
    \draw[<->] (left-a) -- (right-a) node[midway, above] (dev-a) {Устройство А};
    \draw[<->] (left-b) -- (right-b) node[midway, above] (dev-b) {Устройство Б};
\end{tikzpicture}


\end{document}
