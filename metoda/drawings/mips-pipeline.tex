% This file allows to produce either a separate PDF/PNG image
% See standalone documentation to understand underlying magic

\documentclass[tikz,convert={density=150,size=600,outext=.png}]{standalone}
\usetikzlibrary{shapes, calc, arrows, fit, positioning, decorations, patterns, decorations.pathreplacing, chains, snakes}
% Copyright (c) 2016 Grigory Rechistov <grigory.rechistov@phystech.edu>
% This work is licensed under the Creative Commons Attribution-NonCommercial-ShareAlike 4.0 Worldwide.
% To view a copy of this license, visit http://creativecommons.org/licenses/by-nc-sa/4.0/.

\usepackage{fontspec}
\usepackage{xunicode} % some extra unicode support
\usepackage{xltxtra}

\usepackage{amsfonts}
\usepackage{amsmath}
\usepackage{longtable}
\usepackage{csquotes}

\usepackage{polyglossia}
\setdefaultlanguage[spelling=modern]{russian} % for polyglossia
\setotherlanguage{english} % for polyglossia

% Common settings for all fonts
% 1. Attempt to make fonts be of the same size
% 2. Support TeX ligatures like — = emdash, << >> = guillemets
\defaultfontfeatures{Scale=MatchLowercase, Mapping=tex-text}

% Use Computer Modern Unicode
\newfontfamily\russianfont{CMU Serif}
\setromanfont{CMU Serif}
\setsansfont{CMU Sans Serif}
\setmonofont{CMU Typewriter Text}

% Copyright (c) 2016 Grigory Rechistov <grigory.rechistov@phystech.edu>
% This work is licensed under the Creative Commons Attribution-NonCommercial-ShareAlike 4.0 Worldwide.
% To view a copy of this license, visit http://creativecommons.org/licenses/by-nc-sa/4.0/.

% Common packages, commands and their configuration

\newcommand{\abbr}{\textit{англ.}\ }
\newcommand{\todo}[1][]{\textcolor{red}{TODO #1}}

\usepackage{graphicx}
\graphicspath{{pictures/}} % path to pictures, trailing slash is mandatory.

\usepackage{hyperref}
\hypersetup{colorlinks=true, linkcolor=black, filecolor=black, citecolor=black, urlcolor=black , pdfauthor=Grigory Rechistov <grigory.rechistov@phystech.edu>, pdftitle=Программное моделирование вычислительных систем}

\usepackage{footnpag}
\usepackage{indentfirst}
\usepackage{underscore}
\usepackage{url}

\usepackage{listings}
\lstset{basicstyle=\footnotesize\ttfamily, breaklines=true, keepspaces=true }

\usepackage{tikz}
\usetikzlibrary{shapes, calc, arrows, fit, positioning, decorations, patterns, decorations.pathreplacing, chains, snakes}
\usepackage{bytefield}

\usepackage{pgfplots} % Draw plots inside TeX
\pgfplotsset{compat=1.10}

\usepackage{standalone} % Clever way to build TikZ pictures either to PDF or to PNG


\graphicspath{{../pictures/}} % path to pictures, trailing slash is mandatory.

% The actual drawing follows
\begin{document}
\begin{tikzpicture}[>=latex, font=\small]

\draw (0,0) rectangle +(0.5,-5) node[rotate=90, midway] {IF/ID};
\draw (3,0) rectangle +(0.5,-5) node[rotate=90, midway] {ID/EX};
\draw (6,0) rectangle +(0.5,-5) node[rotate=90, midway] {EX/MEM};
\draw (9,0) rectangle +(0.5,-5) node[rotate=90, midway] {MEM/WB};

\node[draw] (pc) at (-3, -2.5) {PC} ;
\node[draw, right=0.5 of pc] (memory1) {Memory};

\node[draw, align=left] (reg-file) at (1.75,-2) {Register\\File};

\node[draw, above= of memory1, chamfered rectangle , chamfered rectangle corners={south west, south east}, rotate=90] (adder) {Adder};

\node[draw, below=of reg-file, align=left] (sign) {Sign\\Extend};

\node[draw, right=2.5 of sign, chamfered rectangle , chamfered rectangle corners={south west, south east}, rotate=90] (alu) {ALU};

\node[draw, above=2 of alu] (zero) {zero?};

\node[draw, left=0.5 of alu.north west, anchor=center, chamfered rectangle , chamfered rectangle corners={south west, south east}, rotate=90, inner sep=1pt] (mux1) {\tiny MUX};

\node[draw, left=0.5 of alu.north east, anchor=center, chamfered rectangle , chamfered rectangle corners={south west, south east}, rotate=90, inner sep=1pt] (mux2) {\tiny MUX};

\node[draw, right=2 of alu.south] (memory2) {Memory};

\node[draw, left=0.5 of alu.north east, anchor=center, chamfered rectangle , chamfered rectangle corners={south west, south east}, rotate=90, inner sep=1pt] (mux3) {\tiny MUX};

\node[draw, right=1.5 of memory2, anchor=center, chamfered rectangle , chamfered rectangle corners={south west, south east}, rotate=90, inner sep=1pt] (mux4) {\tiny MUX};


\end{tikzpicture}


\end{document}
