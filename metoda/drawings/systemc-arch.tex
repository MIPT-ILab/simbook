% This file allows to produce either a separate PDF/PNG image
% See standalone documentation to understand underlying magic

\documentclass[tikz,convert={density=150,size=600,outext=.png}]{standalone}
\usetikzlibrary{shapes, calc, arrows, fit, positioning, decorations, patterns, decorations.pathreplacing, chains, snakes}
% Copyright (c) 2016 Grigory Rechistov <grigory.rechistov@phystech.edu>
% This work is licensed under the Creative Commons Attribution-NonCommercial-ShareAlike 4.0 Worldwide.
% To view a copy of this license, visit http://creativecommons.org/licenses/by-nc-sa/4.0/.

\usepackage{fontspec}
\usepackage{xunicode} % some extra unicode support
\usepackage{xltxtra}

\usepackage{amsfonts}
\usepackage{amsmath}
\usepackage{longtable}
\usepackage{csquotes}

\usepackage{polyglossia}
\setdefaultlanguage[spelling=modern]{russian} % for polyglossia
\setotherlanguage{english} % for polyglossia

% Common settings for all fonts
% 1. Attempt to make fonts be of the same size
% 2. Support TeX ligatures like — = emdash, << >> = guillemets
\defaultfontfeatures{Scale=MatchLowercase, Mapping=tex-text}

% Use Computer Modern Unicode
\newfontfamily\russianfont{CMU Serif}
\setromanfont{CMU Serif}
\setsansfont{CMU Sans Serif}
\setmonofont{CMU Typewriter Text}

% Common packages, commands and their configuration

\newcommand{\abbr}{\textit{англ.}\ }
\newcommand{\todo}[1][]{\textcolor{red}{TODO #1}}


\usepackage{graphicx}
\graphicspath{{pictures/}} % path to pictures, trailing slash is mandatory.

\usepackage{hyperref}
\hypersetup{colorlinks=true, linkcolor=black, filecolor=black, citecolor=black, urlcolor=black , pdfauthor=Grigory Rechistov <grigory.rechistov@phystech.edu>, pdftitle=Программное моделирование вычислительных систем}

\usepackage{footnpag}
\usepackage{indentfirst}
\usepackage{underscore}
\usepackage{url}

\usepackage{listings}
\lstset{basicstyle=\footnotesize\ttfamily, breaklines=true, keepspaces=true }

\usepackage{listings}
\lstset{basicstyle=\footnotesize\ttfamily, breaklines=true, keepspaces=true}

\usepackage{tikz}
\usetikzlibrary{shapes, calc, arrows, fit, positioning, decorations, patterns, decorations.pathreplacing, chains, snakes}
\usepackage{bytefield}

\graphicspath{{../pictures/}} % path to pictures, trailing slash is mandatory.

% The actual drawing follows
\begin{document}
\begin{tikzpicture}[scale=1.13, >=latex, font=\footnotesize]
    \coordinate (a01) at (-5 ,0.5);
    \coordinate (a02) at ( 5 ,0.5);
    \coordinate (a03) at (-5  ,-1);
    \coordinate (a04) at (-2.5,-1);
    \coordinate (a05) at (   0,-1);
    \coordinate (a06) at ( 2.5,-1);
    \coordinate (a07) at (   5,-1);
    \coordinate (a08) at (   0,-2);
    \coordinate (a09) at (  -5,-3);
    \coordinate (a10) at (   0,-3);
    \coordinate (a11) at (   5,-3);
    \coordinate (a12) at (   5,-3.5);
    
    \coordinate (b01) at (-4.5,-0.5);
    \coordinate (b02) at ( 4.5,  -1);
    
    \coordinate (l01) at ( 0,   0);
    \coordinate (l02) at ( 0,-0.75);
    \coordinate (l03) at (-2.5,-1.5);
    \coordinate (l04) at ( 0,-1.5);
    \coordinate (l05) at (-5, -1);
    \coordinate (l06) at ( 2.5, -1);
    \coordinate (l07) at (   0,-3.25);
    
    \draw (a01) rectangle (a12);
    \draw (b01) rectangle (b02);
    
    \draw (a03) rectangle (a10);
    \draw (a04) rectangle (a08);
    
    \draw (a05) rectangle (a11);
    \draw (a08) rectangle (a06);
    \draw (a09) rectangle (a12);
    
    \node at (l01) {Приложение};
    \node at (l02) {Библиотеки, специфичные для выбранной технологии};
    \node[align=left, text width=3cm, anchor = west] at (l03) {Каналы: signal, clock, FIFO, mutex, semaphore};
    \node[align=left, text width=3cm, anchor = west] at (l04) {Утилиты: \\ трассировка, \\ отчёты};
    
    \node[align=left, text width=3cm, anchor = north west] at (l05) {Язык: модули, порты, экспорты,\\ процессы,\\ интерфейсы, каналы, события};
    \node[align=left, text width=3cm, anchor = north west] at (l06) {Типы данных: \\ 4-значная логика, вектора, целые и действительные\\ числа};
    
    \node at (l07) {Язык С++} ;

\end{tikzpicture}


\end{document}
