% This file allows to produce either a separate PDF/PNG image
% See standalone documentation to understand underlying magic

\documentclass[tikz,convert={density=150,size=600,outext=.png}]{standalone}
\usetikzlibrary{shapes, calc, arrows, fit, positioning, decorations, patterns, decorations.pathreplacing, chains, snakes}
% Copyright (c) 2016 Grigory Rechistov <grigory.rechistov@phystech.edu>
% This work is licensed under the Creative Commons Attribution-NonCommercial-ShareAlike 4.0 Worldwide.
% To view a copy of this license, visit http://creativecommons.org/licenses/by-nc-sa/4.0/.

\usepackage{fontspec}
\usepackage{xunicode} % some extra unicode support
\usepackage{xltxtra}

\usepackage{amsfonts}
\usepackage{amsmath}
\usepackage{longtable}
\usepackage{csquotes}

\usepackage{polyglossia}
\setdefaultlanguage[spelling=modern]{russian} % for polyglossia
\setotherlanguage{english} % for polyglossia

% Common settings for all fonts
% 1. Attempt to make fonts be of the same size
% 2. Support TeX ligatures like — = emdash, << >> = guillemets
\defaultfontfeatures{Scale=MatchLowercase, Mapping=tex-text}

% Use Computer Modern Unicode
\newfontfamily\russianfont{CMU Serif}
\setromanfont{CMU Serif}
\setsansfont{CMU Sans Serif}
\setmonofont{CMU Typewriter Text}

% Copyright (c) 2016 Grigory Rechistov <grigory.rechistov@phystech.edu>
% This work is licensed under the Creative Commons Attribution-NonCommercial-ShareAlike 4.0 Worldwide.
% To view a copy of this license, visit http://creativecommons.org/licenses/by-nc-sa/4.0/.

% Common packages, commands and their configuration

\newcommand{\abbr}{\textit{англ.}\ }
\newcommand{\todo}[1][]{\textcolor{red}{TODO #1}}

\usepackage{graphicx}
\graphicspath{{pictures/}} % path to pictures, trailing slash is mandatory.

\usepackage{hyperref}
\hypersetup{colorlinks=true, linkcolor=black, filecolor=black, citecolor=black, urlcolor=black , pdfauthor=Grigory Rechistov <grigory.rechistov@phystech.edu>, pdftitle=Программное моделирование вычислительных систем}

\usepackage{footnpag}
\usepackage{indentfirst}
\usepackage{underscore}
\usepackage{url}

\usepackage{listings}
\lstset{basicstyle=\footnotesize\ttfamily, breaklines=true, keepspaces=true }

\usepackage{tikz}
\usetikzlibrary{shapes, calc, arrows, fit, positioning, decorations, patterns, decorations.pathreplacing, chains, snakes}
\usepackage{bytefield}

\usepackage{pgfplots} % Draw plots inside TeX
\pgfplotsset{compat=1.10}

\usepackage{standalone} % Clever way to build TikZ pictures either to PDF or to PNG


\graphicspath{{../pictures/}} % path to pictures, trailing slash is mandatory.

% The actual drawing follows
\begin{document}
    \begin{tikzpicture}[>=latex, font=\small]
    
    \draw[->] (0,0) -- (10,0) node[above, pos = 0] {Запуск 1};
    \foreach \x in { 1, 2, 3, 4, 5, 6, 7, 8, 9} { 
        \draw (\x,-0.15) -- (\x,0.15) coordinate[pos=0.5] (tick\x);
    };
    
    \node[minimum width=0.3cm, shape=dart, draw, shape border rotate=270, above=0cm of tick5] (ev11) {1};
    \node[minimum width=0.3cm, shape=dart, draw, shape border rotate=270, above=0cm of ev11]  (ev12) {2};
    
    \coordinate [above=2cm of tick1] (src11);     
    \coordinate [above=2cm of tick3] (src12);     
    
    \draw[->] (src11) to[bend right = 30] (ev11);
    \draw[->] (src12) to[bend right = 30] (ev12);
    
    
    \draw[->] (0,-4) -- (10,-4) node[above, pos = 0] {Запуск 2};
    \foreach \x in { 1, 2, 3, 4, 5, 6, 7, 8, 9} { 
        \draw (\x,-4.15) -- (\x,-3.85) coordinate[pos=0.5] (tick\x);
    };
    
    \node[minimum width=0.3cm, shape=dart, draw, shape border rotate=270, above=0cm of tick5] (ev22) {2};
    \node[minimum width=0.3cm, shape=dart, draw, shape border rotate=270, above=0cm of ev22]  (ev21) {1};
    
    \coordinate [above=2cm of tick1] (src22);     
    \coordinate [above=2cm of tick3] (src21);     
    
    \draw[->] (src21) to[bend right = 30] (ev21);
    \draw[->] (src22) to[bend right = 30] (ev22);
    
    \end{tikzpicture}


\end{document}
