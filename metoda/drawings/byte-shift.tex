% This file allows to produce either a separate PDF/PNG image
% See standalone documentation to understand underlying magic

\documentclass[tikz,convert={density=150,size=600,outext=.png}]{standalone}
\usetikzlibrary{shapes, calc, arrows, fit, positioning, decorations, patterns, decorations.pathreplacing, chains, snakes}
% Copyright (c) 2016 Grigory Rechistov <grigory.rechistov@phystech.edu>
% This work is licensed under the Creative Commons Attribution-NonCommercial-ShareAlike 4.0 Worldwide.
% To view a copy of this license, visit http://creativecommons.org/licenses/by-nc-sa/4.0/.

\usepackage{fontspec}
\usepackage{xunicode} % some extra unicode support
\usepackage{xltxtra}

\usepackage{amsfonts}
\usepackage{amsmath}
\usepackage{longtable}
\usepackage{csquotes}

\usepackage{polyglossia}
\setdefaultlanguage[spelling=modern]{russian} % for polyglossia
\setotherlanguage{english} % for polyglossia

% Common settings for all fonts
% 1. Attempt to make fonts be of the same size
% 2. Support TeX ligatures like — = emdash, << >> = guillemets
\defaultfontfeatures{Scale=MatchLowercase, Mapping=tex-text}

% Use Computer Modern Unicode
\newfontfamily\russianfont{CMU Serif}
\setromanfont{CMU Serif}
\setsansfont{CMU Sans Serif}
\setmonofont{CMU Typewriter Text}

% Copyright (c) 2016 Grigory Rechistov <grigory.rechistov@phystech.edu>
% This work is licensed under the Creative Commons Attribution-NonCommercial-ShareAlike 4.0 Worldwide.
% To view a copy of this license, visit http://creativecommons.org/licenses/by-nc-sa/4.0/.

% Common packages, commands and their configuration

\newcommand{\abbr}{\textit{англ.}\ }
\newcommand{\todo}[1][]{\textcolor{red}{TODO #1}}

\usepackage{graphicx}
\graphicspath{{pictures/}} % path to pictures, trailing slash is mandatory.

\usepackage{hyperref}
\hypersetup{colorlinks=true, linkcolor=black, filecolor=black, citecolor=black, urlcolor=black , pdfauthor=Grigory Rechistov <grigory.rechistov@phystech.edu>, pdftitle=Программное моделирование вычислительных систем}

\usepackage{footnpag}
\usepackage{indentfirst}
\usepackage{underscore}
\usepackage{url}

\usepackage{listings}
\lstset{basicstyle=\footnotesize\ttfamily, breaklines=true, keepspaces=true }

\usepackage{tikz}
\usetikzlibrary{shapes, calc, arrows, fit, positioning, decorations, patterns, decorations.pathreplacing, chains, snakes}
\usepackage{bytefield}

\usepackage{pgfplots} % Draw plots inside TeX
\pgfplotsset{compat=1.10}

\usepackage{standalone} % Clever way to build TikZ pictures either to PDF or to PNG


\graphicspath{{../pictures/}} % path to pictures, trailing slash is mandatory.

% The actual drawing follows
\begin{document}
    \begin{tikzpicture}[>=latex]

    \begin{scope}[start chain, node distance = 0cm, font=\ttfamily\small, every node/.style={on chain}]
        \node[align=left] (byte0) {0xc4};
        \node[align=left] (byte1) {0xe2};
        \node[align=left] (byte2) {0x69};
        \node[align=left] (byte3) {0x37};
        \node[align=left] (byte4) {0xd9};
        \node[align=left] (byte5) {0xe4};
        \node[align=left] (byte6) {0x90};
    \end{scope}
    
    \node[below= of byte2, anchor = west, align=left, font=\ttfamily\scriptsize]  (var1) {loop   6b <.text+0x6b> \\ aaa \\ ftst \\ nop};
    
    \node[above= of byte2, anchor = west, align=left, font=\ttfamily\scriptsize]  (var2) { vpcmpgtq \%xmm1,\%xmm2,\%xmm3 \\ in     \$0x90,\%al};
    
    \draw[decorate, decoration={brace, amplitude=6pt}] ([yshift=0.05cm] byte0.north west) -- ([yshift=0.05cm] byte4.north east);
    \draw[decorate, decoration={brace, amplitude=6pt}] ([yshift=0.05cm] byte5.north west) -- ([yshift=0.05cm] byte6.north east);
    
    \draw[decorate, decoration={brace, amplitude=6pt}] ([yshift=0.05cm] byte2.south east) -- ([yshift=0.05cm] byte1.south west);
    \draw[decorate, decoration={brace, amplitude=6pt}] ([yshift=0.05cm] byte3.south east) -- ([yshift=0.05cm] byte3.south west);
    \draw[decorate, decoration={brace, amplitude=6pt}] ([yshift=0.05cm] byte5.south east) -- ([yshift=0.05cm] byte4.south west);
    \draw[decorate, decoration={brace, amplitude=6pt}] ([yshift=0.05cm] byte6.south east) -- ([yshift=0.05cm] byte6.south west);
    
    \end{tikzpicture}

\end{document}
