% This file allows to produce either a separate PDF/PNG image
% See standalone documentation to understand underlying magic

\documentclass[tikz,convert={density=150,size=600,outext=.png}]{standalone}
\usetikzlibrary{shapes, calc, arrows, fit, positioning, decorations, patterns, decorations.pathreplacing, chains, snakes}
% Copyright (c) 2016 Grigory Rechistov <grigory.rechistov@phystech.edu>
% This work is licensed under the Creative Commons Attribution-NonCommercial-ShareAlike 4.0 Worldwide.
% To view a copy of this license, visit http://creativecommons.org/licenses/by-nc-sa/4.0/.

\usepackage{fontspec}
\usepackage{xunicode} % some extra unicode support
\usepackage{xltxtra}

\usepackage{amsfonts}
\usepackage{amsmath}
\usepackage{longtable}
\usepackage{csquotes}

\usepackage{polyglossia}
\setdefaultlanguage[spelling=modern]{russian} % for polyglossia
\setotherlanguage{english} % for polyglossia

% Common settings for all fonts
% 1. Attempt to make fonts be of the same size
% 2. Support TeX ligatures like — = emdash, << >> = guillemets
\defaultfontfeatures{Scale=MatchLowercase, Mapping=tex-text}

% Use Computer Modern Unicode
\newfontfamily\russianfont{CMU Serif}
\setromanfont{CMU Serif}
\setsansfont{CMU Sans Serif}
\setmonofont{CMU Typewriter Text}

% Common packages, commands and their configuration

\newcommand{\abbr}{\textit{англ.}\ }
\newcommand{\todo}[1][]{\textcolor{red}{TODO #1}}


\usepackage{graphicx}
\graphicspath{{pictures/}} % path to pictures, trailing slash is mandatory.

\usepackage{hyperref}
\hypersetup{colorlinks=true, linkcolor=black, filecolor=black, citecolor=black, urlcolor=black , pdfauthor=Grigory Rechistov <grigory.rechistov@phystech.edu>, pdftitle=Программное моделирование вычислительных систем}

\usepackage{footnpag}
\usepackage{indentfirst}
\usepackage{underscore}
\usepackage{url}

\usepackage{listings}
\lstset{basicstyle=\footnotesize\ttfamily, breaklines=true, keepspaces=true }

\usepackage{listings}
\lstset{basicstyle=\footnotesize\ttfamily, breaklines=true, keepspaces=true}

\usepackage{tikz}
\usetikzlibrary{shapes, calc, arrows, fit, positioning, decorations, patterns, decorations.pathreplacing, chains, snakes}
\usepackage{bytefield}

\graphicspath{{../pictures/}} % path to pictures, trailing slash is mandatory.

% The actual drawing follows
\begin{document}
\begin{tikzpicture}[>=latex,font=\small]
	\clip[] (-0.8,1.7) rectangle (4,-3.5); % This is an ugly hack
	
	\node[draw] (switch) {\texttt{switch}};
	\node[minimum width = 1.2cm, draw, below=0.5cm of switch.south east, anchor=west] (ssr1) {SR1};
	\node[minimum width = 1.2cm, draw, below=0.2cm of ssr1] (ssr2) {SR2};
	\node[minimum width = 1.2cm, draw, below=0.2cm of ssr2] (ssr3) {SR3};
	\node[minimum width = 1.2cm, draw, below=0.2cm of ssr3] (ssr4) {SR4};
	
   \draw[->]  (switch.south) .. controls +(0,-1) and +(-1,0) .. (ssr1);
   \draw[->]  (switch.south) .. controls +(0,-1) and +(-1,0) .. (ssr2);
   \draw[->]  (switch.south) .. controls +(0,-1) and +(-1,0) .. (ssr3);
   \draw[->]  (switch.south) .. controls +(0,-1) and +(-1,0) .. (ssr4);
   
   \draw[->]  (ssr1.east) .. controls +(1,0) and +(0,1) .. (switch.north);
   \draw[->]  (ssr2.east) .. controls +(2,0) and +(0,2) .. (switch.north);
   \draw[->]  (ssr3.east) .. controls +(3,0) and +(0,3) .. (switch.north);
   \draw[->]  (ssr4.east) .. controls +(4,0) and +(0,4) .. (switch.north);
\end{tikzpicture}

\end{document}
