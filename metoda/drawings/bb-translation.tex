% This file allows to produce either a separate PDF/PNG image
% See standalone documentation to understand underlying magic

\documentclass[tikz,convert={density=150,size=600,outext=.png}]{standalone}
\usetikzlibrary{shapes, calc, arrows, fit, positioning, decorations, patterns, decorations.pathreplacing, chains, snakes}
% Copyright (c) 2016 Grigory Rechistov <grigory.rechistov@phystech.edu>
% This work is licensed under the Creative Commons Attribution-NonCommercial-ShareAlike 4.0 Worldwide.
% To view a copy of this license, visit http://creativecommons.org/licenses/by-nc-sa/4.0/.

\usepackage{fontspec}
\usepackage{xunicode} % some extra unicode support
\usepackage{xltxtra}

\usepackage{amsfonts}
\usepackage{amsmath}
\usepackage{longtable}
\usepackage{csquotes}

\usepackage{polyglossia}
\setdefaultlanguage[spelling=modern]{russian} % for polyglossia
\setotherlanguage{english} % for polyglossia

% Common settings for all fonts
% 1. Attempt to make fonts be of the same size
% 2. Support TeX ligatures like — = emdash, << >> = guillemets
\defaultfontfeatures{Scale=MatchLowercase, Mapping=tex-text}

% Use Computer Modern Unicode
\newfontfamily\russianfont{CMU Serif}
\setromanfont{CMU Serif}
\setsansfont{CMU Sans Serif}
\setmonofont{CMU Typewriter Text}

% Common packages, commands and their configuration

\newcommand{\abbr}{\textit{англ.}\ }
\newcommand{\todo}[1][]{\textcolor{red}{TODO #1}}


\usepackage{graphicx}
\graphicspath{{pictures/}} % path to pictures, trailing slash is mandatory.

\usepackage{hyperref}
\hypersetup{colorlinks=true, linkcolor=black, filecolor=black, citecolor=black, urlcolor=black , pdfauthor=Grigory Rechistov <grigory.rechistov@phystech.edu>, pdftitle=Программное моделирование вычислительных систем}

\usepackage{footnpag}
\usepackage{indentfirst}
\usepackage{underscore}
\usepackage{url}

\usepackage{listings}
\lstset{basicstyle=\footnotesize\ttfamily, breaklines=true, keepspaces=true }

\usepackage{listings}
\lstset{basicstyle=\footnotesize\ttfamily, breaklines=true, keepspaces=true}

\usepackage{tikz}
\usetikzlibrary{shapes, calc, arrows, fit, positioning, decorations, patterns, decorations.pathreplacing, chains, snakes}
\usepackage{bytefield}

\graphicspath{{../pictures/}} % path to pictures, trailing slash is mandatory.

% The actual drawing follows
\begin{document}
    \begin{tikzpicture}[font=\scriptsize, >=latex]

    \node (guest) {Гость}; 
    \node[right=3cm of guest] (host) {Хозяин}; 
    
    \node[draw, minimum width = 0.7cm, below =0.5cm of guest] (gi1) {инструкция};
    \node[draw, minimum width = 0.7cm, below = 0cm of gi1] (gi2) {инструкция};
    \node[draw, minimum width = 0.7cm, below = 0cm of gi2] (gi3) {инструкция};
    \node[draw, minimum width = 0.7cm, below = 0cm of gi3] (gi4) {инструкция};
    
    \node[draw, minimum width = 1.5cm, minimum height = 1cm, below =0.5cm of host] (hi1) {капсула};
    \node[draw, minimum width = 1.5cm, minimum height = 2cm, below = 0cm of hi1] (hi2) {капсула};
    \node[draw, minimum width = 1.5cm, minimum height = 0.5cm, below = 0cm of hi2] (hi3) {капсула};
    \node[draw, minimum width = 1.5cm, minimum height = 1cm, below = 0cm of hi3] (hi4) {капсула};
    \node[draw, minimum width = 1.5cm, below = 0cm of hi4] (glue) {клей};
    
    \coordinate[right=3.5cm of glue] (next-block);
    \coordinate[above=1.5cm of hi1] (entry);
    
    \draw[->] (gi1.east) to (hi1.west);
    \draw[->] (gi2.east) to (hi2.west);
    \draw[->] (gi3.east) to (hi3.west);
    \draw[->] (gi4.east) to (hi4.west);
    
    \draw[->, dashed] (glue) -- (next-block) node[midway, above, font=\tiny, align=left] {К следующему \\ блоку трансляции};
    \draw[->, dashed] (entry) -- (hi1.north);
    
    \end{tikzpicture}

\end{document}
