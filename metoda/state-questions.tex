\section{\Questions к главе \ref{state}} %\label{state-questions}

\subsection*{Вариант 1}

\begin{questions}

\question[3] Какой байт будет расположен первым в памяти на Little Endian системе при записи числа \texttt{0xaabbccdd} в память?
\begin{solution}[1cm]
0xdd
\end{solution}

\question[3] Выберите правильное отношение для фразы <<машинное слово длиной $w$ байт выровнено в памяти по адресу $addr$>>:
\begin{choices}
    \choice $addr \neq 0\ (\mod w)$ (адрес не делится нацело на длину слова),
    \choice $w = 2^k$ и $addr = 2^m$, $k,m \in \mathbb{N}$ (адрес и длина являются степенями двойки),
    \choice $w = 2^k$ и $addr = 2^m$, $k \leq m $ ,$k,m \in \mathbb{N}$ (адрес и длина являются степенями двойки, степень длины меньше степени адреса),
    \correctchoice $addr = 0\ (\mod w)$ (адрес делится нацело на длину слова).
\end{choices}

\question[3] Почему симулятор не имеет права кэшировать регионы гостевой памяти, помеченные как отображённые на устройства?
\begin{solution}[2cm]
В отличие от простой памяти, хранящей данные без изменений, устройство может менять своё состояние и соответственно содержимое регионов памяти на каждом доступе к нему.
\end{solution}

\question[3] Определение понятия <<машинное слово>>.
\begin{solution}[1cm]
Максимальный объём данных, который процессор данной архитектуры способен обработать за одну инструкцию. 
\end{solution}

\question[3] Какой интегральный тип языка Си наиболее удачно использовать для хранения состояния моделируемого регистра шириной 32 бита?
\begin{choices}
    \choice \texttt{int},
    \choice \texttt{unsigned int},
    \correctchoice \texttt{uint32_t},
    \choice зависит от хозяйской системы.
\end{choices}

\end{questions}

\subsection*{Вариант 2}

\begin{questions}

\question[3] Какой байт будет расположен первым в памяти на Big Endian системе при записи числа \texttt{0xbaadc0de} в память?
\begin{solution}[1cm]
0xba
\end{solution}


\question[3] Какую стратегию подразумевает концепция ленивого вычисления?
\begin{choices}
    \choice Замена точного значения выражения приближённым, но получаемым за меньшее время.
    \correctchoice Запуск вычисления выражения происходит лишь при необходимости использовать его результат.
    \choice Выражение вычисляется сразу после доступности значений всех его входных слагаемых.
    \choice Значение подвыражения, используемого в нескольких других выражениях, сохраняется при первом вычислении и затем переиспользуется.
\end{choices}

\question[3] Определение понятия <<байт>>.
\begin{solution}[1cm]
Минимальная адресуемая в данной архитектуре единица информации.
\end{solution}

\question[3] Сколько бит информации получает процессор при первоначальном возникновении сигнала на линии прерывания?
\begin{choices}
    \correctchoice 1 бит.
    \choice 8 бит.
    \choice Зависит от архитектуры.
\end{choices}

\question[3] Выберите правильные окончания фразы: карта памяти
\begin{choices}
    \correctchoice использует цель по умолчанию, если обрабатываемый запрос не попадает ни в одно из устройств,
    \choice может указывать на устройство не более одного раза,
    \choice должна указывать на все присутствующие в гостевой системе устройства,
    \correctchoice может указывать не только на устройства, но и на другие карты памяти.
\end{choices}


\end{questions}

% К каждой лекции должно быть от 8 до 12 задач, у каждой задачи должно быть 3-5 вариантов формулировок примерно одинаковой сложности. Допускается объединение нескольких последовательных лекций в одну тему и подготовка тестов к темам.
% Задачи должны полностью соответствовать материалам лекций, то есть лекциях должно быть достаточно информации для ответа на все вопросы.
% Формулировка каждого варианта задачи должна содержать всю необходимую информацию и не должна ссылаться на тексты внутри лекции, картинки или другие задачи или варианты задачи.
% Правильные ответы выделяются знаком «+» перед их формулировкой. Правильных ответов может быть несколько. Для тестов с несколькими ответами как минимум один ответ должен быть правильным и как минимум один ответ должен быть неправильным. 
% 
% Структура теста к лекции
% 
% \subsection*{Задача 1}
% 
% \paragraph{Вариант 1} 
% 
%     Чему равно 2+2?
%         Ответ 1. 3
%         + Ответ 2. 4
%         …
%         Ответ N. 5
% \paragraph{Вариант 2}
%     Чему равно 2*2?
%         + Ответ 1. 4
%         + Ответ 2. 2+2
%         …
%         Ответ N. 5
% \paragraph{Вариант 3}
% 
%     Чему равно 2-2?
%         Ответ 1. 0
% 
% 
%         
% \section{Просто подборка вопросов}
% 


 
 