\section{\Questions к главе \ref{chapter05}} %\label{chapter05-questions}

\subsection*{Вариант 1}

\begin{questions}

\question[3] Что из нижеперечисленного может входить в трассу, используемую для симуляции:
\begin{choices}
    \correctchoice доступы во внешнюю память,
    \correctchoice внешние прерывания,
    \choice состояние регистров,
    \correctchoice временные метки,
    \choice дизассемблер текущих инструкций?
\end{choices}

\question[3] Какие сценарии представляют наибольшую сложность для метода симуляции с помощью трасс:
\begin{choices}
\correctchoice многопоточная гостевая система,
\choice гостевое приложение с закрытым исходным кодом,
\choice изучение производительности приложений?
\end{choices}

\question[3] Как называется методика, призванная уменьшить объём данных трассы, требуемых для анализа работы приложения:
\begin{choices}
\choice манипулирование,
\choice фильтрация,
\choice интегрирование,
\correctchoice сэмплирование?
\end{choices}

\question[3] Выберите верное утверджение об использовании трасс сетевых пакетов, пришедших в модель.
\begin{choices}
\choice Они позволяют проводить интерактивную симуляцию.
\choice Они позволяют ускорить симуляцию за счёт уменьшения числа вычислений.
\correctchoice Они позволяют проводить детерминистичную симуляцию.
\end{choices}

\question[3] Выберите верное утверджение для трасс параллельных систем.
\begin{choices}
\choice Они позволяют полностью описать поведение системы для любых длительностей событий,
\choice они отражают корректное поведение, если число участвующих потоков не превышает некоторого числа,
\correctchoice они не гарантируют воспроизведение корректного порядка для событий синхронизации.
\end{choices}


\end{questions}

\subsection*{Вариант 2}

\begin{questions}

\question[3] Какой вид активности невозможен при симуляции трасс:
\begin{choices}
\correctchoice интерактивное взаимодействие с пользователем,
\choice загрузка операционной системы,
\choice работа с периферийными устройствами?
\end{choices}

\question[3] Какие типы событий должны быть отражены в трассе работы приложения для того, чтобы она была полезна:
\begin{choices}
\correctchoice только внешние события: доступы в память, к устройствам,
\choice только внутренние события: изменения регистров,
\choice и внутренние, и внешние события?
\end{choices}

\question[3] Выберите правильный порядок операций при обработке трассы:
\begin{choices}
\choice перематывание -- измерение -- разогрев,
\choice разогрев -- перематывание  -- измерение,
\correctchoice перематывание -- разогрев -- измерение.
\end{choices}

\question[3] Выберите верное утверджение о размерах файлов трасс.
\begin{choices}
\correctchoice Двоичные форматы позволяют получить трассы меньшего размера.
\choice Текстовые форматы позволяют получить трассы меньшего размера.
\choice Размеры текстовых и двоичных трасс приблизительно одинаковы в большинстве случаев.
\end{choices}

\question[3] Выберите верное продолжение фразы: при валидации симулятора с референсной трассой{}\dots
\begin{choices}
\correctchoice симуляция прерывается после первого расхождения состояний,
\choice симуляция продолжается до конца референсной трассы,
\choice симуляция прерывается, если превышено некоторое критическое значение для числа различий между состояниями.
\end{choices}


\end{questions}


% К каждой лекции должно быть от 8 до 12 задач, у каждой задачи должно быть 3-5 вариантов формулировок примерно одинаковой сложности. Допускается объединение нескольких последовательных лекций в одну тему и подготовка тестов к темам.
% Задачи должны полностью соответствовать материалам лекций, то есть лекциях должно быть достаточно информации для ответа на все вопросы.
% Формулировка каждого варианта задачи должна содержать всю необходимую информацию и не должна ссылаться на тексты внутри лекции, картинки или другие задачи или варианты задачи.
% Правильные ответы выделяются знаком «+» перед их формулировкой. Правильных ответов может быть несколько. Для тестов с несколькими ответами как минимум один ответ должен быть правильным и как минимум один ответ должен быть неправильным. 
% 
% Структура теста к лекции
% 
% \subsection*{Задача 1}
% 
% \paragraph{Вариант 1} 
% 
%     Чему равно 2+2?
%         Ответ 1. 3
%         + Ответ 2. 4
%         …
%         Ответ N. 5
% \paragraph{Вариант 2}
%     Чему равно 2*2?
%         + Ответ 1. 4
%         + Ответ 2. 2+2
%         …
%         Ответ N. 5
% \paragraph{Вариант 3}
% 
%     Чему равно 2-2?
%         Ответ 1. 0
% 
% 
%         
% \section{Просто подборка вопросов}
% 


 
 