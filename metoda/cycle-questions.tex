\section{\Questions к главе \ref{cycle}} %\label{cycle-questions}

\subsection*{Вариант 1}

\begin{questions}

\question[3] Выберите правильные варианты ответов:
\begin{choices}
    \correctchoice функциональные модули не имеют внутреннего состояния,
    \choice функциональные модули могут иметь внутреннее состояние,
    \choice функциональные модули всегда имеют внутреннее состояние.
\end{choices}
    
    
\question[3] Выберите правильные варианты ответов:
\begin{choices}
    \correctchoice ширина входа и выхода порта должны быть равны,
    \choice ширина входа и выхода порта могут различаться,
    \choice количество выходов функционального элемента должно быть равно единице,
    \choice количество входов и выходов функционального элемента должно совпадать.
\end{choices}
    
\question[3] Выберите правильные варианты продолжения фразы: процесс исполнения потактовой модели на основе портов 
\begin{choices}
    \correctchoice всегда содержит две фазы, которые обязаны чередоваться,
    \choice всегда содержит две фазы, порядок которых не фиксированный,
    \choice всегда содержит одну фазу, в течение которой работают все субъединицы,
    \choice может содержать более двух чередующихся фаз.
\end{choices}



\end{questions}

\subsection*{Вариант 2}

\begin{questions}

\question[3] Выберите правильные варианты ответов:
\begin{choices}
    \choice при передаче данных порты не сохраняют бит валидности данных,
    \choice при передаче данных порты не сохраняют бит валидности данных, только если он снят,
    \choice при передаче данных порты не сохраняют бит валидности данных, только если он поднят,
    \correctchoice при передаче данных порты сохраняют бит валидности данных.
\end{choices}

\question[3] Выберите правильные варианты продолжения фразы: внутри исполнения фазы функциональных элементов потактовой модели на основе портов 
\begin{choices}
    \choice первыми должны исполняться функции, расположенные в графе правее,
    \correctchoice порядок исполнения функций устройств неважен,
    \choice первыми должны исполняться функции, расположенные в графе левее.
\end{choices}

\question[3] Выберите правильные варианты продолжения фразы: в модели, описанной на основе портов,
\begin{choices}
    \choice функциональные модули имеют различные задержки выполнения,
    \correctchoice функциональные модули не имеют определённой задержки выполнения,
    \choice функция портов является функцией тождественности, а задержка нулевая,
    \correctchoice функция портов является функцией тождественности, а задержка ненулевая,
    \choice функция портов не является функцией тождественности, а задержка нулевая.
\end{choices}


\end{questions}

% К каждой лекции должно быть от 8 до 12 задач, у каждой задачи должно быть 3-5 вариантов формулировок примерно одинаковой сложности. Допускается объединение нескольких последовательных лекций в одну тему и подготовка тестов к темам.
% Задачи должны полностью соответствовать материалам лекций, то есть лекциях должно быть достаточно информации для ответа на все вопросы.
% Формулировка каждого варианта задачи должна содержать всю необходимую информацию и не должна ссылаться на тексты внутри лекции, картинки или другие задачи или варианты задачи.
% Правильные ответы выделяются знаком «+» перед их формулировкой. Правильных ответов может быть несколько. Для тестов с несколькими ответами как минимум один ответ должен быть правильным и как минимум один ответ должен быть неправильным. 
% 
% Структура теста к лекции
% 
% \subsection*{Задача 1}
% 
% \paragraph{Вариант 1} 
% 
%     Чему равно 2+2?
%         Ответ 1. 3
%         + Ответ 2. 4
%         …
%         Ответ N. 5
% \paragraph{Вариант 2}
%     Чему равно 2*2?
%         + Ответ 1. 4
%         + Ответ 2. 2+2
%         …
%         Ответ N. 5
% \paragraph{Вариант 3}
% 
%     Чему равно 2-2?
%         Ответ 1. 0
% 
% 
%         
% \section{Просто подборка вопросов}
% 


 
 