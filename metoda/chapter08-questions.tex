\section{\Questions к главе \ref{chapter08}} %\label{chapter08-questions}

\todo Расширить и переработать

\subsection*{Вариант 1}

\begin{questions}

\question[3] Какие из типов схем PDES позволяют добиться детерминизма симуляции?
\begin{choices}
    \correctchoice Барьерная (с доменами синхронизации).
    \choice Консервативная.
    \choice Оптимистичная.
    \choice Наивная.
\end{choices}

\question[3] Чем чревата излишне частая отправка пустых (null) сообщений в консервативной схеме PDES с детектированием взаимоблокировок?
\begin{solution}[1cm]
Низкой скоростью работы симулятора из-за перегруженности хозяйской сети.
\end{solution}

\question[3] Выберите правильные продолжения фразы: Частая отправка пустых (null)  сообщений нежелательна, так как
\begin{choices}
    \correctchoice это может ограничивать скорость симуляции,
    \choice это может вызвать нарушение каузальности симуляции,
    \choice это может привести к взаимоблокировке потоков,
    \choice это может привести к переполнению очередей сообщений.
\end{choices}

\question[3] Выберите правильные ответы.
\begin{choices}
    \choice    Симуляция, реализованная с помощью схемы PDES, всегда детерминистична.
    \choice Симуляция, реализованная с помощью схемы PDES, недетерминистична из-за возможности потери сообщений между потоками.
    \choice Симуляция, реализованная с помощью схемы PDES, недетерминистична из-за возможности блокировки отдельных потоков.
    \choice Симуляция, реализованная с помощью схемы PDES, недетерминистична из-за варьирующейся скорости работы отдельных потоков.
\end{choices}


\end{questions}

\subsection*{Вариант 2}

\begin{questions}

\question[3] Почему не будет работать \textbf{наивная} схема параллельного DES? Выберите верные ответы.
\begin{choices}
    \correctchoice Недетерминизм модели.
    \choice Невозможно организовать передачу сообщений между потоками.
    \correctchoice Возможно нарушение каузальности.
    \choice Невозможно подобрать точно квоту выполнения.
\end{choices}

\question[3] Чем чревата недостаточно частая отправка пустых (null) сообщений в консервативной схеме PDES с детектированием взаимоблокировок?
\begin{solution}[1cm]
Низкой скоростью работы симулятора из-за частой блокировки потоков, слишком далеко убежавших в симулируемое будущее.
\end{solution}

\question[3] Выберите правильные ответы.
\begin{choices}
    \correctchoice Консервативные схемы PDES не допускают нарушения каузальности.
    \choice Консервативные схемы PDES допускают нарушения каузальности.
    \choice Консервативные схемы PDES допускают нарушения каузальности, но впоследствии их корректируют.
    \choice Оптимистичные схемы PDES не допускают нарушения каузальности.
    \choice Оптимистичные схемы PDES допускают нарушения каузальности.
    \correctchoice Оптимистичные схемы PDES допускают нарушения каузальности, но впоследствии их корректируют.
\end{choices}

\question[3] Выберите правильные свойства домена синхронизации в модели PDES.
\begin{choices}
    \choice Количество моделируемых устройств внутри одного домена фиксировано.
    \choice Не происходит взаимодействия устройств, находящихся в различных доменах.
    \choice Количество моделируемых устройств внутри одного домена ограничено.
    \correctchoice Характерная частота коммуникаций  между доменами превышает частоту коммуникаций внутри каждого.
    \choice Характерная частота коммуникаций  между доменами равна частоте коммуникаций внутри отдельного домена.
\end{choices}


\end{questions}

% К каждой лекции должно быть от 8 до 12 задач, у каждой задачи должно быть 3-5 вариантов формулировок примерно одинаковой сложности. Допускается объединение нескольких последовательных лекций в одну тему и подготовка тестов к темам.
% Задачи должны полностью соответствовать материалам лекций, то есть лекциях должно быть достаточно информации для ответа на все вопросы.
% Формулировка каждого варианта задачи должна содержать всю необходимую информацию и не должна ссылаться на тексты внутри лекции, картинки или другие задачи или варианты задачи.
% Правильные ответы выделяются знаком «+» перед их формулировкой. Правильных ответов может быть несколько. Для тестов с несколькими ответами как минимум один ответ должен быть правильным и как минимум один ответ должен быть неправильным. 
% 
% Структура теста к лекции
% 
% \subsection*{Задача 1}
% 
% \paragraph{Вариант 1} 
% 
%     Чему равно 2+2?
%         Ответ 1. 3
%         + Ответ 2. 4
%         …
%         Ответ N. 5
% \paragraph{Вариант 2}
%     Чему равно 2*2?
%         + Ответ 1. 4
%         + Ответ 2. 2+2
%         …
%         Ответ N. 5
% \paragraph{Вариант 3}
% 
%     Чему равно 2-2?
%         Ответ 1. 0
% 
% 
%         
% \section{Просто подборка вопросов}
% 


 
 