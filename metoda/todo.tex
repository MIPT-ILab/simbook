\section*{Список TODO}

Даннаяя секция предназначена для напоминания авторам, какие темы необходимо раскрыть в последующих редакциях книги. Всем остальным просьба не обращать внимания.

\begin{itemize}
    \item Доступ к гостевым образам дисков --- libguestfs \url{http://libguestfs.org/}.
    \item Обратное исполнение --- выделить в главу, описать сценарии откатов, стратегии хранения точек отката, требования на память и т.д.
    \item Потактовая симуляция --- рассказать про 0-cycle связи внутри узла.
    \item Описать split transactions.
    \item Заключение --- написать перспективы развития.
	\item Добавить главу про гибриды.
	\item Блоки кода сделать неразрывными.
	\item Расширить секцию про SystemC.
	\item Отформатировать блоки кода: \url{http://mydebianblog.blogspot.ru/2012/12/latex.html}.
    \item Обрисовать системы, которые симулируются: процессор (почти есть), платформа (нет), конвеер ЦПУ (нет), сеть узлов (нет).
    \item Виртуализация: описать vmxassist из Xen (V86 для виртуализации реального режима) \url{http://compbio.cs.toronto.edu/repos/snowflock/xen-3.0.3/tools/firmware/README}
\end{itemize}

% Students remarks for year 2nd-class 2014.

% 4/14/2014 22:08:44	8	"Хотелось бы больше информации в презентациях, но, в принципе, ее можно найти в книге.
 
% Очень понравилось."	9	Паравиртуализация, прямое исполнение		9		8	Доволен	Хотелось бы больше разных вопросов.
% 4/14/2014 22:48:25	9	Было мало практики, в связи с этим иногда пропадал интерес	8	Двоичная трансляция, моделирование с использованием трасс, языки разработки моделей и аппаратуры, кеши	Не совсем понятны темы с распараллеливанием симуляции	9		10		
% 4/15/2014 14:11:41	8	"В курс необходимо добавить побольше практической работы. Без этого, информация воспринимается не очень хороше.

% Книжка написана очень абстрактно, иногда не сразу понятно о чем именно ведется речь. Но, это не замечание, потому, что походу это обосновано спецификой материала, у меня лучше бы не получилось.
% (наверное нужно побольше маленьких примеров, привязаных к железу реальному)"	7	бинарная трансляция, декодирование	без обращения к старонней литературе- я не понял толком что такое гирепвизор. Не было четко рассмотрено понятие аппаратной виртуализации (и ее основные моменты) , которая, я считаю, важна для понимания.	9		8	Доволен	"Работа была не обьективной по 2м причинам:

% -некоторые тестовые вопроссы были сомнительными, с неоднозначным ответом. (а за верный ответ наверное- считался ответ в книжке)
% (знаешь- что можно написать и так и так, но выбрать нужно чтото определенное. Можно не угадать)
% -Некоторые ребята на задних парах ""не брезгали книжкой"""
% 4/15/2014 14:23:24	9	Чтобы материал усваивался лучше, можно проводить небольшие тесты на 10-15 минут в начале каждого занятия по материалам предыдущего, так студенты будут лучше ориентироваться в материале.	8	DES, PDES, прямая трансляция, бинарная трансляция, cache, вводная лекция	взаимодействие симуляции с внешним миром, языки разработки моделей	9		10	Недоволен	Контрольная не отражает знаний, так как не была прослежена самостоятельность выполнения заданий. В некоторых заданиях было не совсем понятно что требуется. Хотелось бы, чтобы результаты не сильно влияли на набор на кафедру.
% 4/15/2014 20:28:59	9	"Мне понравилось обсуждение различных алгоритмов. Думаю, стоит добавить больше обсуждений на занятиях и, возможно, анализа эффективности каких-нибудь абстрактных машин и симуляторов, работающих на них. На мой взгляд, лабораторные работы с симуляторами мало чему учили и не были действительно интересными.
% Спасибо за интересный курс!"	8	PDES, DES, интерпретация, двоичная трансляция	Языки разработки моделей и аппаратуры	10		10	Доволен	
% 4/15/2014 22:28:06	9		8			10		1	Могло быть и лучше	
% 4/16/2014 14:04:07	8		9			9		7	Могло быть и лучше	


% Students remarks for year 4-th class 2014

% 5/19/2014 20:02:35	6	"Симуляция отдельных программ - понятная проблема, и подходы к ней можно излагать абстрактно, без примеров.

% Но в симуляции устройств нет никакой интуиции: каким образом тема подшивается к курсу, какие именно события происходят на устройствах, какое место эта симуляция занимает в DOSbox, VirtualBox, происходит ли эта симуляция одновременно с симуляцией программ. Тема DES рассказывается на абстрактном уровне, без конкретного примера. Есть ощущение, что я не понял, зачем всё это нужно.

% Симуляция на основе трасс - также: я вообще не понял проблему, которую данная тема решает. Нужен подробный разбор конкретного примера."	10			10		10	Доволен	
% 5/19/2014 20:06:07	10	"Крайне полезным было разъяснение понятий симуляции, эмуляции, виртуализации. Харизма лектора очень поспособствовала усвоению материала.

% Что улучшить - развить практику вопросов в начале лекции.

% Спасибо большое за курс!"	10	"DES, PDES - совершенно новые для меня темы.
% Параллельная симуляция также позволила посмотреть на атомарные инструкции и модели памяти с немного иного ракурса, что тоже очень и очень полезно."	Языки описания моделей - наименее интересная лекция.	8		10	Доволен	Побольше интервал между вопросами?:)
% 5/19/2014 21:05:52	8		8					10	Доволен	
% 5/20/2014 2:46:25	8		7	Потактовая симуляция, PDES	паравиртуализация	10		10	Доволен	
% 5/20/2014 11:32:53	10		10	Параллельная симуляция	Виртуализация	10		10	Могло быть и лучше	