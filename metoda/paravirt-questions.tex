\section{\Questions к главе \ref{paravirt}} %\label{paravirt-questions}

\subsection*{Вариант 1}

\begin{questions}

\question[3] Выберите свойства, которые должны выполняться для идеальной <<волшебной>> инструкции:
\begin{choices}
    \correctchoice должна быть допустимой во всех режимах работы процессора,
    \choice должна быть привилегированной,
    \choice не должна иметь явных аргументов,
    \correctchoice не должна генерироваться обычными компиляторами,
    \correctchoice не должна вызывать эффектов (т.е. быть NOP),
    \choice не должна иметь неявных аргументов.
\end{choices}

\question[3] Какая инструкция для архитектуры IA-32 не может быть использована как волшебная?
\begin{choices}
\choice \texttt{CPUID} --- идентификация процессора,
\correctchoice \texttt{INT} --- программное прерывание,
\choice \texttt{NOP} --- пустая операция.
\end{choices}

\question[3] Для какой из перечисленных ниже операционных систем паравиртуализационные расширения сложно писать из-за закрытости исходного кода?
\begin{choices}
\correctchoice Microsoft Windows,
\choice GNU/Linux,
\choice FreeBSD.
\end{choices}

\question[3] В чём состоят недостатки сырого формата дисков?
\begin{choices}
\choice невозможность случайного доступа к секторам диска,
\choice нерациональное расходование дискового пространства гостя,
\correctchoice нерациональное расходование дискового пространства хозяина,
\choice отсутствие публичной документации на формат.
\end{choices}

\end{questions}

\subsection*{Вариант 2}

\begin{questions}

\question[3] Почему передача большого объёма данных между гостем и хозяином с помощью волшебной инструкции неэффективна?
\begin{choices}
\correctchoice за один раз можно передать только несколько байт,
\choice побочные эффекты множества волшебных инструкций подряд могут нарушить работу гостя,
\choice побочные эффекты множества волшебных инструкций подряд могут нарушить работу хозяина,
\choice направление передачи данных ограничено только направлением «гость → хозяин».
\end{choices}

\question[3] Назовите приём виртуализации, в котором гостевое приложение модифицируются таким образом, чтобы задействовать некоторую функциональность аппаратуры, присутствующую только внутри модели, но не на реальных системах?
\begin{choices}
\choice гиперсимуляция,
\choice метавиртуализация,
\correctchoice  паравиртуализация,
\choice изоляция.
\end{choices}

\question[3] Дайте определение термину <<проброс устройства>>:
\begin{choices}
\choice передача устройства в эксклюзивное пользование нескольким гостям,
\choice передача устройства в эксклюзивное пользование хозяину,
\correctchoice передача устройства в эксклюзивное пользование единственному гостю.
\end{choices}

\question[3] Для чего используются разностные файлы?
\begin{choices}
\correctchoice хранение изменений гостевого диска за время работы симуляции,
\choice сжатие оригинального образа гостевого диска для того, чтобы он занимал меньше места,
\choice прозрачное шифрование оригинального образа гостевого диска,
\choice расширения размера гостевого диска в случае, когда старый полностью заполнен.
\end{choices}

\end{questions}

% К каждой лекции должно быть от 8 до 12 задач, у каждой задачи должно быть 3-5 вариантов формулировок примерно одинаковой сложности. Допускается объединение нескольких последовательных лекций в одну тему и подготовка тестов к темам.
% Задачи должны полностью соответствовать материалам лекций, то есть лекциях должно быть достаточно информации для ответа на все вопросы.
% Формулировка каждого варианта задачи должна содержать всю необходимую информацию и не должна ссылаться на тексты внутри лекции, картинки или другие задачи или варианты задачи.
% Правильные ответы выделяются знаком «+» перед их формулировкой. Правильных ответов может быть несколько. Для тестов с несколькими ответами как минимум один ответ должен быть правильным и как минимум один ответ должен быть неправильным. 
% 
% Структура теста к лекции
% 
% \subsection*{Задача 1}
% 
% \paragraph{Вариант 1} 
% 
%     Чему равно 2+2?
%         Ответ 1. 3
%         + Ответ 2. 4
%         …
%         Ответ N. 5
% \paragraph{Вариант 2}
%     Чему равно 2*2?
%         + Ответ 1. 4
%         + Ответ 2. 2+2
%         …
%         Ответ N. 5
% \paragraph{Вариант 3}
% 
%     Чему равно 2-2?
%         Ответ 1. 0
% 
% 
%         
% \section{Просто подборка вопросов}
%