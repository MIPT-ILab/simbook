% This page is a mess
\thispagestyle{empty}
\begingroup
\small
\begin{flushleft}
УДК 004.42(075)

ББК 32.973я73\\
\enskip \enskip \enskip \enskip \enskip О75    
\end{flushleft}
\begin{center}
\begin{normalsize}
\textsf{{А\,в\,т\,о\,р\,ы}:}
\end{normalsize}
Г.~С.~Речистов, Е.~А.~Юлюгин, А.~А.~Иванов, \\
П.~Л.~Шишпор, Н.~Н.~Щелкунов\\
\end {center}

\begin{center}
Рецензенты:

Кафедра информационных систем управления и информационных технологий Московского государственного университета приборостроения и информатики\\
Доктор технических наук, профессор \textit{Т.~Ю.~Морозова}

Кадидат технических наук \textit{Н.~Б.~Преображенский }

\end{center}

\noindent O75 ~~~ {Основы программного моделирования ЭВМ}: учеб. пособие/ \\Г.~С.~Речистов, Е.~А.~Юлюгин и др. --- 2-е изд., испр. и доп. --- М.~: МФТИ 2013 --- \pageref{page:lastpage}~с.\\
\textbf{ISBN 978-5-7417-0444-8}
\medskip

Рассмотрены вопросы моделирования аппаратных средств однопроцессорных и многопроцессорных вычислительных систем. Основное внимание уделено технологиям построения программных симуляторов, в том числе на основе интерпретации, двоичной трансляции, аппаратной виртуализации, многопоточного исполнения и моделирования с использованием трасс.

Предназначено для студентов вузов, аспирантов и специалистов по средствам проектирования цифровых систем, разработке и исследованию проблемно-ориентированных архитектур ЭВМ и смежным направлениям. 

{\raggedleft УДК 004.358 О75 \par}

{\raggedleft ББК 32.973я73 \par}
\vfill

\begin{center}
% The table in the bottom of the page.
\begin{tabular}{lp{0.56\textwidth}}
\textbf{ISBN 978-5-7417-0444-8} & \textcopyright~~Речистов~Г.~С., Юлюгин~Е.~А.,\\
& Иванов А. А., Шишпор П. Л., \\
& Щелкунов Н. Н., 2013 \\
                            & \textcopyright~~Федеральное государственное автономное образовательное учреждение высшего профессионального образования «Московский физико-технический институт (государственный университет)», 2013
\end{tabular}

\end{center}

\endgroup
