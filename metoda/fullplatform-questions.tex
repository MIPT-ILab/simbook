\section{\Questions к главе \ref{fullplatform}} %\label{fullplatform-questions}

\subsection*{Вариант 1}

\begin{questions}

\question[3] Определение понятия <<квота>>, используемого в симуляции многопроцессорных систем.
\begin{choices}
    \choice Минимальное число шагов, симулируемых моделью процессора без переключения на симуляцию других.
    \choice Значение числа исполненных шагов, возвращаемое моделью процессора при передаче управления.
    \choice Максимальное число шагов, симулируемых всеми моделями процессоров модели без переключения на симуляцию других устройств.
    \correctchoice Максимальное число шагов, симулируемых моделью процессора без переключения на симуляцию других.
\end{choices}


\question[3] Выберите правильные свойства для неисполняющей модели.
\begin{choices}
    \correctchoice Изменения состояния происходит асинхронно по отношению к остальным устройствам.
    \correctchoice Характер взаимодействия с другими агентами представлятся в виде «запрос--отклик».
    \choice Модель меняет своё состояние каждый или почти каждый такт.
    \correctchoice Модель может быть характеризована как «чёрный ящик».
    \choice События от сторонних устройств к модели приходят редко.
%     \choice Интерес для исследователя представляют внутренние процессы моделируемого устройства
\end{choices}

\question[3] Выберите правильный вариант продолжения фразы: Симулируемое время в моделях DES 
\begin{choices}
    \choice изменяется непрерывно,
    \choice изменяется скачками фиксированной длительности,
    \correctchoice изменяется скачками, длительность которых различна.
\end{choices}
    
\question[3] Выберите правильный вариант окончания фразы: в моделях DES события могут обрабатываться, если они находятся
\begin{choices}
    \choice только в голове очереди событий (самые поздние),
    \correctchoice только в хвосте очереди событий (самые ранние),
    \choice в любой позиции в очереди событий.
\end{choices}

\question[3] Выберите правильный вариант окончания фразы: в моделях DES новые события могут быть добавлены
\begin{choices}
    \choice только к голове очереди событий,
    \choice только к хвосту очереди событий,
    \correctchoice в любую позицию в очереди событий.
\end{choices}
    
\question[3] Выберите сценарии, когда скорость симуляции, превышающая скорость работы реальной системы, нежелательна:
\begin{choices}
    \choice программа вычисляет значение некоторой функции в узлах сетки и выводит результаты на экран,
    \correctchoice система ожидает ввода пользователя в течение ограниченного времени,
    \choice программа взаимодействует по моделируемой сети с другой моделируемой системой.
\end{choices}


\end{questions}

\subsection*{Вариант 2}

\begin{questions}

\question[3] Как могут проявиться недостатки излишне большой квоты при симулировании многопроцессорных систем?
% \begin{solution}[2cm]
%     Симуляция будет некорректной --- будут происходить тайм-ауты взаимодействия моделируемых процессоров.
% \end{solution}
\begin{choices}
    \choice Скорость моделирования будет низкой.
    \choice Скорость моделирования будет высокой.
    \choice Невозможно будет остановить моделирование.
    \choice Число моделируемых агентов будет ограничено.
    \correctchoice Симуляция будет некорректной. % --- будут происходить тайм-ауты взаимодействия моделируемых процессоров.
\end{choices}

\question[3] Выберите правильные свойства для исполняющей модели.
\begin{choices}
    \choice Изменения состояния происходит асинхронно по отношению к остальным устройствам.
    \correctchoice Модель меняет своё состояние почти каждый такт.
    \choice Характер взаимодействия с другими агентами представлятся в виде «запрос--отклик».
%    \choice События от сторонних устройств к модели приходят редко.
    \choice Модель может быть характеризована как «чёрный ящик».
    \correctchoice Интерес для исследователя представляют внутренние процессы моделируемого устройства.
\end{choices}

\question[3] Выберите правильные возможности из перечисленных.
\begin{choices}
    \choice Скорость течения симулируемого времени может быть меньше скорости течения реального времени.
    \choice Скорость течения симулируемого времени может быть больше скорости течения реального времени.
    \choice Скорость течения симулируемого времени приблизительно равна скорости течения реального времени.
    \correctchoice Все вышеперечисленные варианты верны.
\end{choices}   

\question[3] Выберите правильный вариант окончания фразы: в моделях DES одно значение метки времени
\begin{choices}
    \choice может соответствовать максимум одному событию,
    \choice может соответствовать нескольким событиям, порядок их обработки при этом неопределён,
    \correctchoice может соответствовать нескольким событиям, порядок их обработки при этом определён,
    \choice всегда соответствует нескольким событиям, некоторые из них могут быть псевдособытиями.
\end{choices}

\question[3] Выберите правильный вариант окончания фразы: в моделях DES события из очереди могут быть удалены
\begin{choices}
    \choice только из головы очереди событий,
    \choice только из хвоста очереди событий,
    \correctchoice из любой позиции в очереди событий.
\end{choices}

\question[3] Выберите правильное выражение для отношения скоростей моделирования систем с $N$ гостевыми процессорами и с одним хозяйским процессором при однопоточной симуляции:
\begin{choices}
    \correctchoice $\frac{S(N)}{S(1)} = O(1/N)$,
    \choice $\frac{S(N)}{S(1)} = O(N)$,
    \choice $\frac{S(N)}{S(1)} = O(1/N^2)$,
    \choice $\frac{S(N)}{S(1)} = O(N^2)$,
    \choice $\frac{S(N)}{S(1)} = O(\ln{}N)$.
\end{choices}


\end{questions}

% К каждой лекции должно быть от 8 до 12 задач, у каждой задачи должно быть 3-5 вариантов формулировок примерно одинаковой сложности. Допускается объединение нескольких последовательных лекций в одну тему и подготовка тестов к темам.
% Задачи должны полностью соответствовать материалам лекций, то есть лекциях должно быть достаточно информации для ответа на все вопросы.
% Формулировка каждого варианта задачи должна содержать всю необходимую информацию и не должна ссылаться на тексты внутри лекции, картинки или другие задачи или варианты задачи.
% Правильные ответы выделяются знаком «+» перед их формулировкой. Правильных ответов может быть несколько. Для тестов с несколькими ответами как минимум один ответ должен быть правильным и как минимум один ответ должен быть неправильным. 
% 
% Структура теста к лекции
% 
% \subsection*{Задача 1}
% 
% \paragraph{Вариант 1} 
% 
%     Чему равно 2+2?
%         Ответ 1. 3
%         + Ответ 2. 4
%         …
%         Ответ N. 5
% \paragraph{Вариант 2}
%     Чему равно 2*2?
%         + Ответ 1. 4
%         + Ответ 2. 2+2
%         …
%         Ответ N. 5
% \paragraph{Вариант 3}
% 
%     Чему равно 2-2?
%         Ответ 1. 0
% 
% 
%         
% \section{Просто подборка вопросов}
% 


 
 