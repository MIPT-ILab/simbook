\section{\Questions к главе \ref{caches}} %\label{caches-questions}

\subsection*{Вариант 1}

\begin{questions}
\question[3] Выберите правильные варианты продолжения фразы: использование кэшей при работе приложения целесообразно, если
\begin{choices}
    \correctchoice    программа показывает временную локальность доступов,
    \choice программа не обращается в оперативную память,
    \choice программа работает с очень большим объёмом данных,
    \correctchoice программа показывает пространственную локальность доступов,
    \choice программа работает с объёмом данных, меньшим ёмкости кэша.
\end{choices}

\question[3] Выберите все правильные окончания фразы: функциональные симуляторы часто не содержат в себе модель кэша, потому что
\begin{choices}
    \correctchoice они влияют только на задержки, но не на семантику инструкций,
    \choice всегда имеется возможность переиспользовать хозяйский кэш для нужд симуляции,
    \correctchoice такие модели сильно замедляют симуляцию.
\end{choices}

\question[3] Данные могут попадать в кэш при следующих операциях:
\begin{choices}
    \correctchoice чтение памяти (load),
    \correctchoice запись в память (store),
    \choice арифметические операции,
    \choice операции с числами с плавающей запятой,
    \correctchoice предвыборка данных (prefetch),
    \correctchoice загрузка инструкции (fetch),
    \choice инвалидация линии (invalidate).
\end{choices}

\end{questions}

\subsection*{Вариант 2}
\begin{questions}

\question[3] Выберите правильные варианты окончания: линия данных с фикированным адресом
\begin{choices}
    \choice всегда попадает в одну и ту же ячейку кэша,
    \correctchoice всегда попадает в один и тот же сет,
    \choice может быть сохранён в любой ячейке кэша.
\end{choices}

    \question[3] Выберите правильные варианты.
    \begin{choices}
        \choice Темпы роста скорости оперативной памяти и процессоров одинаковы с 80-х годов ХХ века.
        \choice Темп роста скорости оперативной памяти опережает темпы роста скорости работы процессоров.
        \correctchoice Темп роста скорости процессоров опережает темпы роста скорости оперативной памяти. 
    \end{choices}

\question[3] Кэши необходимо симулировать даже в функциональной модели, если они используются для
\begin{choices}
    \correctchoice создания транзакционной памяти,
    \choice моделирования работы системы с гарвардской архитектурой,
    \choice поддержания когерентности в SMP системах.
\end{choices}





\end{questions}

% К каждой лекции должно быть от 8 до 12 задач, у каждой задачи должно быть 3-5 вариантов формулировок примерно одинаковой сложности. Допускается объединение нескольких последовательных лекций в одну тему и подготовка тестов к темам.
% Задачи должны полностью соответствовать материалам лекций, то есть лекциях должно быть достаточно информации для ответа на все вопросы.
% Формулировка каждого варианта задачи должна содержать всю необходимую информацию и не должна ссылаться на тексты внутри лекции, картинки или другие задачи или варианты задачи.
% Правильные ответы выделяются знаком «+» перед их формулировкой. Правильных ответов может быть несколько. Для тестов с несколькими ответами как минимум один ответ должен быть правильным и как минимум один ответ должен быть неправильным. 
% 
% Структура теста к лекции
% 
% \subsection*{Задача 1}
% 
% \paragraph{Вариант 1} 
% 
%     Чему равно 2+2?
%         Ответ 1. 3
%         + Ответ 2. 4
%         …
%         Ответ N. 5
% \paragraph{Вариант 2}
%     Чему равно 2*2?
%         + Ответ 1. 4
%         + Ответ 2. 2+2
%         …
%         Ответ N. 5
% \paragraph{Вариант 3}
% 
%     Чему равно 2-2?
%         Ответ 1. 0
% 
% 
%         
% \section{Просто подборка вопросов}
% 


 
 