\section{\Questions к главе \ref{bt}} %\label{bt-questions}

\subsection*{Вариант 1}

\begin{questions}

\question[3] Какие виды программ обычно исполняются в непривилегированном режиме процессора?
\begin{choices}
    \correctchoice Пользовательские,
    \choice операционная система,
    \choice монитор виртуальных машин.
\end{choices}

\question[3] Какие из нижеперечисленных сценариев подпадают под определение \emph{самомодифицирующийся код}:
\begin{choices}
    \choice программа читает один байт  секции кода,
    \choice программа изменяет один байт в секции данных,
    \choice программа читает один байт из секции данных,
    \correctchoice программа изменяет байт в секции кода?
\end{choices}

\question[3] Для какого типа гостевой программы двоичная трансляция может показать производительность хуже, чем интерпретация той же программы?
\begin{choices}
    \choice Драйвер аппаратного устройства, например, клавиатуры,
    \correctchoice JIT-компилятор, например, языка Java или .NET,
    \choice Программа решения матричных уравнений,
    \choice Программа, выделяющая огромный объём памяти.
\end{choices}

\question[3] Перечислите особенности двоичной трансляции по сравнению с компиляцией языков высокого уровня, мешаюшие применению полного списка классических оптимизаций.
\begin{choices}
    \choice В коде нет комментариев.
    \correctchoice Отсутствуют границы процедур.
    \choice Неизвестен использованный при сборке компилятор и язык программирования.
    \correctchoice Отсутствует информация о переменных.
\end{choices}

\question[3] Выберите правильные составляющие задачи <<code discovery>> (обнаружение кода) в ДТ:
\begin{choices}
    \choice поиск кода внутри исполняемого файла,
    \correctchoice поиск границ инструкций при работе двоичного транслятора,
    \choice     поиск границ инструкций при работе интерпретатора,
    \correctchoice различение гостевого кода от гостевых данных,
    \choice     декодирование гостевых инструкций,
    \choice поиск некорректных гостевых инструкций.
\end{choices}

\question[3] Дайти определение понятия \emph{клей}, используемого в двоичной трансляции.
\begin{choices}
    \choice блок хозяйского машинного кода, моделирующий одну конкретную гостевую инструкцию,
    \correctchoice хозяйский код, необходимый для передачи управления между блоками трансляции,
    \choice гостевой код, необходимый для передачи управления между блоками трансляции,
    \choice инструкции, вставляемые в блок трансляции на фазе оптимизации.
\end{choices}

\question[3] Выберите верные утверждения, относящиеся к режиму прямого исполнения (DEX).
\begin{choices}
\correctchoice DEX позволяет исполнять большую часть гостевых инструкций без замедления.
\choice DEX используется для любых комбинаций гостевой и хозяйской архитектур.
\choice В режиме DEX исполняется код, полученный в результате двоичной трансляции.
\correctchoice Безопасное использование режима DEX возможно, только если хозяйская архитектура предоставляет специальный режим привилегий.
\end{choices}

\question[3] Выберите верные утверждения про свойства блоков трансляции (БТ)
\begin{choices}
    \choice Каждый БТ имеет более одной точки выхода.
    \correctchoice Каждый БТ имеет минимум одну точку входа.
    \correctchoice Каждый БТ состоит минимум из одной капсулы.
\end{choices}

\end{questions}

\subsection*{Вариант 2}

\begin{questions}

\question[3] Какой класс инструкций из перечисленных ниже наиболее сложен с точки зрения симуляции в режиме \textit{прямого исполнения}:
\begin{choices}
    \choice арифметические,
    \correctchoice привилегированные,
    \choice с плавающей запятой,
    \choice условные и безусловные переходы?
\end{choices}

\question[3] Какие виды программ обычно исполняются в привилегированном режиме процессора?
\begin{choices}
    \choice Пользовательские,
    \correctchoice операционная система,
    \correctchoice монитор виртуальных машин.
\end{choices}

\question[3] Определение понятия \emph{капсула}, используемого в двоичной трансляции.
\begin{choices}
    \correctchoice блок хозяйского машинного кода, моделирующий одну гостевую инструкцию,
    \choice хозяйский код, необходимый для передачи управления между блоками трансляции,
    \choice гостевой код, необходимый для передачи управления между блоками трансляции,
    \choice инструкции, вставляемые в блок трансляции на фазе оптимизации.
\end{choices}

\question[3] Какие порядки размеров капсул в системе двоичной трансляции наиболее вероятны:
\begin{choices}
    \choice 0 инструкций,
    \choice 1-2 инструкции,
    \correctchoice 10 -- 1000 инструкций,
    \choice 10000 — 100000 инструкций?
\end{choices}

\question[3] Выберите все необходимые условия корректности применения гиперсимуляции процессора:
\begin{choices}
    \choice нет обращений к внешней памяти,
    \correctchoice нет обращений к внешним устройствам,
    \choice только один процессор в системе,
    \correctchoice состояние внешних устройств не меняется,
    \choice состояние процессора не меняется.
\end{choices}

\question[3] Почему оптимизации, применяемые при статической двоичной трансляции, не используются для динамической?
\begin{choices}
    \correctchoice Фаза трансляции занимает слишком много времени,
    \choice они не дают прироста производительности генерируемого кода,
    \correctchoice они требуют знаний о глобальной структуре приложения.
\end{choices}


\question[3] Чем различаются понятия \textit{двоичный транслятор} (ДТ) и \textit{бинарный транслятор} (БТ):
\begin{choices}
    \choice ДТ генерирует динамический код, а БТ — статический,
    \choice БТ генерирует динамический код, а ДТ — статический,
    \correctchoice ничем, это синонимы?
\end{choices}

\question[3] Выберите верные утверждения, относящиеся к режиму прямого исполнения (DEX).
\begin{choices}
\choice DEX позволяет исполнять все существующие гостевые инструкции без замедления.
\correctchoice DEX может использоваться при совпадении гостевой и хозяйской архитектур.
\choice В режиме DEX исполняется код, полученный в результате интерпретации.
\correctchoice Безопасное использование режима DEX возможно, только если в нём исполняется только инструкции непривилегированного режима.
\end{choices}


\end{questions}


% К каждой лекции должно быть от 8 до 12 задач, у каждой задачи должно быть 3-5 вариантов формулировок примерно одинаковой сложности. Допускается объединение нескольких последовательных лекций в одну тему и подготовка тестов к темам.
% Задачи должны полностью соответствовать материалам лекций, то есть лекциях должно быть достаточно информации для ответа на все вопросы.
% Формулировка каждого варианта задачи должна содержать всю необходимую информацию и не должна ссылаться на тексты внутри лекции, картинки или другие задачи или варианты задачи.
% Правильные ответы выделяются знаком «+» перед их формулировкой. Правильных ответов может быть несколько. Для тестов с несколькими ответами как минимум один ответ должен быть правильным и как минимум один ответ должен быть неправильным. 
% 
% Структура теста к лекции
% 
% \subsection*{Задача 1}
% 
% \paragraph{Вариант 1} 
% 
%     Чему равно 2+2?
%         Ответ 1. 3
%         + Ответ 2. 4
%         …
%         Ответ N. 5
% \paragraph{Вариант 2}
%     Чему равно 2*2?
%         + Ответ 1. 4
%         + Ответ 2. 2+2
%         …
%         Ответ N. 5
% \paragraph{Вариант 3}
% 
%     Чему равно 2-2?
%         Ответ 1. 0
% 
% 
%         
% \section{Просто подборка вопросов}
% 


 
 