\section{\Questions к главе \ref{lang}} %\label{lang-questions}

\subsection*{Вариант 1}

\begin{questions}

\question[3] Какое утверждение наилучшим образом характеризует термин SystemC?
\begin{choices}
    \choice Компилятор языка Си с дополнениями для моделирования систем.
    \choice Язык программирования, похожий на Си.
    \choice Язык программирования, похожий на С++.
    \correctchoice Набор библиотек для С++.
\end{choices}

\question[3] Язык DML используется для разработки
\begin{choices}
    \correctchoice функциональных моделей,
    \choice потактовых моделей,
    \choice гибридных моделей.
\end{choices}

\question[3] Текущая реализация комилятора DMLC является
\begin{choices}
    \choice компилятором типа source-to-source с промежуточным языком С++,
    \choice компилятором, преобразующим исходный текст в байткод Java,
    \correctchoice компилятором типа source-to-source с промежуточным языком Си,
    \choice классическим компилятором,
    \choice частичным интерпретатором.
\end{choices}

\question[3] Закончите фразу: Языки разработки аппаратуры
\begin{choices}
\choice не используются для начального моделирования устройств, так как могут быть преобразованы только в netlist,
\correctchoice не используются для начального моделирования устройств, так как получаемые модели очень медленны,
\choice не используются для начального моделирования устройств, так как могут содержать в себе синтезируюмую часть,
\choice используются для начального моделирования устройств.
\end{choices}

\question[3] Что означает уровень <<X>> на входе логического элемента цифровой схемы?
\begin{choices}
    \correctchoice Фактическое значения сигнала не влияет на работу узла.
	\choice Напряжение на входе ниже значения, используемого для кодирования логического нуля.
    \choice Контакт не подключен ни к одному выходу схемы.
	\choice Cопротивление между входом и выходом сигнала очень велико.
\end{choices}

\end{questions}

\subsection*{Вариант 2}

\begin{questions}

\question[3] Какое утверждение наилучшим образом характеризует термин <<TLM>>?
\begin{choices}
    \choice Язык программирования, похожий на Си.
    \choice Язык программирования, похожий на С++.
    \choice Среда исполнения моделей DES.
    \correctchoice Развитие стандарта SystemC.
\end{choices}

\question[3] Язык DML используется для разработки
\begin{choices}
    \correctchoice неисполняющих моделей,
    \choice исполняющих моделей,
    \choice как исполняющих, так и неисполняющих моделей.
\end{choices}

\question[3] Какой способ наиболее удобен и надёжен для поддержания набора инструментов моделирования в синхронизированном состоянии при постоянном изменении входной спецификации процессора?
\begin{choices}
    \correctchoice Генерация всех инструментов из единого описания.
    \choice Тщательное сравнение всех инструментов после каждого изменения одного из них.
    \choice Создание одного инструмента, поддерживающего максимальное количество функций.
\end{choices}

\question[3] Закончите фразу: Синтезируемое подмножество языков разработки аппаратуры
\begin{choices}
\choice не может быть использовано для создания netlist и RTL-описаний,
\choice используется только для отладки моделей,
\correctchoice  используется для создания netlist и RTL-описаний.
\end{choices}

\question[3] Что означает уровень <<Z>> на входе логического элемента цифровой схемы?
\begin{choices}
    \correctchoice Контакт не подключен ни к одному выходу схемы.
	\choice Сигнал непрерывно изменяется в течение всего такта.
    \choice Фактическое значения сигнала не влияет на работу узла.
	\choice Напряжение на входе превышает значение, используемое для кодирования логической единицы.
\end{choices}

\end{questions}

% К каждой лекции должно быть от 8 до 12 задач, у каждой задачи должно быть 3-5 вариантов формулировок примерно одинаковой сложности. Допускается объединение нескольких последовательных лекций в одну тему и подготовка тестов к темам.
% Задачи должны полностью соответствовать материалам лекций, то есть лекциях должно быть достаточно информации для ответа на все вопросы.
% Формулировка каждого варианта задачи должна содержать всю необходимую информацию и не должна ссылаться на тексты внутри лекции, картинки или другие задачи или варианты задачи.
% Правильные ответы выделяются знаком «+» перед их формулировкой. Правильных ответов может быть несколько. Для тестов с несколькими ответами как минимум один ответ должен быть правильным и как минимум один ответ должен быть неправильным. 
% 
% Структура теста к лекции
% 
% \subsection*{Задача 1}
% 
% \paragraph{Вариант 1} 
% 
%     Чему равно 2+2?
%         Ответ 1. 3
%         + Ответ 2. 4
%         …
%         Ответ N. 5
% \paragraph{Вариант 2}
%     Чему равно 2*2?
%         + Ответ 1. 4
%         + Ответ 2. 2+2
%         …
%         Ответ N. 5
% \paragraph{Вариант 3}
% 
%     Чему равно 2-2?
%         Ответ 1. 0
% 
% 
%         
% \section{Просто подборка вопросов}
% 