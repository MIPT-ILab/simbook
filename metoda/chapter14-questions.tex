\section{\Questions к главе \ref{chapter14}} %\label{chapter14-questions}

\subsection*{Вариант 1}

\begin{questions}
\question[3] Может ли привилегированная инструкция когда-либо вызывать событие ловушки?
\begin{solution}[1cm]
Да, если это ловушка защиты памяти, т.е. инструкция обратилась к памяти, не входящей в текущий разрешённый сегмент.
\end{solution}

\question[1] Сколько режимов процессора используется в модели, описанной в работе Голдберга и Попека?
\begin{choices}
    \choice 1,
    \correctchoice 2,
    \choice 3,
    \choice 4.
\end{choices}

\question[3] Какие из указанных ниже ситуаций не нарушают принципа эквивалентности виртуального и реального окружений?
\begin{choices}
    \choice Инструкция \texttt{FOOBAR} имеет различающуюся семантику.
    \correctchoice Инструкция \texttt{FOOBAR} выполняется в два раза медленнее.
    \choice Инструкция \texttt{FOOBAR} не существует в хозяине.
    \correctchoice Инструкция \texttt{FOOBAR} не может обратиться к физической памяти, потому что внутри ВМ объём ОЗУ меньше.
\end{choices}

\question[3] Дайте определение понятия <<привилегированная инструкция>>.
\begin{solution}[1cm]
Инструкции, исполнение которых в режиме пользователя всегда вызывает ловушку потока управления.
\end{solution}

\question[3] Каким термином был обозначен новый режим процессора в системах, поддерживающих аппаратную виртуализацию?
\begin{choices}
    \choice kernel mode,
    \choice protected mode,
    \choice trusted mode,
    \correctchoice root mode.
\end{choices}

\end{questions}

\subsection*{Вариант 2}

\begin{questions}

\question[3] Выберите правильные варианты продолжения фразы: инструкция может одновременно быть привилегированной и
\begin{choices}
\choice безвредной,
\correctchoice служебной,
\choice безвредной и служебной.
\end{choices}

\question[3] Какие переходы между режимами возможны при возникновении события ловушки?
\begin{choices}
\choice        Из привилегированного в пользовательский.
\correctchoice Из пользовательского в привилегированный.
\correctchoice Из привилегированного в привилегированный.
\choice        Из пользовательского в пользовательский.
\end{choices}

\question[3] Дайте определение понятия <<безвредная инструкция>>.
\begin{solution}[1cm]
Инструкция, не являющаяся служебной.
\end{solution}

\question[3] Какие из нижеперечисленных особенностей реальных ЭВМ опущены в модели Голдберга и Попека?
\begin{choices}
    \correctchoice Существование внешних прерываний.
    \choice Присуствие оперативной памяти.
    \correctchoice Наличие внешних постоянных хранилищ.
    \choice Механизмы виртуальной памяти.
\end{choices}

\question[3] Каким образом можно избежать излишне частого сброса содержимого TLB при работе нескольких ВМ?
\begin{solution}[1cm]
Использовать TLB с поддержкой тэгов для того, чтобы помечать принадлежащие независимым ВМ записи.
\end{solution}

\end{questions}
