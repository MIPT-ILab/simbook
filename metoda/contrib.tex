\chapter*{Об этой книге}\label{chap:contrib}

Главы данной книги соответствуют основным лекциям курса <<Основы программного моделирования ЭВМ>>, читаемого в Московском физико-техническом институте.

Нам очень важно мнение читателя. Если вы обнаружили опечатку, стилистическую, фактическую ошибку, которые, более чем вероятно, встречаются в тексте, или имеете замечания по содержанию и предложения по тому, как можно улучшить данный материал, то просим сообщить об этом по e-mail 

\begin{center}
\href{mailto:grigory.rechistov@phystech.edu}{\texttt{grigory.rechistov@phystech.edu}}
\end{center}


%\section*{Авторы}

%Данную книгу подготовил следующий коллектив лаборатории суперкомпьютерных технологий для биомедицины, фармакологии и малоразмерных структур им. В.~М.~Пентковского МФТИ: Г.~С.~Речистов, Е.~А.~Юлюгин, А.~А.~Иванов, П.~Л.~Шишпор, Н.~Н.~Щелкунов.

Актуальная версия текста данной книги доступна в Интернет по адресу:

% \url{http://iscalare.mipt.ru/materials/course_materials/}.
{\scriptsize\url{http://atakua.doesntexist.org/wordpress/simulation-course-russian/}}

\section*{Благодарности}

Авторы выражают также благодарность всем слушателям курса и следующим людям, сообщившим свои замечания и исправления к тексту книги: Илье Куприку, Денису Шиляеву, Денису Лытову, Анатолию Костину, Виталию Антонову, Даниилу Альфонсо, Дмитрию Бородий, Ивану Андрееву, Наталье Иванчиковой, Марине Шимченко, Максиму Кузнецову, Святославу Кузьмичу, Егору Кривову, Кириллу Ашейчику, Аюпову Амиру, Даниилу Саргину, Леониду Снегирёву, Александру Кравцову.

Некоторые секции книги были первоначально опубликованы как посты на сайте Хабрахабр: \url{http://habrahabr.ru/company/intel/blog/}. Здесь они включены в адаптированном варианте.

% Хабраюзеру valplo

