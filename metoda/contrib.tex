\chapter*{Об этой книге}\label{chap:contrib}

Главы данной книги соответствуют лекциям курса <<Основы программного моделирования ЭВМ>>, читаемого в Московском физико-техническом институте.

Нам очень важно мнение читателя. Если вы обнаружили опечатку, стилистическую, фактическую ошибку, которые c большой вероятностью встречаются в тексте, или имеете замечания по содержанию и предложения по тому, как можно улучшить данные материалы, то просим сообщить об этом по e-mail 

\begin{center}
\href{mailto:grigory.rechistov@phystech.edu}{\texttt{grigory.rechistov@phystech.edu}}
\end{center}


\section*{Предыдущие издания книги}

Над первыми двумя изданиями книги работал следующий коллектив авторов: Г.~С.~Речистов, Е.~А.~Юлюгин, А.~А.~Иванов, П.~Л.~Шишпор, Н.~Н.~Щелкунов, Д.~А.~Гаврилов.

Актуальная версия текста данной книги доступна в Интернет по адресу:

% \url{http://iscalare.mipt.ru/materials/course_materials/}.
{\scriptsize\url{http://atakua.doesntexist.org/wordpress/simulation-course-russian/}}

\section*{Благодарности}

%\dictum{}

Авторы выражают благодарность всем студентам-слушателям курса. Следующие люди сообщали свои замечания и предлагали исправления к тексту книги: Илья Куприк, Денис Шиляев, Денис Лытов, Анатолий Костин, Виталий Антонов, Даниил Альфонсо, Дмитрий Бородий, Иван Андреев, Наталья Иванчикова, Марина Шимченко, Максим Кузнецов, Святослав Кузьмич, Егор Кривов, Кирилл Ашейчик, Амир Аюпов, Даниил Саргин, Леонид Снегирёв, Александр Кравцов, Валерий Конычев, Всеволод Ливинский, Александра Цветкова.

Некоторые секции книги были первоначально опубликованы как посты на сайте Хабрахабр: \url{http://habrahabr.ru/company/intel/blog/}. Здесь они включены в адаптированном варианте.


