\section{\Questions к главе \ref{parallel}} %\label{parallel-questions}

\subsection*{Вариант 1}

\begin{questions}

\question[3] Выберите верные варианты продолжения фразы: при параллельном моделировании SMP-системы
\begin{choices}
    \choice один хозяйский поток симулирует ровно один гостевой процессор,
    \correctchoice один хозяйский поток симулирует один или более гостевых процессоров,
    \choice несколько хозяйских потоков симулируют один гостевой процессор.
\end{choices}

\question[3] Выберите проблемы, которые необходимо учитывать и решать при \textit{создании алгоритма} параллельного симулятора.
\begin{choices}
    \correctchoice Возможность гонок данных.
    \correctchoice Возможность взаимоблокировок процессов.
    \choice Неэффективная работа параллельного приложения.
    \choice Недетерминизм симуляции.
\end{choices}

\question[3] Какие из типов схем PDES позволяют добиться детерминизма симуляции?
\begin{choices}
    \correctchoice Барьерная (с доменами синхронизации).
    \choice Консервативная.
    \choice Оптимистичная.
    \choice Наивная.
\end{choices}

\question[3] Какие проблемы создаёт излишне частая отправка пустых (null) сообщений в консервативной схеме PDES с детектированием взаимоблокировок?
\begin{choices}
    \correctchoice Низкая скорость симуляции при большом числе потоков.
    \choice Некорректная симуляция при малом числе потоков.
    \choice Некорректная симуляция при большом числе потоков.
    \choice Возможность взаимоблокировок при большом числе потоков.
\end{choices}

\question[3] Выберите правильные свойства атомарных инструкций.
\begin{choices}
    \choice Имеют только один операнд, являющйся регистром.
    \choice Имеют только один операнд, находящийся в общей памяти.
    \correctchoice Имеют операнд-назначение, находящийся в общей памяти.
    \correctchoice Выглядят как неделимые для всех потоков исполнения.
    \choice В результате работы всегда изменяют значение своего операнда-назначения.
\end{choices}

\question[3] Выберите верные свойства последовательной симуляции, связанные со свойствами атомарных инструкций.
\begin{choices}
    \correctchoice Все гостевые инструкции в последовательной симуляции исполняются атомарным образом.
    \choice Атомарные инструкции требуют более сложных сервисных процедур для своей эмуляции, чем это требуется в параллельных симуляторах.
    \choice Невозможно обеспечить корректную симуляцию атомарных инструкций в последовательных симуляторах.
\end{choices}

\question[3] Выберите верное продолжение фразы: для корректной работы параллельного симулятора модель консистентности памяти хозяйской системы 
\begin{choices}
    \choice должна быть эквивалентна модели гостя,
    \correctchoice должна быть более строгой, чем используемая в госте,
    \choice должна быть более слабой, чем используемая в госте,
    \choice может быть как более строгой, так и более слабой по сравнению с используемой в госте.
\end{choices}


% \question[3] Выберите правильные продолжения фразы: Частая отправка пустых (null)  сообщений нежелательна, так как
% \begin{choices}
%     \correctchoice это может ограничивать скорость симуляции,
%     \choice это может вызвать нарушение каузальности симуляции,
%     \choice это может привести к взаимоблокировке потоков,
%     \choice это может привести к переполнению очередей сообщений.
% \end{choices}

% \question[3] Выберите правильные ответы.
% \begin{choices}
%     \choice Симуляция, реализованная с помощью схемы PDES, всегда детерминистична.
%     \choice Симуляция, реализованная с помощью схемы PDES, недетерминистична из-за возможности потери сообщений между потоками.
%     \choice Симуляция, реализованная с помощью схемы PDES, недетерминистична из-за возможности блокировки отдельных потоков.
%     \choice Симуляция, реализованная с помощью схемы PDES, недетерминистична из-за варьирующейся скорости работы отдельных потоков.
% \end{choices}


\end{questions}

\subsection*{Вариант 2}

\begin{questions}

\question[3] Почему не будет работать \textbf{наивная} схема параллельного DES? Выберите верные ответы.
\begin{choices}
    \correctchoice Недетерминизм модели.
    \choice Невозможно организовать передачу сообщений между потоками.
    \correctchoice Возможно нарушение каузальности.
    \choice Невозможно подобрать точно квоту выполнения.
\end{choices}

% \question[3] Чем чревата недостаточно частая отправка пустых (null) сообщений в консервативной схеме PDES с детектированием взаимоблокировок?
% \begin{solution}[1cm]
% Низкой скоростью работы симулятора из-за частой блокировки потоков, слишком далеко убежавших в симулируемое будущее.
% \end{solution}

\question[3] Выберите правильные ответы.
\begin{choices}
    \correctchoice Консервативные схемы PDES не допускают нарушения каузальности.
    \choice Консервативные схемы PDES допускают нарушения каузальности.
    \choice Консервативные схемы PDES допускают нарушения каузальности, но впоследствии их корректируют.
    \choice Оптимистичные схемы PDES не допускают нарушения каузальности.
    \choice Оптимистичные схемы PDES допускают нарушения каузальности.
    \correctchoice Оптимистичные схемы PDES допускают нарушения каузальности, но впоследствии их корректируют.
\end{choices}

\question[3] Выберите правильные свойства домена синхронизации в модели PDES.
\begin{choices}
    \choice Количество моделируемых устройств внутри одного домена фиксировано.
    \choice Не происходит взаимодействия устройств, находящихся в различных доменах.
    \choice Количество моделируемых устройств внутри одного домена ограничено.
    \correctchoice Характерная частота коммуникаций  между доменами превышает частоту коммуникаций внутри каждого.
    \choice Характерная частота коммуникаций  между доменами равна частоте коммуникаций внутри отдельного домена.
\end{choices}

\question[3] Выберите верные утверждения о симуляции атомарных инструкций.
\begin{choices}
    \correctchoice Гостевые атомарные инструкции могут быть реализованы, если хозяйская система имеет аппаратную поддержку транзакционной памяти, например, в форме пары load linked/store сonditional.
    \correctchoice Гостевые атомарные инструкции могут быть реализованы, если хозяйская система имеет аппаратную поддержку  атомарной операции <<сравнить и обменять значения>>.
    \choice Гостевые атомарные инструкции всегда могут быть просимулированы с использованием хозяйских атомарных инструкций.
    \choice Гостевые атомарные инструкции всегда могут быть просимулированы с использованием критических секций, например, одного глобального замка.
\end{choices}

\question[3] С точки зрения эффективности работы системы распределённой общей памяти наиболее дорогой операцией являеется:
\begin{choices}
    \correctchoice Запись в регион одним потоком при условии, что несколько других потоков читают этот же регион.
    \choice Запись в регион одним потоком при условии, что обращения остальных потоков к нему редко.
    \choice Одновременное чтение региона памяти несколькими потоками.
    \choice Исполнение кода из региона памяти, выполняемое сразу несколькими потоками.
\end{choices}

\question[3] Перечислите причины, по которым необходимо проводить балансировку нагрузки на хозяйские потоки в параллельной симуляции.
\begin{choices}
    \choice Для уменьшения объёма потребляемой симуляцией памяти.
    \choice Для предотвращения взаимоблокировок в консервативных схемах.
    \choice Для минимизации числа откатов в оптимистичных схемах.
    \correctchoice Для повышения производительности симуляции.
\end{choices}

\question[3] Какая из перечисленных ниже ситуаций может создавать недетерминистичность при параллельной симуляции?
\begin{choices}
    \choice Событие, находящееся в будущем.
    \choice Событие, находящееся в прошлом.
    \choice Два события в разных потоках, имеющих одинаковую метку времени.
    \correctchoice Два события в одном  потоке, имеющие одинаковую метку времени.
    \choice Отставшее сообщение в оптимистичной схеме.
    \choice Пустое сообщение в консервативной схеме.
\end{choices}


\end{questions}

% К каждой лекции должно быть от 8 до 12 задач, у каждой задачи должно быть 3-5 вариантов формулировок примерно одинаковой сложности. Допускается объединение нескольких последовательных лекций в одну тему и подготовка тестов к темам.
% Задачи должны полностью соответствовать материалам лекций, то есть лекциях должно быть достаточно информации для ответа на все вопросы.
% Формулировка каждого варианта задачи должна содержать всю необходимую информацию и не должна ссылаться на тексты внутри лекции, картинки или другие задачи или варианты задачи.
% Правильные ответы выделяются знаком «+» перед их формулировкой. Правильных ответов может быть несколько. Для тестов с несколькими ответами как минимум один ответ должен быть правильным и как минимум один ответ должен быть неправильным. 
% 
% Структура теста к лекции
% 
% \subsection*{Задача 1}
% 
% \paragraph{Вариант 1} 
% 
%     Чему равно 2+2?
%         Ответ 1. 3
%         + Ответ 2. 4
%         …
%         Ответ N. 5
% \paragraph{Вариант 2}
%     Чему равно 2*2?
%         + Ответ 1. 4
%         + Ответ 2. 2+2
%         …
%         Ответ N. 5
% \paragraph{Вариант 3}
% 
%     Чему равно 2-2?
%         Ответ 1. 0
% 
% 
%         
% \section{Просто подборка вопросов}
% 
 