\section{\Questions к главе \ref{chapter01}} %\label{chapter01-questions}

% \begin{choices}
% \correctchoice 
% \choice 
% \end{choices}


\subsection*{Вариант 1}

\begin{questions}

\question[5] Определите понятие <<функциональный симулятор>>.
\begin{choices}
\correctchoice Модель, обеспечивающая корректное выполнение алгоритмов отдельных инструкций.
\choice Модель, гарантирующая правильность длительностей симулируемых операций.
\choice Модель, задача которой состоит в максимально качественном представлении одной функции.
\choice Модель, показывающая максимальную производительность при работе.
\end{choices}

\question[5] Определите понятие <<полноплатформенный симулятор>>.
\begin{choices}
\choice Модель, ограниченная в точности уровнем платформы.
\choice Модель, обеспечивающая корректную работу гостевых пользовательских приложений.
\correctchoice Симулятор, способный запускать операционные системы и потому содержащий модели периферийных устройств.
\end{choices}

\question[3] Перечислите все правильные виды сложностей, возникающих при разработке цифровых систем, успешно решаемых с помощью моделирования.
\begin{choices}
\correctchoice Необходимость знать характеристики новой технологии как можно раньше.
\correctchoice Необходимость выявления ошибок проектирования на ранних стадиях.
\choice Большое энергопотребление реальных образцов.
\correctchoice Малое количество опытных образцов и их высокая цена.
\end{choices}

\question[3] Выберите правильные условия изоляции исполнения гостевого приложения.
\begin{choices}
\correctchoice Приложение не должно иметь возможности обнаружить, что оно исполняется оно внутри виртуальной машины или на реальной аппаратуре
\choice Приложение не испытывает наблюдаемого уменьшения в производительности при симуляции по сравнению с реальной аппаратурой.
\choice Приложение не может обращаться к определённому набору присутствующих на реальной аппаратуре ресурсов.
\correctchoice Приложение не должно иметь возможности обнаружить, исполняются ли помимо него другие гости.
\end{choices}

\question[3] Как расшифровывается обозначение <<RTL-модель>> в контексте разработки аппаратуры?
\begin{choices}
\choice Run-time library.
\choice Register-transistor logic.
\choice Real-time layer.
\correctchoice Register transfer level.
\end{choices}

\question[3] Выберите правильные свойства гипервизора первого типа.
\begin{choices}
\choice Работают внутри операционной системы.
\correctchoice Не требуют для своей работы операционной системы.
\choice Могут работать как под управлением ОС, так и без неё.
\end{choices}

\question[3] Определение величины MIPS, используемой для измерения скорости программ.
\begin{choices}
\correctchoice Количество миллионов инструкций, исполняющихся за одну секунду.
\choice Число секунд, требуемых для исполнения одной инструкции.
\choice Число тактов, требуемых для исполнения одной инструкции.
\choice Количество макрокоманд, исполняющихся за одну секунду.
\end{choices}
    
% \question[3] Какой из указанных ниже бенчмарков используется для оценки и сравнения эффективности работы систем виртуализации:
% \begin{choices}
% \choice SPECfp,
% \choice SPECpower,
% \choice SPECint,
% \correctchoice SPECvirt,
% \choice SPECjbb?
% \end{choices}

\end{questions}
\subsection*{Вариант 2}

\begin{questions}

\question[3] Определите понятие <<потактовый симулятор>>.
\begin{choices}
\choice Модель, обеспечивающая корректное выполнение алгоритмов отдельных инструкций.
\correctchoice Модель, правильным образом высчитывающая задержки, связанные с исполнением отдельных операций.
\choice Модель, показывающая максимальную производительность при работе.
\end{choices}

\question[3] Определите понятие <<симулятор уровня приложений>>.
\begin{choices}
\choice Модель, ограниченная в точности уровнем платформы.
\choice Модель, обеспечивающая корректную работу гостевых пользовательских приложений.
\correctchoice Симулятор, способный запускать операционные системы и потому содержащий модели периферийных устройств.
\end{choices}

\question[3] Перечислите все правильные виды сложностей, возникающих при разработке цифровых систем, успешно решаемых с помощью моделирования.
\begin{choices}
\correctchoice Большое число составляющих систему устройств со сложными взаимосвязями.
\choice Сложность получения лицензий на новое оборудование.
\correctchoice Обеспечение поддержки аппаратуры программными средствами разработки.
\choice Опасность конкурентного шпионажа.
\end{choices}

\question[3] Перечислите стадии создания нового устройства с задействованием моделирования в правильном порядке.
\begin{choices}
\choice Функциональная модель.
\choice Разработка концепции устройства.
\choice RTL-модель.
\choice Потактовая модель.
\choice Выпуск продукции на рынок.
\choice Экспериментальные образцы.
\end{choices}
\begin{solution}
    Правильная последовательность: 2 -- 1 -- 4 -- 3 -- 6 -- 5.
\end{solution}

\question[3] Определение гибридного симулятора.
\begin{choices}
\correctchoice Модель, частично реализованная в программе для обычного компьютера, а частично --- на специализированном оборудовании.
\choice Модель, способная работать как в режиме потактового, так и в режиме функционального симулятора.
\choice Модель, имеющая два режима работы: первый --- полноплатформенный, второй --- режима приложения.
\choice Модель, работающая как гипервизор первого типа, но имеющая функции гипервизора второго типа.
\end{choices}

\question[3] Определение гипервизора второго типа.
\begin{choices}
\correctchoice Работают внутри операционной системы.
\choice Не требуют для своей работы операционной системы.
\choice  Могут работать как под управлением ОС, так и без неё.
\end{choices}

\question[3] Определение понятия FLOPS.
\begin{choices}
\choice Число арифметических операций над числами с плавающей запятой, выполняемых за один такт.
\choice Число арифметических операций над числами с фиксированной запятой, выполняемых за одну секунду.
\correctchoice Число арифметических операций над числами с плавающей запятой, выполняемых за одну секунду.
\choice Число секунд, требуемых для выполнения одной арифметической операции над числами с фиксированной запятой.
\end{choices}

% \question[3] Определение понятия \textit{floating point number}.
% \begin{solution}[1cm]
% Число с плавающей запятой, записываемое в формате $$mantissa \cdot 2^{exponent}$$ $$1 \le mantissa < 2.$$
% \end{solution}

\end{questions}

% К каждой лекции должно быть от 8 до 12 задач, у каждой задачи должно быть 3-5 вариантов формулировок примерно одинаковой сложности. Допускается объединение нескольких последовательных лекций в одну тему и подготовка тестов к темам.
% Задачи должны полностью соответствовать материалам лекций, то есть лекциях должно быть достаточно информации для ответа на все вопросы.
% Формулировка каждого варианта задачи должна содержать всю необходимую информацию и не должна ссылаться на тексты внутри лекции, картинки или другие задачи или варианты задачи.
% Правильные ответы выделяются знаком «+» перед их формулировкой. Правильных ответов может быть несколько. Для тестов с несколькими ответами как минимум один ответ должен быть правильным и как минимум один ответ должен быть неправильным. 
% 
% Структура теста к лекции
% 
% \subsection*{Задача 1}
% 
% \paragraph{Вариант 1} 
% 
%     Чему равно 2+2?
%         Ответ 1. 3
%         + Ответ 2. 4
%         …
%         Ответ N. 5
% \paragraph{Вариант 2}
%     Чему равно 2*2?
%         + Ответ 1. 4
%         + Ответ 2. 2+2
%         …
%         Ответ N. 5
% \paragraph{Вариант 3}
% 
%     Чему равно 2-2?
%         Ответ 1. 0
% 
% 
%         
% \section{Просто подборка вопросов}
% 


 
 