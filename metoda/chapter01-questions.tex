\section{\Questions к главе \ref{chapter01}} %\label{chapter01-questions}

\subsection*{Вариант 1}

\begin{questions}

\question[5] Определите понятие <<функциональный симулятор>>.
\begin{solution}[2cm]
Модель, обеспечивающая корректное выполнение алгоритмов отдельных инструкций, но при этом не гарантирующая корректность симулируемых длительностей операций.
\end{solution}


\question[5] Определите понятие <<полноплатформенный симулятор>>.
\begin{solution}[2cm]
Симулятор,  способный запускать операционные системы и потому содержащий модели периферийных устройств.
\end{solution}

\question[3] Перечислите все правильные виды сложностей, возникающих при разработке цифровых систем, успешно решаемых с помощью моделирования.
\begin{choices}
\correctchoice Необходимость знать характеристики новой технологии как можно раньше.
\correctchoice Необходимость выявления ошибок проектирования на ранних стадиях.
\choice Большое энергопотребление реальных образцов.
\end{choices}

\question[3] Критерий изоляции исполнения гостевого приложения.
\begin{solution}[1cm]
Приложение не должно иметь возможности обнаружить следующие факты: 1) исполняется оно внутри виртуальной машины или на реальной аппаратуре; 2) исполняются ли помимо него другие гости. Приложение не должно иметь возможность безконтрольно изменять состояние монитора виртуальных машин.
\end{solution}

\question[3] Как расшифровывается обозначение <<RTL-модель>> в контексте разработки аппаратуры?
\begin{choices}
\choice Run-time library.
\correctchoice Register transfer level.
\choice Register-transistor logic.
\end{choices}

\question[3] Определение гипервизора первого типа.
\begin{solution}[2cm]
Гипервизоры первого типа (автономные гипервизоры) работают прямо на хозяйской аппаратуре, т.е. не требуют для своей работы операционной системы, беря её функции на себя и являясь привилегированными приложениями.
\end{solution}

\question[3] Определение величины MIPS, используемой для измерения скорости программ.
\begin{solution}[1cm]
Количество миллионов инструкций, исполняющихся за одну секунду.
\end{solution}
    
\question[3] Какой из указанных ниже бенчмарков используется для оценки и сравнения эффективности работы систем виртуализации:
\begin{choices}
\choice SPECfp,
\choice SPECpower,
\choice SPECint,
\correctchoice SPECvirt,
\choice SPECjbb?
\end{choices}


\end{questions}
\subsection*{Вариант 2}

\begin{questions}

\question[3] Определение потактового симулятора.
\begin{solution}[2cm]
Модель, обеспечивающая корректное выполнение алгоритмов отдельных инструкций и при этом высчитывающая задержки, возникающие при их исполнении.
\end{solution}

\question[3] Определение симулятора уровня приложений.
\begin{solution}[2cm]
Модель, обеспечивающая корректную работу гостевых пользовательских приложений, состоящих из непривилегированных инструкций, а также эмулирующая базовый набор системных вызовов некоторой ОС.
\end{solution}

\question[3] Перечислите все правильные виды сложностей, возникающих при разработке цифровых систем, успешно решаемых с помощью моделирования.
\begin{choices}
\correctchoice Большое число составляющих систему устройств со сложными взаимосвязями.
\choice Сложность получения лицензий на новое оборудование.
\correctchoice Обеспечение поддержки аппаратуры программными средствами разработки.
\end{choices}

\question[3] Перечислите стадии создания нового устройства с задействованием моделирования в правильном порядке.
\begin{choices}
\choice Функциональная модель.
\choice Разработка концепции устройства.
\choice RTL-модель.
\choice Потактовая модель.
\choice Выпуск продукции на рынок.
\choice Экспериментальные образцы.
\end{choices}
\begin{solution}
    Правильная последовательность: 2 -- 1 -- 4 -- 3 -- 6 -- 5.
\end{solution}

\question[3] Определение гибридного симулятора.
\begin{solution}[2cm]
Модель, частично реализованная в программе для обычного компьютера, а частично --- на специализированном оборудовании, например, на ПЛИС.
\end{solution}

\question[3] Определение гипервизора второго типа.
\begin{solution}[2cm]
Гипервизоры второго типа  не заменяют операционную систему, но работают поверх неё как обычное пользовательское приложение.
\end{solution}

\question[3] Определение понятия FLOPS.
\begin{solution}[1cm]
Количество арифметических операций над числами с плавающей запятой, выполняемых за одну секунду.
\end{solution}

\question[3] Определение понятия \textit{floating point number}.
\begin{solution}[1cm]
Число с плавающей запятой, записываемое в формате $$mantissa \cdot 2^{exponent}$$ $$1 \le mantissa < 2.$$
\end{solution}


\end{questions}


% К каждой лекции должно быть от 8 до 12 задач, у каждой задачи должно быть 3-5 вариантов формулировок примерно одинаковой сложности. Допускается объединение нескольких последовательных лекций в одну тему и подготовка тестов к темам.
% Задачи должны полностью соответствовать материалам лекций, то есть лекциях должно быть достаточно информации для ответа на все вопросы.
% Формулировка каждого варианта задачи должна содержать всю необходимую информацию и не должна ссылаться на тексты внутри лекции, картинки или другие задачи или варианты задачи.
% Правильные ответы выделяются знаком «+» перед их формулировкой. Правильных ответов может быть несколько. Для тестов с несколькими ответами как минимум один ответ должен быть правильным и как минимум один ответ должен быть неправильным. 
% 
% Структура теста к лекции
% 
% \subsection*{Задача 1}
% 
% \paragraph{Вариант 1} 
% 
%     Чему равно 2+2?
%         Ответ 1. 3
%         + Ответ 2. 4
%         …
%         Ответ N. 5
% \paragraph{Вариант 2}
%     Чему равно 2*2?
%         + Ответ 1. 4
%         + Ответ 2. 2+2
%         …
%         Ответ N. 5
% \paragraph{Вариант 3}
% 
%     Чему равно 2-2?
%         Ответ 1. 0
% 
% 
%         
% \section{Просто подборка вопросов}
% 


 
 