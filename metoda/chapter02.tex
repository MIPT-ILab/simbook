\chapter[Обзор существующих полноплатформенных симуляторов]{Обзор существующих полноплатформенных симуляторов систем, основанных на архитектуре IA-32}\label{chapter02}

\dictum[Станислав Лем. Кибериада. Сказки роботов.]{Тут-то гениальный Цереброн, атаковав проблему методами точных наук, установил, что имеется три типа драконов: нулевые, мнимые и отрицательные. Все они, как было сказано, не существуют, однако каждый тип --- на свой особый манер.}

\todo Переписать

% \section{Терминология и классификация}
% Существует большое количество полноплатформенных симуляторов, как коммерческих, так и открытых. Области их применения разнообразны. В данной главе мы предложим классификацию полноплатформенных симуляторов, опирающуюся на их устройство; расскажем о принципах действия, которые лежат в основе симуляторов, а также опишем наиболее известные полноплатформенные симуляторы и области их применения. Речь будет идти главным образом о симуляторах платформ, основанных на архитектуре процессоров Intel\textup{\textregistered} IA-32.

% Как было сказано ранее в главе~\ref{chapter01}, полноплатформенный симулятор --- модель, представляющая всю вычислительную систему, включающая в себя процессор, материнскую плату, периферийные устройства, оперативную память и другие устройства, необходимые для привычной работы пользовательских и системных приложений. 

%Термины \textit{виртуальная машина} и \textit{полноплатформенный симулятор} --- синонимы. Программу, позволяющую запускать и управлять работой нескольких виртуальных машин на одной реальной, называют \textit{гипервизором} или \textit{монитором виртуальных машин} (\abbr virtual machine monitor). %Применение полноплатформенных симуляторов для виртуализации называют полной виртуализацией. 

\section{Характеристики полноплатформенных симуляторов}

В связи с тем, что поведение симуляторов всё же часто отличается от поведения реальных платформ, важно также знать, какие гостевые операционные системы удалось запустить под его управлением. Например, может оказаться, что какая-нибудь недавно вышедшая операционная система на некотором продукте ещё не загружается. Для программ, запускаемых как прикладные задачи внутри хозяйской операционной системы, дополнительно необходимо знать, для каких операционных систем они были скомпилированы. 

Таким образом, наиболее полно сценарий использования некоторого симулятора описывается четвёркой понятий (\textit{Хозяйский ЦПУ, хозяйская ОС}) \textrightarrow (\textit{Гостевой ЦПУ, гостевая ОС}). Пример такого описания для запуска Qemu: (x86, Windows) \textrightarrow (amd64, Linux); эту запись следует понимать как <<Linux на процессоре архитектуры Intel\texttrademark EM64T (64-битном) симулируется на процессоре архитектуры Intel\textregistered IA-32 (32-битном) под управлением ОС Windows>>.

Описанные выше подходы к измерению производительности неудовлетворительны для гипервизоров, которые управляют работой многих виртуальных машин, работающих на одной системе. Описанная ситуация весьма типична для сценариев виртуализации и консолидации. Для адекватной оценки производительности гипервизоров была развита соответствующая методология и созданы эталонные тесты (например, некоммерческой организацией Standard Performance Evaluation Corporation разработан тест SPECvirt~\cite{specvirt}). 

\iftoggle{hasquiz}{
    \section{Вопросы к главе \ref{chapter02}} %\label{chapter02-questions}

\todo Убрать

 
}{}

\iftoggle{webpaper}{
    \printbibliography[title={Литература}]
}{}