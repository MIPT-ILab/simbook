\section{\Questions к главе \ref{chapter03}} %\label{chapter03-questions}

\subsection*{Вариант 1}

\begin{questions}
\question[1] Какие из указанных ниже компонентов обязательны для реализации интерпретатора:
\begin{choices}
    \correctchoice декодер,
    \choice дизассемблер,
    \choice кодировщик (енкодер),
    \correctchoice блоки реализации семантики отдельных инструкций,
    \choice кэш декодированных инструкций.
\end{choices}

\question[3] Опишите, что происходит на стадии Fetch работы процессора.
\begin{solution}[1cm]
Четние из памяти машинного кода, соответствующего текущей инструкции.
\end{solution}

\question[3] Опишите, что происходит на стадии \textbf{Writeback} работы процессора. Для каких инструкций эта стадия будет опущена?
\begin{solution}[2cm]
Запись результатов исполнения инструкции в память. Если результат должен быть сохранён в регистре, то фаза опускается.
\end{solution}

\question[3] Какой вид программ обычно исполняется в привилегированном режиме процессора?
\begin{solution}[1cm]
Операционные системы, мониторы виртуальных машин первого типа.
\end{solution}

\question[1] Какие эффекты могут наблюдаться при невыровненном (unaligned) чтении из памяти в существующих архитектурах:
\begin{choices}
    \correctchoice возникновение исключения,
    \correctchoice     замедление операции по сравнению с аналогичной выровненной,
    \choice данные будут считаны лишь частично,
    \choice  возможны все перечисленные выше ситуации?
\end{choices}

\question[3] Какая из следующих типов ситуаций при исполнении процессора является асинхронной по отношению к работе текущей инструкции?
\begin{choices}
    \correctchoice прерывание (interrupt),
    \choice ловушка (trap),
    \choice исключение (exception),
    \choice промах (fault)?
\end{choices}

\question[3] Выберите правильный вариант окончания фразы: Сцепленный интерпретатор работает быстрее переключаемого (switched), так как
\begin{choices}
    \correctchoice удачно использует предсказатель переходов хозяйского процессора,
    \choice кэширует недавно исполненные инструкции,
    \choice транслирует код в промежуточное представление,
    \choice не требует обработки исключений.
\end{choices}

\end{questions}

\subsection*{Вариант 2}

\begin{questions}

\question[3] Какой из типов регистров всегда присутствует во всех классических архитектурах:
\begin{choices}
\choice указатель стека,
\choice аккумулятор,
\correctchoice указатель текущей инструкции,
\choice регистр флагов,
\choice индексный регистр.
\end{choices}

\question[3] Опишите, что происходит на стадии \textbf{Decode} работы процессора.
\begin{solution}[1cm]
Анализ машинного слова считанной инструкции для определения кода операции и операндов.
\end{solution}

\question[3] Опишите, что происходит на стадии \textbf{Advance PC} работы процессора. Для каких инструкций эта стадия будет опущена?
\begin{solution}[2cm]
Изменение указателя текущей инструкции таким образом, чтобы он указывал на инструкцию, следующую за только что исполненной. Если же произошла передача управления, то счётчик инструкций уже был изменён, и фаза опускается.
\end{solution}

\question[3] Какой вид программ обычно исполняется в непривилегированном режиме процессора?
\begin{solution}[1cm]
Пользовательские приложения.
\end{solution}


\question[3] Почему самый простой вид декодера машинных инструкций --- однотабличный --- не пользуется большой популярностью?
\begin{solution}[1cm]
    Размер таблицы растёт экспоненциально от длины опкода. Для архитектур, использующих больше 16 бит для кодирования инструкций, такая таблица становится несообразно огромной.
\end{solution}

\question[3] Выберите правильные варианты окончания фразы: Наличие единственного \texttt{switch} для всех гостевых инструкций в коде интерпретатора
\begin{choices}
    \choice увеличивает его скорость по сравнению со схемой сцепленной интерпретации,
    \choice упрощает его алгоритмическую структуру по сравнению со схемой сцепленной интерпретации,
    \correctchoice уменьшает его скорость по сравнению со схемой сцепленной интерпретации,
    \choice не влияет на скорость работы интерпретатора.
\end{choices}

\question[3] Почему редко представляется возможным при симуляции процессора разместить все гостевые регистры на физических регистрах?
\begin{solution}[2cm]
Число регистров недостаточно. Некоторые регистры могут иметь особенный смысл в хозяйской архитектуре и не допускают произвольных манипуляций.
\end{solution}

\end{questions}


% К каждой лекции должно быть от 8 до 12 задач, у каждой задачи должно быть 3-5 вариантов формулировок примерно одинаковой сложности. Допускается объединение нескольких последовательных лекций в одну тему и подготовка тестов к темам.
% Задачи должны полностью соответствовать материалам лекций, то есть лекциях должно быть достаточно информации для ответа на все вопросы.
% Формулировка каждого варианта задачи должна содержать всю необходимую информацию и не должна ссылаться на тексты внутри лекции, картинки или другие задачи или варианты задачи.
% Правильные ответы выделяются знаком «+» перед их формулировкой. Правильных ответов может быть несколько. Для тестов с несколькими ответами как минимум один ответ должен быть правильным и как минимум один ответ должен быть неправильным. 
% 
% Структура теста к лекции
% 
% \subsection*{Задача 1}
% 
% \paragraph{Вариант 1} 
% 
%     Чему равно 2+2?
%         Ответ 1. 3
%         + Ответ 2. 4
%         …
%         Ответ N. 5
% \paragraph{Вариант 2}
%     Чему равно 2*2?
%         + Ответ 1. 4
%         + Ответ 2. 2+2
%         …
%         Ответ N. 5
% \paragraph{Вариант 3}
% 
%     Чему равно 2-2?
%         Ответ 1. 0
% 
% 
%         
% \section{Просто подборка вопросов}
% 


 
 