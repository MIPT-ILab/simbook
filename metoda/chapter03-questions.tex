\section{\Questions к главе \ref{chapter03}} %\label{chapter03-questions}

\subsection*{Вариант 1}

\begin{questions}
\question[1] Какие из указанных ниже компонентов обязательны для реализации интерпретатора:
\begin{choices}
    \correctchoice декодер,
    \choice дизассемблер,
    \choice кодировщик (енкодер),
    \correctchoice блоки реализации семантики отдельных инструкций,
    \choice кэш декодированных инструкций.
\end{choices}

\question[3] Опишите, что происходит на стадии Fetch работы процессора.
\begin{choices}
    \correctchoice Чтение из памяти машинного кода, соответствующего текущей инструкции.
    \choice Определение операции и аргументов из кода инструкции.
    \choice Исполнение инструкции.
    \choice Запись результатов в память.
    \choice Продвижение указателя инструкций.
\end{choices}

\question[3] Опишите, что происходит на стадии \textbf{Writeback} работы процессора. % Для каких инструкций эта стадия будет опущена?
\begin{choices}
    \choice Чтение из памяти машинного кода, соответствующего текущей инструкции.
    \choice Определение операции и аргументов из кода инструкции.
    \choice Исполнение инструкции.
    \correctchoice Запись результатов в память.
    \choice Продвижение указателя инструкций.
\end{choices}

\question[3] Какие данные, кроме машинного кода, обычно необходимо иметь для однозначного декодирования инструкции?
\begin{choices}
    \choice Число процессоров в системе.
    \correctchoice Режим процессора.
    \choice Частота процессора.
    \choice Температура системы.
    \correctchoice Текущий адрес инструкции.
\end{choices}

\question[1] Какие эффекты могут наблюдаться при невыровненном (unaligned) чтении из памяти в существующих архитектурах:
\begin{choices}
    \correctchoice возникновение исключения,
    \correctchoice     замедление операции по сравнению с аналогичной выровненной,
    \choice данные будут считаны лишь частично,
    \choice  возможны все перечисленные выше ситуации?
\end{choices}

\question[3] Какая из следующих типов ситуаций при исполнении процессора является асинхронной по отношению к работе текущей инструкции?
\begin{choices}
    \correctchoice прерывание (interrupt),
    \choice ловушка (trap),
    \choice исключение (exception),
    \choice промах (fault)?
\end{choices}

\question[3] Выберите правильный вариант окончания фразы: Сцепленный интерпретатор работает быстрее переключаемого (switched), так как
\begin{choices}
    \correctchoice удачно использует предсказатель переходов хозяйского процессора,
    \choice кэширует недавно исполненные инструкции,
    \choice транслирует код в промежуточное представление,
    \choice не требует обработки исключений.
\end{choices}

\end{questions}

\subsection*{Вариант 2}

\begin{questions}

\question[3] Какой из типов регистров всегда присутствует во всех классических архитектурах:
\begin{choices}
\choice указатель стека,
\choice аккумулятор,
\correctchoice указатель текущей инструкции,
\choice регистр флагов,
\choice индексный регистр.
\end{choices}

\question[3] Опишите, что происходит на стадии \textbf{Decode} работы процессора.
\begin{choices}
    \choice Чтение из памяти машинного кода, соответствующего текущей инструкции.
    \correctchoice Определение операции и аргументов из кода инструкции.
    \choice Исполнение инструкции.
    \choice Запись результатов в память.
    \choice Продвижение указателя инструкций.
\end{choices}

\question[3] Опишите, что происходит на стадии \textbf{Advance PC} работы процессора. %Для каких инструкций эта стадия будет опущена?
\begin{choices}
    \choice Чтение из памяти машинного кода, соответствующего текущей инструкции.
    \choice Определение операции и аргументов из кода инструкции.
    \choice Исполнение инструкции.
    \choice Запись результатов в память.
    \correctchoice Продвижение указателя инструкций.
\end{choices}

% \question[3] Какой вид программ обычно исполняется в непривилегированном режиме процессора?
% \begin{solution}[1cm]
% Пользовательские приложения.
% \end{solution}
\question[3] Выберите верные варианты утверждений о полях инструкций.
\begin{choices}
    \correctchoice  Логическое поле может состоять из нескольких битовых полей.
    \choice Битовое поле может состоять из нескольких логических полей.
    \choice Логическое поле всегда определяется одним битовым.
    \choice Битовое поле всегда определяется одним логическим.
    \correctchoice Логические и битовые поля связаны нелинейным образом.
    \choice Логические и битовые поля связаны линейным образом.
\end{choices}

\question[3] Почему декодер, использующий единую таблицу со всеми возможными шаблонами, не используется на практике?
\begin{choices}
    \correctchoice  Размер таблицы растёт недопустимо быстро от длины опкода.
    \choice Он не позволяет декодировать инструкции с переменной длиной.
    \choice Генерация таблицы требует дополнительной ручной работы.
    \choice Невозможно декодировать префиксы с помощью этого метода.
\end{choices}

\question[3] Выберите правильные варианты окончания фразы: Наличие единственного \texttt{switch} для всех гостевых инструкций в коде интерпретатора
\begin{choices}
    \choice увеличивает его скорость по сравнению со схемой сцепленной интерпретации,
    \choice упрощает его алгоритмическую структуру по сравнению со схемой сцепленной интерпретации,
    \correctchoice уменьшает его скорость по сравнению со схемой сцепленной интерпретации,
    \choice не влияет на скорость работы интерпретатора.
\end{choices}

\question[3] Какие причины при симуляции процессора не позволяют разместить все гостевые регистры на физических регистрах?
\begin{choices}
    \choice Это приводит к замедлению симуляции.
    \correctchoice Недостаточно хозяйских регистров.
    \correctchoice Некоторые регистры могут иметь особенный смысл в хозяйской архитектуре.
    \correctchoice Ширина регистров хозяйской архитектуры может отличаться.
    \choice Различный порядок байт (endianness) архитектур.
\end{choices}

\end{questions}


% К каждой лекции должно быть от 8 до 12 задач, у каждой задачи должно быть 3-5 вариантов формулировок примерно одинаковой сложности. Допускается объединение нескольких последовательных лекций в одну тему и подготовка тестов к темам.
% Задачи должны полностью соответствовать материалам лекций, то есть лекциях должно быть достаточно информации для ответа на все вопросы.
% Формулировка каждого варианта задачи должна содержать всю необходимую информацию и не должна ссылаться на тексты внутри лекции, картинки или другие задачи или варианты задачи.
% Правильные ответы выделяются знаком «+» перед их формулировкой. Правильных ответов может быть несколько. Для тестов с несколькими ответами как минимум один ответ должен быть правильным и как минимум один ответ должен быть неправильным. 
% 
% Структура теста к лекции
% 
% \subsection*{Задача 1}
% 
% \paragraph{Вариант 1} 
% 
%     Чему равно 2+2?
%         Ответ 1. 3
%         + Ответ 2. 4
%         …
%         Ответ N. 5
% \paragraph{Вариант 2}
%     Чему равно 2*2?
%         + Ответ 1. 4
%         + Ответ 2. 2+2
%         …
%         Ответ N. 5
% \paragraph{Вариант 3}
% 
%     Чему равно 2-2?
%         Ответ 1. 0
% 
% 
%         
% \section{Просто подборка вопросов}
% 


 
 