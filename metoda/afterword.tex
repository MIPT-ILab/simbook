\chapter{Заключение}\label{afterword}

\dictum[Ian Bogost~\cite{the-great-pretender}]{A computer, it turns out, is just a particular kind of machine that works by pretending to be another machine. This is precisely what today's computers do -- they pretend to be calculators, ledgers, typewriters, film splicers, \dots \footnotemark}

\footnotetext{Компьютер, как оказывается, это просто особенный тип машины, задача которой --- притворяться другими машинами. Это то, что современные компьютеры делают --- они притворяются калькуляторами, гроссбухами, печатными машинками, звукомонтажными столами\dots}

Один из основателей той области пересечения математики и технических наук, которая в настоящее время именуется <<computer science>>, Алан Тьюринг сделал достаточно большой вклад в развитие компьютерной симуляции. Это и так называемая машина Тьюринга вкупе с тезисом Тьюринга---Чёрча, определяющие теоретическую возможность представления функциональности одной вычислительной машины через имитацию её действий на другой. Это и тест Тьюринга --- мысленный эксперимент, отделяющий понятия <<человек>> и <<разумность>> и являющийся предтечей к  организации взаимодействия реальной и виртуальной систем через <<ширму>> интерфейсов таким образом, что ни одна из них не может определить, насколько реальна вторая. За более чем полвека эти идеи развились до текущего их состояния, и теперь компьютерную симуляцию мы можем наблюдать вокруг себя ежеминутно: электронные деньги и письма вместо бумажных, текстовый процессор вместо ручки и бумаги, беспроводной телефон вместо его привязанного к розетке предшественника, онлайн-игры вместо догонялок во дворе\dots

Виртуализация настолько прочно вошла в обиход при использовании ЭВМ, что мы почти никогда не отдаём себе отчёт, что пользуемся ей. Как минимум с одной её формой мы сталкиваемся каждый раз, когда взаимодействуем с компьютером, на котором работает многозадачная ОС, --- ведь каждый процесс изолирован в своём <<контейнере>>, обеспеченный виртуальными ресурсами, скрывающими за собой реальные физические. Пользовательской программе не приходится учитывать, что одновременно с ней на процессорное время претендуют многие задачи и что значительную часть времени она может находиться в замороженном состоянии.

К сожалению, в этой книге остались незатронутыми некоторые важные темы, связанные с моделированием вычислительных систем. Среди них стоит упомянуть такие вопросы, как эффективное представление графических ускорителей внутри виртуальных окружений, разработка гибридных симуляторов, использование моделей для верификации корректности\dots{}
При этом существующие главы не настолько полны, иллюстрированы и ясно изложены, как этого хотелось бы авторам, которые, однако, не оставляют надежды приблизиться к желаемому идеалу полноты в последующих редакциях этого учебника. При этом симуляция, моделирование и виртуализация как области технического знания не стоят на месте и активно развиваются, всё больше проникая в повседневную жизнь, создавая новые возможности в работе с уже привычными вещами. % Угнаться за прогрессом, а уж тем более полностью описать состояние дел в области довольно сложно. 

Остаётся надеяться, что в процессе чтения этой книги читатель осознал универсальность и несомненную практическую ценность идеи представления вычислительной машины одного типа на компьютерах иногда совсем иной структуры, открыл для себя новые возможности привычных для него вещей, получил ответы на имевшиеся у него вопросы или же просто хорошо провёл время.
